\subsection{Паросочетания. Вершинные покрытия. Теорема Кёнига.}


Пусть дан граф G = (V,\ E), \textbf{паросочетание M} в G — это множество попарно несмежных рёбер, то есть рёбер, не имеющих общих вершин.

\textbf{Вершинным покрытием} называется такое множество вершин S, что для любого ребра хотя бы один из концов лежит в S. Нетрудно проверить, что дополнение к вершинному покрытию — независимое множество и, наоборот, дополнение к независимому множеству — вершинное покрытие. Для двудольных графов вершинные покрытия оказываются связанными с паросочетаниями.


\textbf{Теорема Кёнига}. В любом двудольном графе максимальный размер паросочетания равен минимальному размеру вершинного покрытия.

