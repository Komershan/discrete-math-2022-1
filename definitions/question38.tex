\subsection{Двудольные графы, критерий двудольности графа. Булев куб.}


 \textbf{Двудольным графом} называется неориентированный граф, в котором вершины можно разделить на две доли — левую и правую, и все рёбра соединяют вершины из разных долей (нет рёбер, соединяющих вершины одной доли). Другими словами, чтобы задать двудольный граф, надо указать два конечных множества L (левую долю) и R (правую долю) и указать, какие вершины левой доли соединены с какими вершинами правой доли. \\

\textbf{Критерий двудольности графа.} Граф является двудольным тогда и только тогда, когда не содержит в себе циклы нечетной длины. \\


\textbf{Булев куб размерности n} — это неориентированный граф, вершинами которого являются двоичные слова длины n, а рёбра соединяют слова, отличающиеся в одной позиции.