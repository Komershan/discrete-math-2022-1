\subsection{Цепи и антицепи в частично упорядоченных множествах. Теорема Дилуорса.}

Пусть $(P, \le_P)$ -- частично упорядоченное множество. Тогда:

\textbf{Цепь} -- это подмножество $P' \subseteq P$, что для любого $x, y \in P'$, элементы $x$ и $y$ сравнимы.

\textbf{Антицепь} -- это подмножество $P' \subseteq P$, что для любого $x, y \in P'$, элементы $x$ и $y$ несравнимы.

Размером цепи $P'$ назовем мощность множества $\mid P' \mid$. Аналогично определим размер для антицепи. Тогда:

\textbf{Теорема Дилуорса}: Наибольший размер антицепи в порядке равен наименьшему количеству цепей в разбиениях порядка на непересекающиеся цепи.

(работает только для конечных порядков, т.е порядков, которые определены на конечных множествах)

