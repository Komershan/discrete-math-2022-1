\subsection{Сочетания с повторениями. Количество решений уравнения $x_1 + x_2 + . . . + x_n = k$ в неотрицательных целых числах.}
\begin{itemize}
	\item \textbf{Сочетание с повторениями}
	
	Сочетанием с повторениями из $n$ элементов по $k$ называют неупорядоченный $k$-элеметный набор, в котором количество каждого элемента может быть произвольным. Их количество обозначается $\overline C^k_n$ и равно:
	$$\overline C^k_n = \begin{pmatrix}n+k-1\\k\end{pmatrix} = \begin{pmatrix}n+k-1\\n-1\end{pmatrix}$$
	
	\item \textbf{Количество решений уравнения $x_1 + x_2 + ... + x_n = k, x_i \geqslant 0, \: x_i \in \Z$}
	
	Количество решений равно $\displaystyle \begin{pmatrix}n+k-1\\n - 1\end{pmatrix}$

	\item \textbf{Количество решений уравнения $x_1 + x_2 + ... + x_n = k, \: x_i \in \N$}
	
	Количество решений равно $\displaystyle \begin{pmatrix}k-1\\n - 1\end{pmatrix}$
\end{itemize}