\subsection{Операции с частично упорядоченными множествами: сумма порядков, покоординатный порядок, лексикографический порядок.}

Пусть $(P, \le_P)$, $(Q, \le_Q)$ -- частично упорядоченные множества. Тогда:

\textbf{Покоординатный порядок} -- это такой порядок $\le_{P \times Q}$, определенный на множетсве $(P \times Q, \le_{P \times Q})$, что $(p_1, q_1) \le_{P \times Q} (p_2, q_2) \Leftrightarrow p_1 \le_P p_2$ и $q_1 \le_Q q_2$.

\textbf{Лексикографический порядок} -- это такой порядок $<_{lex}$, определенный на множестве $(P \times Q, <_{lex})$, что $(p_1, q_1) <_{lex} (p_2, q_2) \Leftrightarrow p_1 <_P p_2$ или $p_1 = p_2$ и $q_1 <_Q q_2$

\textbf{Сумма порядков}. Можно определить только для множеств $P, Q$ таких, что $P \cap Q = \emptyset$. Пусть $P + Q = P \cup Q$. Тогда на $(P + Q, \le)$ \textbf{сумма порядков} -- это такой порядок, что:

\begin{equation*}
    x \le y \Leftrightarrow
    \begin{cases}
        x, y \in P, x \le_P y \\
        x, y \in Q, x \le_Q y \\
        x \in P, y \in Q
    \end{cases}
\end{equation*}
