\subsection{Полиномиальные коэффициенты. Их алгебраический и комбинаторный смысл.}
$$(x_1 + x_2 + ... + x_k)^n = \sum_{\alpha_1+\alpha_2+...+\alpha_k=n}\begin{pmatrix}n\\\alpha_1,&\alpha_2,&...,&\alpha_k\end{pmatrix}x_1^{\alpha_1}x_2^{\alpha_2}...x_k^{\alpha_n}$$
Где $\begin{pmatrix}n\\\alpha_1,&\alpha_2,&...,&\alpha_k\end{pmatrix} = \frac{n!}{\alpha_1!\alpha_2!...\alpha_k!}$. Это число называют полиномиальным коэффициентом. 

Собственно алгебраический смысл - коэффициенты разложения суммы $(x_1 + x_2 + ... + x_k)^n$.

Комбинаторный смысл - полиномиальный коэффициент равен числу упорядоченных разбиений $n$-элементного множества на $k$ подмножеств размеров (мощностей) $\alpha_1, \alpha_2,...,\alpha_k$