\subsection{Минимальные и максимальные элементы в частичных порядках. Наибольшие и наименьшие элементы}

Пусть $(P, \le_P)$ -- частично упорядоченное множество. Тогда:

Элемент $x \in P$ называется \textbf{наименьшим}, если $\forall y \in P : x \le y$, \textbf{ при этом любой $x$ сравним с $y$}.

Элемент $x \in P$ называется \textbf{наибольшим}, если $\forall y \in P : y \le x$, \textbf{при этом любой $x$ сравним с $y$}.

Элемент $x \in P$ называется \textbf{минимальным}, если $\nexists y \in P: y < x$.

Элемент $y \in P$ называется \textbf{максимальным}, если $\nexists y \in P: x < y$.

Минимальных, максимальных элементов в множестве может быть несколько, наименьших, наибольших - не больше одного.
