\subsection{Деревья. Теорема об эквивалентных определениях дерева.}

Эквивалентные определения дерева:

\begin{enumerate}
    \item G - минимальный связный граф
    \item G - связен и $|E| = |V| - 1$
    \item в G между любыми 2 вершинами $\exists!$ простой путь
    \item G - связен и в нем нет простых циклов
\end{enumerate}

Обычно дерево обозначают через $T$.\\

\textbf{Предки} - все вершины на пути от корня до вершины, не включая саму вершину.\\

\textbf{Потомок} - вершина, которая не является предком.\\

\textbf{Лист} - вершина степени 1.