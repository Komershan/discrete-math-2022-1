\subsection{Принцип двойственности, класс самодвойственных функций, лемма о несамодвойственной функции.}

   Пусть $f(x_1, \ldots, x_n) \in P_2$. Тогда $f^*(x_1, \ldots, x_n) = \neg f(\neg x_1, \ldots, \neg x_n)$ -- \textbf{двойственная функция}.

   Пример: $(\neg x)^* = \neg(\neg(\neg x)) = \neg x$

   $(x \wedge y)^* = \neg((\neg x) \wedge (\neg y)) = x \vee y$.

   $(f^*)^* = f$

\textbf{Принцип двойственности:}  

   Пусть $f(x_1,\ldots,x_n) = f_0(f_1(x_1,\ldots,x_n),\ldots,f_k(x_1,\ldots,x_n))$. Тогда:

   $$f^*(x_1,\ldots,x_n) = f_0^*(f_1^*(x_1,\ldots,x_n),\ldots,f_k^*(x_1,\ldots,x_n))$$

 \textbf{Функция $f \in P_2$ называется самодвойственной}, если $f^* = f$.

  $S = \{f \in P_2 | f* = f\}$ -- множество всех самодвойственных функций.

  Пример: $x \in S, \neg x \in S, x \oplus y \oplus z \in S$

  \textbf{Лемма о несамодвойственной функции}:

   Пусть $f(x_1, \ldots, x_n) \notin S$. Тогда подставляя вместо переменных функции $x, \neg x$, можно получить константу.
