\subsection{Частично упорядоченные множества: строгий и нестрогий частичные порядки, их связь, линейный порядок}

Говорят, что бинарное отношение $R$, определенное на множестве $P$, является \textbf{строгим частичным порядком}, если для него выполнены такие свойства:

\hspace{0.5cm}\parbox{12cm} {
    1. $\forall a \in P, \neg aRa$ (антирефлексивность)

    2. $\forall a, b, c \in P, aRb, bRc \Rightarrow aRc$ (транзитивность)
}

Из транзитивности и антирефлексивности следует то, что отношения строгого порядка не обладают свойством симметричности ($aRb$ и $bRa$ не может выполняться, т.к тогда по транзитивности $aRb, bRa \Rightarrow aRa$, что противоречит антирефлексивности)

Обычно отношения строгого порядка обозначают как $<$.

Говорят, что бинарное отношение $R$, определенное на множестве $P$, является \textbf{нестрогим частичным порядком}, если для него выполнены такие свойства:

\hspace{0.5cm}\parbox{12cm} {
    1. $\forall a \in P, aRa$ (рефлексивность)

    2. $\forall a, b \in P$, $aRb$ и $bRa \Rightarrow a = b$ (антисимметричность)

    3. $\forall a, b, c \in P$, $aRb$, $bRc \Rightarrow aRc$ (транзитивность)
}

Обычно отношения нестрогого порядка обозначают как $\le$

\textbf{Связь строгого и нестрогого частичных порядков:} Из отношения не строгого порядка на $P$ можно получить отношение строгого порядка на $P$ и наоборот следующим образом

\begin{center}
    $a \le b \Leftrightarrow a < b$ или $a = b$

    $a < b \Leftrightarrow a \le b$ и $a \neq b$
\end{center}

Множество $P$ называется \textbf{частично упорядоченным}, если на нем определен порядок $R$.

Обозначается как $(P, <_P)$ или $(P, \le_P)$ для строгого и нестрогого порядков соответственно.

\textbf{Линейный порядок} -- это такой порядок $(P, \le_P)$, что для любых элементов $x, y \in P$, $x \le y$ или $y \le x$. Иначе говоря, в линейном порядке любые два элемента сравнимы.
