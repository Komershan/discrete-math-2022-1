\subsection{Бесконечно убывающие цепи. Фундированные множества. Принцип математической индукции для фундированных множеств}

Пусть $(P, <_P)$ -- частично упорядоченное множество. Тогда:

\textbf{Цепь} -- это подмножество $P' \subseteq P$, что для любого $x, y \in P'$, элементы $x$ и $y$ сравнимы.

\textbf{Бесконечно убывающей цепью} обозначим последовательность элементов порядка $x_1 > x_2 > \ldots$, в котором каждый элемент меньше предыдущего ($x_i < x_{i-1}$). (это тоже цепь, просто задаем её как последовательность)

Порядок $<_P$ называется \textbf{фундированным}, если (несколько определений): 1. Любая убывающая цепь в нем конечна или 2. Каждое непустое подмножество $S \subseteq P$ имеет минимальный элемент. Множество, на котором определен данный порядок, также называется фундированным.

Для фундированных множеств выполняется принцип математической индукции: если для утверждения $A(p)$, зависящего от элемента порядка, для любого $p$ верно утверждение «если $A(q)$ верно при всех $q < p$, то и $A(p)$ верно». Тогда утверждение $A(p)$ верно при любом $p \in P$.
