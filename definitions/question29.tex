\subsection{Бесконечно убывающие цепи. Фундированные множества. Принцип математической индукции для фундированных множеств}

Пусть $(P, <_P)$ -- частично упорядоченное множество. Тогда:

\textbf{Цепь} -- это подмножество $P' \subseteq P$, что для любого $x, y \in P'$, элементы $x$ и $y$ сравнимы.

\textbf{Бесконечно убывающей цепью} обозначим бесконечную последовательность элементов порядка $x_1 > x_2 > \ldots$, в котором каждый элемент меньше предыдущего ($x_i < x_{i-1}$). (это тоже цепь, просто задаем её как последовательность). В бесконечно убывающей цепи нет минимума, т.к иначе последовательность $x_i$ содержала бы конечное число элементов.

Порядок $P$ называется \textbf{фундированным}, если для него выполнено одно из следующих свойств:

\hspace{0.5cm}\parbox{17cm} {
    1. каждое непустое подмножество имеет минимальный элемент

    2. любая убывающая цепь конечна

    3. Для порядка $P$ спраедлив принцип индукции: если для утверждения $A(p)$, зависящего от элемента порядка, для любого $p$ верно утверждение «если $A(q)$ верно при всех $q < p$, то и $A(p)$ верно». Тогда утверждение $A(p)$ верно при любом $p \in P$.
}


Для фундированных множеств выполняется принцип математической индукции: если для утверждения $A(p)$, зависящего от элемента порядка, для любого $p$ верно утверждение «если $A(q)$ верно при всех $q < p$, то и $A(p)$ верно». Тогда утверждение $A(p)$ верно при любом $p \in P$.
