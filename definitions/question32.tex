\subsection{Ориентированные и неориентированные графы. Степени вершин. Лемма о рукопожатиях. Понятия пути, цикла, простого пути, простого цикла.}
\textbf{Неориентированный граф} - пара множества вершин и множества ребер.

$G = (V, E)$, $|V| < \infty$.

$E \subseteq \left\{ {a, b} | a, b \in V, \ a \neq b\right\}$\\

\textbf{Ориентированный граф} - пара множества вершин и множества ребер.

$G = (V, E)$, $|V| < \infty$.

$E \subseteq \left\{ (a, b) | a, b \in V, \ a \neq b\right\}$\\

\textbf{Степень вершины} - количество ребер исходящих из вершины.

Для неориентированного графа:

$\deg (v) = |\{e \in E | v \in e\}|$\\

Для ориентированного графа:

$\deg_{+} (v) = |\{(v, a) \in E | a \in V\}|$

$\deg_{-} (v) = |\{(b, v) \in E | b \in V\}|$\\

\textbf{Лемма о рукопожатиях}

Для неориентированного графа:

$\sum\limits_{v \in V} \deg (v) = 2 |E|$\\

Для ориентированного графа:

$\sum\limits_{v \in V} \deg_{+} (v) = \sum\limits_{v \in V} \deg_{-} (v) = |E|$\\

\textbf{Смежные вершины.} Вершины $v_1, v_2$ называются смежными, если $\exists e \in E : e = \{v_1, v_2\}$.\\

\textbf{Путь} - последовательность смежных вершин. $(v_1, v_2, v_3, ..., v_n)$\\

\textbf{Простой путь} - путь, в котором все вершины различны.\\

\textbf{Цикл} - путь, у которого первая и последняя вершины одинаковы.\\

\textbf{Простой цикл} - путь, у которого совпадают только первая и последняя вершины, длины больше или равной 3.\\

\textbf{Длина пути} - количество вершин в пути - 1.