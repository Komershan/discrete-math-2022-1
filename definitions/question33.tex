\subsection{Отношение достижимости и компоненты связности графа. Неравенство, связывающее число вершин, ребер и компонент связности в графе. Компоненты сильной связности ориентированного графа.}
\textbf{Отношение достижимости.} Вершина $u$ достижима из вершины $v$, если $\exists$ путь из $v$ в $u$. Так же говорят,
что вершины $v$ и $u$ - связны ($u \sim v$). Отношение достижимости называют отношением связности.\\

\textbf{Отношение сильной связности.} $u$ и $v$ - сильно связны, если $\exists$ ориентированный путь $u - v$
и $\exists$ ориентированный путь $v - u$.\\

\textbf{Компонента связности графа.} Так как отношение связности является отношением эквивалентности, то множество вершин
можно разбить на компоненты - компоненты связности.\\

\textbf{Неравенство, связывающее число вершин, ребер и компонент связности в графе.}

Количество компонент связности $\geq |V| - |E|$\\

\textbf{Компоненты сильной связности ориентированного графа.} Так как отношение сильно связности является отношением эквивалентности,
то множество вершин ориентированного графа можно разбить на компоненты - компоненты сильной связности.