\subsection{Формулы, полные системы связок, примеры. Дизъюнктивная нормальная форма, СДНФ.}

\textit{Связка -- это любая булева функция. Вроде как точно связку не определяют, тем не менее, под связками понимают именно булевы функции}

Пример множества связок: $F = \{\neg, \wedge, \vee\}$.

Пусть $F$ это множество связок. Тогда, функция $f: \{0, 1\}^n \rightarrow \{0, 1\}$ \textbf{выразима в системе связок} $F$, если $\exists$ формула $\varphi$ под данной системой $F$ (или $f$ можно выразить через функции системы связок $F$):

$$\forall (x_1, \ldots, x_n) \in \{0, 1\}^n: f(x_1, \ldots, x_n) = \varphi(x_1, \ldots, x_n)$$

Формула $\varphi$ строится последовательно:

\hspace{0.5cm}\parbox{11cm}{
       1. Переменная $x_i$ сама по себе является формулой

       2. Переменная $g(\varphi_1, \ldots, \varphi_n)$, где $g \in F$ и $\varphi_1, \varphi_2, \ldots, \varphi_n$ формулы -- тоже формула.

       3. Если $\varphi(x_1, x_2, \ldots, x_n)$ -- формула, то $\varphi(x_1, x_2, \ldots, x_n, x_{n + 1})$ тоже формула (где $x_{n + 1}$ фиктивная переменная, так мы умеем расширять количество аргументов у формулы).
}

Константы по умолчанию не являются формулами, их надо выражать из связок.

$[F]$ -- множество всех булевых функций, выразимых в $F$ (или \textbf{замыкание} $F$)

$F$ -- \textbf{полная система связок}, если $[F]$ -- все булевы функции ($P_2$).

Пусть $x^a = x$ если $a = 1$ и $\neg x$ если $a = 0$. Тогда:

\textbf{Коньюнкт} -- $x_1^{a_1} \wedge x_2^{a_2} \wedge \ldots \wedge x_k^{a_k}$

\textbf{Дизьюнктивная Нормальная Форма (ДНФ)} -- представление функции $f(x_1, x_2, \ldots, x_n)$ как дизьюнкции коньюнктов.

\textbf{Пример:} для функции $(A \vee B) \wedge (C \vee \neg D)$, ДНФ -- $A^1 \wedge C^1 \vee A^1 \wedge D^{0} \vee B^1 \wedge C^1 \vee B^1 \wedge D^{0}$
