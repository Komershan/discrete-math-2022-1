\subsection{Эйлеровы циклы в ориентированных и неориентированных графах. Критерий существования эйлерова цикла.}

Цикл (в неориентированном или ориентированном графе) называется эйлеровым, если он проходит по всем рёбрам графа ровно по одному разу (любое ребро соединяет соседние вершины в цикле, и никакое ребро не делает это дважды). \\ 

Граф называется эйлеровым, если в нём есть эйлеров цикл. \\

Есть простой критерий эйлеровости графов и орграфов. Прежде всего заметим, что добавление и удаление изолированных вершин, то есть тех вершин, из которых
не выходит и в которые не входит ни одного ребра, не изменяет свойство эйлеровости графа. \\

\textbf{Теорема 1.} В ориентированном графе без изолированных вершин существует эйлеров цикл тогда и только тогда, когда граф сильно связен и у любой вершины входящая степень равна исходящей

\textbf{Теорема 2.} Неориентированный граф без вершин нулевой степени содержит эйлеров цикл тогда и только тогда, когда он связен и степени всех вершин чётны.
