\subsection{LYM-лемма, теорема Шпернера о размере максимальной антицепи в булевом кубе.}
\textbf{Отношение порядка на булевом кубе.} Вершины булева куба - двоичные слова, тогда, если слово $x$ является подсловом $y$ (с точки зрения единиц), то $x \leq y$ (покоординатное сравнение).

\textbf{LYM-лемма}, или \textit{LYM-inequality}. Дан булев куб, пусть $A$ в нем - антицепь, $a_k$ - количество элементов в антицепи, в которых ровно $k$ единиц. Тогда утверждается, что выполнено:
\[
\sum_{k = 0}^{n} \frac{a_k}{C_{n}^{k}} \leq 1 
\]
\noindent \textbf{Теорема Шпернера.} Длина максимальной антицепи в булевом кубе равна $C_{n}^{[\frac{n}{2}]}.$ \\
