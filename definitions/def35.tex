\subsection{Полное двоичное дерево. Остовное дерево в графе.}
\textbf{Полное двоичное дерево} - дерево, где каждой вершине можно присвоить булевый кортеж и тогда все вершины
будут представимы в виде $\bigcup\limits_{k = 0}^{n} \{0, 1\}^{k}$. Тогда ребра будут между вершинами $a_1, ..., a_k$ и
$a_1, ..., a_k, a_{k + 1}$.

В полном двоичном дереве $2^n$ листьев.\\

% TODO тут картинку надо сделать с деревом

\textbf{Остовное дерево в графе.} Дан граф $G = (V, E)$. Тогда остовное дерево в G - это $T = (V, E'), E' \subseteq E, T$ - дерево.