\subsection{Отношения эквивалентности. Классы эквивалентности.}

Отношение $R$ на $A$ называют:

\textbf{Рефлексивным}, если $\forall a \in A \colon aRa$.

\textbf{Симметричным}, если $\forall a, b \in A \colon aRb \Leftrightarrow bRa$.

\textbf{Транзитивным}, если $\forall a, b, c \colon aRb \wedge bRc \Rightarrow aRc$.

Пример: отношение $a < b$ транзитивно, но не рефлексивно и не симметрично. Отношение $a + b = a \cdot b$ симметрично, но не рефлексивно и не транзитивно.

Отношение $R$ на $A$ называют \textbf{отношением эквивалентности}, если отношение $R$ рефлексивно, симметрично и транзитивно.

Пример: Отношение $a = b$ -- рефлексивно ($\forall a \in A \colon a = a$), симметрично ($\forall a, b \in A \colon a = b \Rightarrow b = a$), транзитивно ($\forall a, b, c \in A \colon a = b. b = c \Rightarrow a = c$).

Если $R$ на $A$ -- отношение эквивалентности, то множество $A$ можно разбить на классы эквивалентности $A_i$

\textbf{Классы эквивалентности} -- это разбиение множества $A$ отношением эквивалентности $R$ на непересекающиеся классы $(\forall i \neq j \colon A_i \cap A_j = \varnothing,\; \cup_{i \in I} A_i = A)$ такое, что $\forall x, y \in A_i \colon xRy$ и $\forall x \in A_i, y \in A_j, i \neq j: \neg xRy$. (то есть если два элемента принадлежат одному классу эквивалентности, они находятся в отношении $R$ и наоборот).
