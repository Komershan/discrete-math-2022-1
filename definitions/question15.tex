\subsection{Полином Жегалкина. Теорема о представлении булевой функции полиномом Жегалкина.}

\textbf{Моном} -- это выражение вида $x_{i_1} \wedge x_{i_2} \wedge ... \wedge x_{i_k}$.

(0 и 1 -- тоже мономы)

\textbf{Полином Жегалкина} -- многочлен вида $\bigoplus\limits_{(i_1,\ldots,i_k), k = 0 \ldots n} a_{i_1\ldots i_k} \wedge x_{i_1} \wedge x_{i_2} \wedge \ldots \wedge x_{i_k}$

Пример: $1 \oplus (x \wedge y) \oplus (x \wedge y \wedge z)$

\textbf{Теорема о представлении булевой функции полиномом Жегалкина:} каждую булеву функцию можно однозначно представить в виде полинома Жегалкина.
