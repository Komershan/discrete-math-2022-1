\subsection{Задание булевых функций таблицами истинности. Количество булевых функций от $n$ переменных}

\begin{itemize}
	\item \textbf{Задание булевых функций таблицами истинности}

	Любую булеву функцию можно задать с помощью таблицы истинности. Выглядит она следующим образом: в каждой строке перечисляются значения набора переменных $x_1, x_2, \ldots, x_n$, после чего перечисляется значение функции от данного набора переменных.
	
	Пример для базовых булевых операций:

   \begin{tabular}{|c|c|c|c|c|c|c|}
       \hline
       $x$ & $y$ & $\wedge$ & $\vee$ & $\oplus$ & $\to$ & $\equiv$\\
       \hline
       0 & 0 & 0 & 0 & 0 & 1 & 1 \\
       \hline
       0 & 1 & 0 & 1 & 1 & 1 & 0 \\
       \hline
       1 & 0 & 0 & 1 & 1 & 0 & 0 \\
       \hline
       1 & 1 & 1 & 1 & 0 & 1 & 1 \\
       \hline
   \end{tabular}

	\item \textbf{Количество булевых функций}
	
	Количество булевых функций от $n$ переменных равно $2^{2^n}$. Следует из того, что различных наборов на $n$ переменных всего $2^n$, ну и для каждого набора мы можем выбрать значение 0 или 1
\end{itemize}