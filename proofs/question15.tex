\subsection{Подмножество счетного множества конечно или счетно. Во всяком бесконечном множестве есть счетное подмножество. Объединение конечного или счётного числа конечных или счётных множеств конечно или счётно. Декартово произведение конечного числа счетных множеств счетно. Счетность множества конечных последовательностей натуральных чисел.}

\begin{itemize}
	\item \textbf{Подмножество счетного множества конечно или счетно.}\\

Пусть B - счетно. $A \subseteq B$, тогда A - счетно или конечно.\\

\noindent \textbf{Доказательство:} \\

Так как B - счетно, то занумеруем все элементы из B и выпишем их в ряд. Теперь вычеркнем все элементы из $B \backslash A$.

\begin{equation*}
    B : \not b_1, b_2, b_3, \not b_4, ..., b_5, \not b_6, ...
\end{equation*}

Остались только элементы из A и это все элементы A, значит мы занумеровали все элементы из A. Чтд\\

\textbf{Во всяком бесконечном множестве есть счетное подмножество.}\\

Если A - бесконечное множество, то $\exists B \subseteq A$, что $B$ - счетно.\\

\noindent \textbf{Доказательство:}\\

$\exists a_1 \in A \Rightarrow B_1 = \{a_1\}$\\

$\exists a_2 \in A \backslash B_1 \Rightarrow B_2 = \{a_1, a_2\}$\\

$\exists a_3 \in A \backslash B_2 \Rightarrow B_3 = \{a_1, a_2, a_3\}$\\

...\\

$\exists a_k \in A \backslash B_{k - 1} \Rightarrow B_k = \{a_1, a_2, ..., a_k\}$\\

$B = \bigcup\limits_{i = 1}^{\infty} B_i$, очевидно, что B - счетно. Чтд\\

\item \textbf{Объединение конечного или счётного числа конечных или счётных множеств конечно или счётно.}

Пусть нам дано не более чем счетное количество множеств $A_1, A_2, ..., A_n, ...$. Тогда докажем, что их объединение -
не более, чем счетно.

\noindent \textbf{Доказательство:}

Выпишем в столбец все множества $A_1, A_2, ...$, так можно, так как их не более чем, счетно. В строку выпишем элементы
этих множеств.

\begin{equation*}
    \begin{array}{c|cccc}
        A_1 & a_{11} & a_{12} & a_{13} & ...\\
        A_2 & a_{21} & a_{22} & a_{23} & ...\\
        A_3 & a_{31} & a_{32} & a_{33} & ...\\
        \vdots & ... & ... & ... & \ddots\\
    \end{array}
\end{equation*}

Теперь будем набирать элементы по диагоналям, сначала берем с первой, потом со второй и тд. Так мы получим все элементы из A.
И они будут занумерованы. Если какие-то элементы совпали, то их можно просто пропустить.

\begin{equation*}
    A = a_{11}, a_{21}, a_{12}, a_{31}, a_{22}, a_{13}, ...
\end{equation*}

Ну или можно представить это в виде

\begin{equation*}
    A = \bigcup\limits_{i = 2}^{\infty} \bigcup\limits_{j = 1}^{i - 1} a_{j (j - i)}
\end{equation*}

Значит A - счетно. Чтд\\

\item \textbf{Декартово произведение конечного числа счетных множеств счетно.}

Сначала докажем, что если A, B - счетны. То $A \times B$ - тоже счетно.

\noindent \textbf{Доказательство:}
\begin{equation*}
    A \times B = \left\{ (a, b) | a \in A, b \in B \right\} = \bigcup\limits_{i = 1}^{\infty} \underbrace{A \times \{b_i\}}_{\text{счетное множество}}
\end{equation*}
Но очевидно, что $A \times \{b_i\}$ - счетное множество, так как это просто множество A, к каждому элементу в котором приписали $b_i$. Значит
$A \times B$ - счетное объединение счетных множеств, значит оно счетно.

Но раз $A \times B$ - счетно, то перейдя к равномощным $\N \times \N = \N^2$ - тоже счетно, значит можно по индукции доказать,
что $\forall k \ \ \N^k$ - счетно. Чтд\\

\item \textbf{Счетность множества конечных последовательностей натуральных чисел.}

Пусть $n$ - длина максимальной последовательности, значит такое множество можно представить в виде $\bigcup\limits_{k = 1}^{n} \N^k$.

$\bigcup\limits_{k = 1}^{n} \N^k$ - счетно, так как это счетное объединение счетных множеств. Кстати, тут
$\N^k$ - можно считать за все слова длины $k$ в алфавите $\N$.

\end{itemize}