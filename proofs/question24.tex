\subsection{LYM-лемма, теорема Шпернера о размере максимальной антицепи в булевом кубе.}

\textbf{LYM-лемма}, или \textit{LYM-inequality}. Дан булев куб, пусть $A$ в нем - антицепь, $a_k$ - количество элементов в антицепи, в которых ровно $k$ единиц. Тогда утверждается, что выполнено:
\[
\sum_{k = 0}^{n} \frac{a_k}{C_{n}^{k}} \leq 1 
\]

\noindent \textbf{Доказательство:}

Посчитаем количество цепей максимальной длины двумя способами. Для начала разберемся какой длины максимальная цепь. Будем рассматривать элементы цепи в порядке увеличения. Тогда если после $x$ идет $y$, то $x$ - подслово $y$, это значит, что в $y$ единицы обязательно в тех же местах что и в $x$ + хотя бы еще одна в других местах. Каждый раз количество единиц в вершины строго увеличивается, а значит, чтобы достичь цепь максимальной длины, нужно увеличивать вес(количество единиц) вершины на 1. Получаем, что максимальная длина цепи $n + 1$. \\

Посчитаем первым способом количество цепей максимально длины. Чтобы дойти от $00...0$ до $11...1$. Нам нужно вставить в каком-то порядке $n$ единиц, причем каждый порядок задает свою цепь. Получаем, что у нас $n!$ вариантов последовательно вставить единицы, а значит и $n!$ цепей. \\

Посчитаем вторым способом. Зафиксируем какую-то вершину куба $x$, вес которой $k$. Сколько цепей максимальной длины проходит через нее? По тем же соображениям $k! \cdot (n - k)!$, потому что нам нужно каким-то порядком сначала поставлять $k$ заданных единиц, а потом дойти из $x$ до $11...1$, проставив уже $n - k$ единиц. 

Тогда сколько цепей максимальной длины проходит через вершины антицепи $A$? Заметим тот факт, что через каждую вершину проходят свои уникальные цепи. Пусть это не так, тогда $x_1$ и $x_2$ находятся в одной цепи, значит их можно сравнить, значит они не могут быть в одной антицепи. Раз через каждую вершину проходят уникальные цепи максимальной длины, можно выписать неравенство:
\[
\sum_{k = 0}^{n} a_k \cdot k! \cdot (n - k)! \leq n!
\]
то есть количество уникальных цепей максимальный длины, проходящих через вершины антицепи $A$ не превосходит общего количества цепей максимальной длины. Делим неравенство на правую сторону, получаем то, что и требовалось доказать:
\[
\sum_{k = 0}^{n} \frac{a_k}{C_{n}^{k}} \leq 1
\]
	\\ \\ \\

\noindent \textbf{Теорема Шпернера.} Длина максимальной антицепи в булевом кубе равна $C_{n}^{[\frac{n}{2}]}.$ \\

\noindent \textbf{Лемма}, что $max_{0 \leq k \leq n} C_{n}^{k} = C_{n}^{[\frac{n}{2}]}$. Будет использоваться, но не доказываться. \\

\noindent \textbf{Доказательство:} 

Возьмем, то, что мы получили в LYM-лемме и воспользуемся нашей локальной леммой, получим:
\[
1 \geq \sum_{k = 0}^{n} \frac{a_k}{C_{n}^{k}} \geq \sum_{k = 0}^{n} \frac{a_k}{C_{n}^{[\frac{n}{2}]}} \Rightarrow \sum_{k = 0}^{n} a_k \leq C_{n}^{[\frac{n}{2}]}
\]
правая часть неравенства не что иное, как количество элементов в антицепи $A$. \\

Доказали, что небольше, как найти пример, где ровно. Посмотрим на все вершины весом $[\frac{n}{2}]$. Очевидно, что они все несравнимы, а их количество как раз равно $C_{n}^{[\frac{n}{2}]}$. Что и требовалось доказать.

