\subsection{Паросочетания. Вершинные покрытия. Теорема Кёнига}

\textbf{Теорема Кёнига} В любом двудольном графе максимальный размер паросочетания равен минимальному размеру вершинного покрытия. \\

\noindent \textbf{Доказательство:} 

В одну сторону легко. Если P - паросочетание в двудольном графе $G =  (L, R, E)$, то любое вершинное покрытие содержит хотя бы по одному концу каждого ребра паросочетания и поэтому его размер не меньше размера паросочетания. Значит, минимальный размер вершинного покрытия не меньше максимального размера паросочетания. (Факт верен для любых графов) \\

Теперь в другую сторону (тут уже верно только для двудольных): рассмотрим минимальное по размеру вершинное покрытие $X \sqcup Y, X \subseteq L, Y \subseteq R$, в графе G. Проверим выполнение условия теоремы Холла для ограничения $G_{X, G(X) \setminus Y}$ графа на множество вершин $X$ в левой доле и множество вершины $G(X) \setminus Y$ в правой доле (оставляем в $G_{X, G(X) \setminus Y}$ только рёбра между указанными вершинами). Пусть $S \subseteq X$. \\

Множество $(X \setminus S) \sqcup Y \sqcup G_{X, G(X) \setminus Y}(S)$ является вершинным покрытием в $G$: все рёбра, покрытые вершинами из S, покрыты также либо вершинами из Y, либо соседями вершин из S в правой доле. Поскольку мы выбрали минимальное по размеру вершинное покрытие, $|G_{X, G(X) \setminus Y}G_{X, G(X) \setminus Y}(S)| \geq |S|$, что и означает выполнение условия теоремы Холла. \\

Аналогично проверяется выполнение условия теоремы и для графа $G_{L \setminus X, Y}$, полученного ограничением $G$ на вершины $L \setminus X$ в левой доле и Y в правой доле (так как $X \sqcup Y$ - вершинное покрытие исходного графа, $L \setminus X$ входит в множество соседей Y в левой доле). \\

По теореме Холла в $G_{X, G(X) \setminus Y}$ есть паросочетание размера $|X|$, а в $G_{L \setminus X, Y} $ есть паросочетание размера $|Y|$. Рёбра этих паросочетаний не совпадают по построению. Значит, объединение этих паросочетаний даёт паросочетание размера $|X| + |Y|$ в графе $G$. Таким образом, размер максимального паросочетания в G не меньше размера минимального вершинного покрытия. \\