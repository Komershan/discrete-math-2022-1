\subsection{Эквивалентность различных определений деревьев: число вершин и число ребер, минимально связные графы, графы без простых циклов, графы с единственностью простых путей. Существование остовного дерева в связном графе.}
\begin{itemize}
	\item \textbf{Эквивалентность различных определений деревьев: число вершин и число ребер, минимально связные графы, графы без простых циклов, графы с единственностью простых путей}
		\begin{itemize}
		    \item
		    Существует 4 эквивалентных определения графа $G$, который является деревом:
		    \begin{enumerate}
		    \item $G$ связен и $|E| = |V| - 1$, то есть количество ребер на один меньше, чем количество вершин
		    \item $G$ - минимально связный граф
		    \item $G$ связен и в нем нет циклов
		    \item в $G$ между любыми вершинами $v, u$ существует единственный простой путь
		    \end{enumerate}
		
		    \item Покажем, что все 4 определения эквивалентны.
		    \begin{itemize}
		    \item
		    Докажем, что из \textbf{1} следует \textbf{2}. Воспользуемся неравенством, что кол-во компонент связности больше, либо равно $(|V| - |E|)$. В нашем случае, $|V| - |E| = 1$, поэтому при удалении любого ребра, $|V| - |E|$ станет равно $2$, то есть кол-во компонент связности станет строго больше 1. Отсюда следует, что наш граф - минимальный связный граф.
		
		    \item
		    Докажем, что из \textbf{2} следует \textbf{3}. Предположим противное - в нашем графе есть цикл. Тогда можно удалить любое ребро из этого цикла, и граф все ещё останется связным в силу того, что удаленное ребро можно "компенсировать оставшейся частью цикла". Выходит, наш граф не минимально связный - противоречие.
		
		    \item
		    Докажем, что из \textbf{3} следует \textbf{4}. Предположим противное - между двумя вершинами $v$ и $u$ существует два простых пути. Найдем первую вершину, после которой эти пути отличаются, пусть это будет вершина $a$. Таким образом, первый и второй путь начинаются с одинаковой части $v \to a$. Найдем первую вершину после вершины $a$, где они совпадают, пусть это будет вершина $b$. Такие вершины обязательно найдутся, к примеру, в качестве вершины $a$ подойдет вершина $v$, в качестве вершины $b$ - вершина $u$. Заметим, что участки путей между вершинами $a$ и $b$ не пересекаются, выходит мы нашли цикл на вершинах $a$ и $b$ - противоречие.
		
		    \item
		    Докажем, что из \textbf{4} следует \textbf{1}. Для начала заметим такой факт. Если мы добавляем ребро $(v; u)$ в граф, при этом они уже находятся в одной компоненте связности, то между вершинами $v$ и $u$ будет хотя бы два простых различных пути. Первый - по ребру $(v; u)$. Второй - вершины $v$ и $u$ были в одной компоненте связности, поэтому между ними существовал какой-либо простой путь. Вернемся к нашему доказательству, выкинем все ребра из нашего и будем их по очереди туда добавлять. Заметим, что между любыми двумя вершинами $v$ и $u$ существует единственный простой путь. Поэтому каждый раз, когда мы добавляем новое ребро, оно должно иметь концы в разных компонентах связности. В начале процесса граф состоял из $|V|$ компонент связности. При добавлении любого ребра, кол-во компонент уменьшается на 1. Отсюда следует, что мы сможем добавить не более $|V| - 1$ ребра. Но ведь наш граф связный, поэтому в нём одна компонента связности, выходит мы обязаны добавить хотя бы $|V| - 1$ ребер. Отсюда - в нашем графе должно быть ровно $|V| - 1$ ребер.
		    \end{itemize}
		\end{itemize}

	\item \textbf{Существование остовного дерева в связном графе}
		\begin{itemize}
		    \item Докажем, что в любом связном графе существует остовное дерево. Будем удалять из нашего графа ребра, при удалении которых не увеличивается количество компонент связности. Заметим, что если таких ребер не осталось, то мы имеем минимальный связный граф. Отсюда следует, что получившийся граф - \textbf{остовное дерево}.
		\end{itemize}

\end{itemize}