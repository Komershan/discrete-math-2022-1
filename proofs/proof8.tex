\subsection{Формулы, полные системы связок. Полнота системы связок «конъюнкция, дизъюнкция, отрицание». Дизъюнктивная нормальная форма, СДНФ.}
\textbf{Теорема 4.6}: $\{\neg, \wedge, \vee\}$ -- полная система связок\\

\noindent \textbf{Доказательство}:\\

Выразим функции $f$, равные единице только на одном конкретном наборе. Пусть такая функция $f(x_1, x_2, \ldots, x_n)$ равна единице на наборе $a_1, a_2, \ldots, a_n$. Тогда $f = y_1 \wedge y_2 \wedge \ldots \wedge y_n$, где $y_i = \neg x_i$, если $a_i = 0$ и $y_i = x_i$ иначе.\\

Обозначим за $x^a = x$ если $a = 1$ и $\neg x$ если $a = 0$.\\

То есть $f_{a_1, a_2, \ldots, a_n}(x_1, \ldots, x_n) = x_1^{a_1} \wedge x_2^{a_2} \wedge \ldots \wedge x_n^{a_n}$ -- функция, которая принимает 1 только на наборе $a_1, a_2, \ldots, a_n$.\\

Пусть $f$ принимает 1 на некоторых наборах.\\

Тогда $f = \bigvee\limits_{(a_1, \ldots, a_n) \in \{0, 1\}^n : f(a_1, a_2, \ldots, a_n) = 1} f_{a_1, a_2, \ldots, a_n} = \bigvee\limits_{(a_1, \ldots, a_n): f(a) = 1}x_1^{a_1}x_2^{a_2}\ldots x_n^{a_n}$\\

Частный случай: тождественный ноль, мы можем его выразить как $x_1 \wedge \neg x_1$.\\

Вообще, такое представление функции имеет название СДНФ или совершенная дизьюнктивная нормальная форма.\\