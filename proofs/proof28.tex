\subsection{Эйлеровы циклы в ориентированных и неориентированных графах. Критерий существования эйлерова цикла.}
	
\textbf{Определение:} \\
Цикл называется эйлеровым, если он проходит по всем рёбрам графа по одному
разу (любое ребро входит в цикл, и никакое ребро не входит дважды).
\\ \\

\noindent \textbf{Критерий существования:}

\textit{Неориентированный граф} без вершин нулевой степени содержит эйлеров цикл тогда и только тогда, когда он связен и степени всех вершин чётны. \\ 

\textit{Ориентированный граф} без вершин нулевой степени (в которые не
входит и из которых не выходит рёбер) содержит эйлеров цикл тогда и только
тогда, когда он сильно связен и у любой вершины входящая степень равна исходящей.	\\ \\ 

\noindent \textbf{Доказательство:} \\
\indent Будем доказывать параллельно оба варианта теоремы. Пусть сначала эйлеров цикл есть. Тогда он проходит через все вершины (поскольку они имеют ненулевую степень), и по нему можно дойти от любой вершины до любой. Значит, граф связен (сильно связен в ориентированном случае). \\

Теперь про степени. Возьмём какую-то вершину $v$, пусть она встречается в цикле
$k$ раз. Идя по циклу, мы приходим в неё k раз и уходим k раз, значит, использовали
$k$ входящих и $k$ исходящих рёбер. При этом, раз цикл эйлеров, других рёбер у этой
вершины нет, так что в ориентированном графе её входящая и исходящая степени
равны k, а в неориентированном графе её степень равна $2k$. Таким образом, в одну
сторону критерий доказан. \\ 

Рассуждение в обратную сторону чуть сложнее. Будем рассматривать пути, которые не проходят дважды по одному ребру. (Таков, например, путь из одного ребра.) Выберем среди них самый длинный путь
\[
a_1 \to a_2 \to a_3 \to \dots \to a_{n - 1} \to a_n
\]
и покажем, что он является искомым циклом, то есть что $a_1 = a_n$ и что он содержит
все рёбра. \\ 

В самом деле, если он самый длинный, то добавить к нему ребро $a_n \to a_{n +1}$
уже нельзя, то есть все выходящие из an рёбра уже использованы. Это возможно,
лишь если $a_1 = a_n$. В самом деле, если вершина an встречалась только внутри пути
(пусть она входит $k$ раз внутри пути и ещё раз в конце пути), то мы использовали
$k + 1$ входящих рёбер и $k$ выходящих, и больше выходящих нет. Это противоречит
равенству входящей и исходящей степени (в ориентированном случае) или чётности
степени (в неориентированном случае). \\ 

Итак, мы имеем цикл, и осталось доказать, что в него входят все рёбра. В самом
деле, если во всех вершинах цикла использованы все рёбра, то из вершин этого цикла
нельзя попасть в вершины, не принадлежащие циклу, то есть использованы все
вершины (мы предполагаем, что граф связен или сильно связен) и, следовательно,
все рёбра. С другой стороны, если из какой-то вершины $a_i$ выходит ребро $a_i \to v$,
то путь можно удлинить до 
\[
a_i \to a_{i + 1} \to \dots \to a_n = a_1 \to a_2 \to \dots \to a_i \to v
\]
вопреки нашему выбору (самого длинного пути). Аналогично можно получить противоречие и для входящего ребра v → ai
, добавив его в начало. (А можно заметить,
что если есть неиспользованное входящее ребро, то есть и неиспользованное выходящее.) Это рассуждение было для ориентированного случая, но в неориентированном
всё аналогично. Теорема доказана. \\

Помимо эйлеровых циклов, можно рассматривать \textit{эйлеровы пути} — пути в графе, которые проходят один раз по каждому ребру. (Для неориентированных графов:
рисуем картинку, не отрывая карандаша от бумаги, но не обязаны вернуться в исходную точку.) Для них тоже есть критерий: в неориентированном случае нужно,
чтобы граф был связен и было не более двух вершин нечётной степени.