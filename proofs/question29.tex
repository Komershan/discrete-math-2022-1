\subsection{Двудольные графы, критерий двудольности графа. Пример: булев куб.}

\textbf{Определение:}

Двудольным графом называется неориентированный граф, в котором можно разбить вершины на две доли — левые и правые, что все рёбра соединяют вершины из
разных долей (нет рёбер, соединяющих вершины одной доли). \\ \\ 

\noindent \textbf{Критерий двудольности:}

Граф двудольный тогда и только тогда, когда он двураскрашиваемый, то есть не содержит циклов нечетной длины.

Очевидно доказать экививалентность утверждений граф двудольный и граф двураскрашиваемый, так что приведем доказательство того, что в двураскрашиваемом графе нет циклов нечетной длины. \\

\noindent \textbf{Доказательство:}

Докажем сначала, что в двураскрашиваемом графе нет циклов нечётной длины.
По контрапозиции, это условие равносильно тому, что если в графе есть цикл нечётной длины, то
его нельзя раскрасить в два цвета. Это утверждение легко проверить. Если правильная раскраска
есть, то в силу симметрии можно считать, что первая вершина цикла покрашена в цвет 1, тогда
вторая вершина покрашена в цвет 2 и так далее, то есть каждая нечётная вершина будет покрашена
в цвет 1, а каждая чётная — в цвет 2. Тогда последняя вершина цикла будет покрашена в тот же
цвет, что и первая, что невозможно. \\ 

Докажем теперь, что если в графе нет циклов нечётной длины, то он двураскрашиваемый. Для
этого построим раскраску. Выберем в каждой компоненте связности по вершине c, которую назовём
центром, и покрасим её в цвет 2; все вершины на расстоянии (все расстояния и пути подразумеваются минимальными по количеству ребер) 1 от неё покрасим в цвет 1, все вершины
на расстоянии 2 — в цвет 2 и так далее: вершины на чётном расстоянии от центра покрасим в цвет
2, а на нечётном в цвет 1. \\

Предположим, что в результате этой процедуры получилась неправильная раскраска. Это означает, что у некоторого ребра $\{u, v\}$ концы были покрашены в один цвет, а это произошло, если расстояния от центра c некоторой компоненты до вершин $u$ и $v$ имеют одинаковую чётность. Заметим, что если расстояния от центра до $u$ и $v$ не равны, то путь до одной из вершин можно было сократить, проходя через другую вершину (так как расстояния отличаются как минимум на 2). Получаем, что расстояния от центра до $v$ и $u$ равны. \\

Но тогда путь от центра до $v$ + ребро $\{v, u\}$ + путь от $u$ до центра имеют нечетную длину (пути могут пересекаться, но простоту цикла в теореме ничего не сказано). Получили противоречие. \\ \\

\noindent \textbf{Булев куб двураскрашиваемый}

Будем называть четностью вершины $v = (x_1, \dots, x_n)$ число $parity(v) = x_1 + \dots + x_n
\ \text{mod}\ 2$. Тогда заметим, что если $v, u$ связаны ребром, то $parity(v) \neq parity(u)$. Значит если у нас существует цикл нечетной длины $k$
\[
v_1 \to v_2 \to \dots \to v_k \to v_1
\]
то, так как четность на каждом шаге меняется, получаем $parity(v_1) = parity(v_3) = \dots = parity(v_k)$, но соседние вершины не могут иметь одну четность. Получаем противоречие.