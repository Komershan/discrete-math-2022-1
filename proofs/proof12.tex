\subsection{Критерий Поста полноты системы булевых функций.}
\textbf{Критерий Поста.} $[F] = P_2 \Leftrightarrow F \nsubseteq L, F \nsubseteq T_0, F \nsubseteq T_1, F \nsubseteq S, F \nsubseteq M$\\

\noindent \textbf{Доказательство:}\\

$1. \Rightarrow$\\

От противного: пусть $F \subseteq C$, где $C$ это какой-то класс. Тогда $[F] \subseteq [C] = C$.\\

$2. \Leftarrow$\\

Из леммы о функции, не лежащей в классе, сохраняющем константу, заметим, что мы можем выразить из функций не из $T_0$ и не из $T_1$ либо константы, либо отрицание $\neg x$. Пусть мы смогли выразить отрицание. Тогда по лемме о несамодвойственной функции мы также можем выразить 0 и 1. Пусть мы не смогли выразить отрицание. Тогда мы точно смогли выразить 0 и 1, потому по лемме о немонотонной функции, используя 0, 1 и $x$, мы можем выразить $\neg x$.
По лемме о нелинейной функции, коньюнкция $x \wedge y \in [0, \neg x, f]$. Потому мы также можем выразить коньюнкцию, а из коньюнкции и отрицания можем выразить дизьюнкцию, потому мы получили полную систему связок.
