\documentclass{article}
\usepackage[left=2cm, right=2cm, top=2cm, bottom=2cm]{geometry}
\usepackage[russian]{babel}
\usepackage{amsmath}
\usepackage{amssymb}
\usepackage{amsthm}

\begin{document}
    \subsection{Ациклические орграфы, топологическая сортировка}
    \begin{itemize}
        \item
        Рассмотрим следующие утверждения:
        \begin{enumerate}
            \item Ориентированный граф $G$ ацикличен. 
            \item Все компонентны сильной связности $G$ состоят из одной вершины.
            \item Вершины $G$ можно пронумеровать числами от $1$ до $n$ таким образом, что если из вершины $v$ можно достичь вершину $u$, то номер, сопоставленный вершине $v$ будет меньше номера, сопоставленного вершине $u$. 
        \end{enumerate}

        \item 
        Докажем, что эти утверждения эквивалентны.
        \begin{itemize}
        \item Докажем, что из \textbf{1} следует \textbf{2}. Предположим, что существуют две вершины $v, u$, которые находятся в одной компоненте сильной связности. Значит есть путь $v \to u$ и $u \to v$, то есть есть существует цикл на вершинах $v$ и $u$ - противоречие. 

        \item Докажем, что из \textbf{2} следует \textbf{1}. Предположим, что в графе $G$ бы существовал цикл. Рассмотрим две вершины $v, u$ из этого цикла. Посколько из $v$ можно попасть в $u$, а из $u$ в $v$, то они должны лежать в одной компоненте связности, следовательно размер этой компоненты будет хотя бы 2 - противоречие. 

        \item Докажем, что из \textbf{3} следует \textbf{1}. Предположим, что в графе $G$ бы существовал цикл. Рассмотрим любое ребро из этого цикла, пусть оно имеет вид: $v \to u$. Тогда число, присвоенное вершине $v$ должно быть меньше числа, присвоенного вершине $u$. С другой стороны, существует путь из вершины $u$ в вершину $v$. Каждый раз, когда мы переходим к следующей вершине, номер, присвоенный ей, будет увеличиваться, поэтому когда мы придем из $u$ в $v$, номер вершины $v$ должен оказаться больше номера вершины $u$ - противоречие. 

        \item Докажем, что из \textbf{1} следует \textbf{3}. Воспользуемся мат. индукцией по количеству вершин в графе. \\
        \textbf{База}: $n = 1$ - очевидно. \\
        \textbf{Переход}: Пусть верно для $n = (k - 1)$, докажем для $n = k$. Найдем в нашем графе вершину, в которую не входит ни одно ребро. Такая всегда найдется, ведь иначе в нашем графе есть цикл. \\
        Покажем это: будем идти по обратным ребрам от произвольной вершины v. По предположению не существует вершины, в которую не входят ребра, поэтому мы можем идти по обратным ребрам бесконечно долго - неизбежно получится цикл. \\
        Итак, удалим вершину, в которую не входит ни одно ребро, присвоим ей номер 1. Останется часть графа на $(k - 1)$ вершине, которую можно занумеровать числами от $1$ до $(k - 1)$. Прибавим к каждому номеру единицу. Получим граф, в котором каждой вершине сопоставлено число от $1$ до $k$, причем если вершина $v$ достижима из вершины $u$, то номер, сопоставленной вершине $v$ будет меньше номера, сопоставленного вершине $u$.
        
        \end{itemize}
    \end{itemize}
\end{document}
