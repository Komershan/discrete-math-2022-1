\subsection{Доказательство эквивалентности трех определений фундированных множеств}

\textbf{Следующие три определения фундированных множеств эквивалентны}:

\hspace{0.5cm}\parbox{17cm} {
    1. каждое непустое подмножество имеет минимальный элемент

    2. любая убывающая цепь конечна

    3. Для порядка $P$ спраедлив принцип индукции: если для утверждения $A(p)$, зависящего от элемента порядка, для любого $p$ верно утверждение «если $A(q)$ верно при всех $q < p$, то и $A(p)$ верно». Тогда утверждение $A(p)$ верно при любом $p \in P$. 
}

\textbf{Доказательство:}

\hspace{0.5cm}\parbox{17cm} {
    Докажем, что из  1 следует 2 и из 2 следует 1.

    $1 \Rightarrow 2$. От противного: пусть существует бесконечная убывающая цепь. Тогда она не имеет минимальный элемент, т.к если бы имела, то цепь не была бы бесконечной.

    $2 \Rightarrow 1$. Пусть какое-то непустое подмножество не имеет минимального элемента. Попробуем построить в нем бесконечную убывающую цепь. Возьмем какой-нибудь элемент $x_0$, он не минимальный, т.к существует $x_1 < x_0$, для $x_1$ же в свою очередь существует $x_2 < x_1$ и так далее. Получаем бесконечную убывающую цепь.

    Таким образом, $1 \Leftrightarrow 2$. Докажем, что из 1 следует 3 и из 3 следует 1.

    $1 \Rightarrow 3$. Рассмотрим множество всех $x$ таких, что $A(x)$ ложно. Из 1 следует, что в данном множестве есть минимальный элемент $m$. Если это не минимум множества в целом. то для всех $y < m$ из множества, $A(y) = 1$, потому $A(m) = 1$ -- противоречие. К тому же $m$ не может быть минимальным элементом в множестве в целом, т.к тогда для него не выполняется принцип индукции, а мы предположили что он выполняется. Потому множество всех $x$ таких, что $A(x) = 0$ пусто.

    $3 \Rightarrow 1$. Выделим из $P$ непустое подмножество $X$, в котором нет минимального элемента. Рассмотрим следующее индуктивное утверждение $A(p): p \notin X$.

    Индуктивное утверждение выполняется: если $q < p$ и $A(q) = 1$, то $A(p) = 1$ (иначе $p$ -- минимальный элемент в $X$). Потому $\forall p, p \notin X \Rightarrow X = \emptyset$.
}
