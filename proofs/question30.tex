\subsection{Теорема Холла.}

\textbf{Теорема Холла.} Если для каждого множества $X$ вершин двудольного графа $G = (L, R, E)$ множество соседей $G(X) \subseteq R$ содержит не меньше вершин, чем $X$, то в графе $G$ есть паросочетания размера $|L|$ \\

\noindent \textbf{Доказательство:}

Полная индукция по количеству элементов в левой доле $L$.

\textit{База индукции.} Если в $L$ всего одна вершина $x$, то у неё есть хотя бы один сосед y в правой доле $R$ по условию теоремы. Получаем паросочетание с ребром $\{x, y\}$.

\textit{Шаг индукции.} Предположим, что утверждение теоремы выполняется для всех двудольных графов, в которых левая доля содержит меньше n вершин. Рассмотрим граф $G = (L, R, E)$, для которого выполняются условия теоремы и в $L$ ровно $n$ вершин. Разберём два случая. \\

\textbf{Первый случай}: в левой доле есть такое множество $X$, для которого $|X| = |G(X)|$. Выделим из графа два подграфа. Первый, $G_1$, имеет доли $X$, $G(X)$ и все рёбра между этими вершинами. Второй, $G_2$, имеет доли $L \setminus X$, $R \setminus G(X)$ и все рёбра между этими вершинами. Для обоих графов выполняются условия теоремы Холла. Для $G_1$ это очевидно по построению. Докажем выполнение условий теоремы для графа $G_2$ от противного. Пусть для подмножества $Z \subseteq L \setminus X$ соседей в $R \setminus G(X)$ меньше, чем вершин в $Z$. Тогда в графе $G$ соседей у множества $X \cup Z$ меньше $|Z \cup X|$ (ведь множества $X$ и $Z$ не пересекаются, а соседей у $X$ ровно $|X|$). \\ 

Итак, для $G_1$, $G_2$ выполняются условия теоремы, а количество вершин в них меньше n. Поэтому по предположению индукции в каждом из них есть паросочетание размера левой доли. Объединяя эти два паросочетания, получаем искомое паросочетание в $G$ размера $|L|.$ \\

\textbf{Второй случай}: для каждого $X \subseteq L$ выполняется неравенство $|X| < |G(X)|$.

Выберем вершину $a \in L$ и её соседа $b \in R$ (в этом случае соседей у каждой вершины больше одного, нас устроит любой).

Если в графе $G' = ((L \setminus {a}), (R \setminus {b}, E'))$, полученном из $G$ выбрасыванием вершин $a$, $b$ и инцидентных им рёбер, есть паросочетание $P$ размера $n - 1$, то в графе $G$ есть паросочетание размера n: к рёбрам из $P$ добавим ребро $\{a, b\}$.

Если такого паросочетания нет, условие теоремы Холла для $G'$ нарушается в силу индуктивного предположения. Какое-то «особое» множество $X \subseteq L \setminus \{a\}$ имеет мало соседей в $(R \setminus \{b\}: |X| > |G'(X)|$. Но в графе $G$ у множества $X$ есть разве что ещё один сосед b. Поэтому для этого множества выполняется равенство $|X| = |G(X)|$. Это первый случай, который уже рассмотрен выше.