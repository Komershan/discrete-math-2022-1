\subsection{Теорема о классах эквивалентности для отношения эквивалентности.}

\textbf{Теорема.} Если $R$ -- отношение эквивалентности на $A$, то существует разбиение множества $A$ отношением эквивалентности $R$ на непересекающиеся классы $(\forall i \neq j \colon A_i \cap A_j = \varnothing,\; \cup_{i \in I} A_i = A)$ такое, что $\forall x, y \in A_i \colon xRy$ и $\forall x \in A_i, y \in A_j, i \neq j: \neg xRy$. (то есть если два элемента принадлежат одному классу эквивалентности, они находятся в отношении $R$ и наоборот).

\textbf{Доказательство:}

$a \in A$

$[a] = \{b \in A | aRb\}$ -- класс эквивалентности для $a$.

1) $a \in [a]$, т.к $aRa$.

2) Пусть $\neg (aRb) \Rightarrow [a] \cap [b] = \varnothing$.

Действительно, если $x \in [a] \cap [b]$, то $aRx$ и $bRx$ $\Rightarrow$ $aRx$ и $xRb$ 
$\Rightarrow$ $aRb$ -- противоречие.

3) Пусть $aRb$. Тогда $[a] = [b]$.

Действительно, пусть $x \in [b]$, то есть $bRx$.

Тогда $aRb$ и $bRx$ $\Rightarrow$ $aRx$ $\Rightarrow$ $x \in [a]$.

Значит $[b] \subseteq [a]$

Аналогично можно заключить, что $[a] \subseteq [b]$. Из этого следует, что классы совпадают.

4) $A$ есть объединение набора непересекающихся множеств (классов эквивалентности). Классы внутри пересекаться не могут, т.к если пересекаются, то по транзитивности это один и тот же класс.

5) Пусть $x, y \in [a]$. Тогда $aRx$ и $aRy$ $\Rightarrow$ $xRa$ и $aRy$ $\Rightarrow xRy$.

6) Если $x, y \in$ разным классам, то $\neg(xRy)$.

От противного: пусть $x \in [a], y \in [b], [a] \neq [b], xRy$.

Тогда $aRx$, $xRy$, $yRb$ $\Rightarrow$ $aRb$, то есть $[a] = [b]$.
