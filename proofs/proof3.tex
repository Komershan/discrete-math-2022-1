\subsection{Бинарные отношения, теорема об ассоциативности композиции отношений. Функции. Критерий существования функции, обратной к данной. Композиция биекций является биекцией.}

\begin{itemize}
	\item \textbf{Ассоциативность композиции бинарных отношений}
	
	Пусть $R \subseteq A \times B$, $S \subseteq B \times C$, $T \subseteq C \times D$. Тогда $T \circ (S \circ R) = (T \circ S) \circ R$. Иначе говоря, композиция отношений обладает свойством ассоциативности.

	\textbf{Доказательство:}
	
	$a \in A, d \in D$
	
	$(a, d) \in T \circ (S \circ R) \Leftrightarrow \exists z \in C: a(S \circ R)z$ и $zTd$ $\Leftrightarrow$ $\exists y \in B, z \in C: aRy$, $ySz$ и $zTd$.
	
	Правая часть расписывается аналогично:
	
	$(a, d) \in (T \circ S) \circ R \Leftrightarrow \exists y \in B: aRy$ и $y(T \circ S)d \Leftrightarrow \exists z \in C, y \in B: aRy$, $ySz$ и $zTd$.
	
	\item \textbf{Критерий существования функции, обратной к данной}
	
	\textbf{Теорема.} У функции $f : A \to B$ существует обратная $\Leftrightarrow$ $f$ - биекция.
	
	\textbf{Доказательство.} 
	
	1) $\Leftarrow$ Знаем, что $f$ - биекция, то есть $\forall y\ \exists! x : f(x) = y$. Зададим $g : B \to A$ - всюду определенная функция $g(y) = x$. Проверяем: $f \circ g (y) = f (g(y)) = f(x) = y = id_B$ и $g \circ f(x) = g(f(x)) = g(y) = x = id_A$ \\
	
	2) $\Rightarrow$ Знаем, что существует $g : B \to A$, что $g \circ f = id_A, f \circ g = id_B$. Докажем, что $f$ - биекция. Возьмем $x_1, x_2 \in A$, пусть $f(x_1) = f(x_2)$, тогда $g(f(x_1))  = g(f(x_2)) \Rightarrow x_1 = x_2$ по $id_A$, то есть $f$ - инъекция. \\
	
	Пусть $x = g(y)$, тогда $f(x) = f(g(y)) = y$, значит $f$ - сюръекция. А значит и биекция, доказано.
	
	\item \textbf{Композиция биекций является биекцией}
	
	Пусть заданы $f : A \to B$ и $g : B \to C$ - биекции, докажем, что $g \circ f$ - биекция.
	
	1) $g \circ f$ - инъекция, рассмотрим $a_1, a_2 \in A, a_1 \neq a_2$, так как $f$ - инъекция $f(a_1) \neq f(a_2)$, так как $g$ - инъекция, то $g(f(a_1)) \neq g(f(a_2))$, а значит $g \circ f$ - инъекция.
	
	2) $g \circ f$ - сюръекция, возьмем $c \in C$, тогда так как $g$ - сюръекция $\exists b \in B : g(b) = c$, так как $f$ - сюръекция, то $\exists a \in A : f(a) = b$. По определению композиции, $(a, c)$ в композиции, если $\exists b : afb, bgc$, что мы и сделали.
	
	$g \circ f$ сюръекция + инъекция, значит биекция. 
\end{itemize}
