\subsection{Связь строгих и нестрогих частичных порядков. Изоморфизм порядков. Примеры. Доказательство попарной неизоморфности порядков $\mathbb{Z}, \mathbb{Q}, [0, 1]$ и $(0, 1)$. Доказательство неизоморфности покоординатных порядков на $[0, 1]^2$ и $\{(x, y) \in \mathbb{R}^2 | |x| + |y| \le 1\}$.}

\textbf{Связь строгого и нестрогого частичных порядков:} Из отношения не строгого порядка на $P$ можно получить отношение строгого порядка на $P$ и наоборот следующим образом

\begin{center}
    $a \le b \Leftrightarrow a < b$ или $a = b$

    $a < b \Leftrightarrow a \le b$ и $a \neq b$
\end{center}

\textbf{Доказательство:}

1. Докажем, что отношение $(a < b$ или $a = b)$ обладает свойствами отношения нестрогого порядка:

\hspace{0.5cm}\parbox{15cm}{
    $\textbullet$ Рефлексивность следует из того, что если $a = b$, то объекты находятся в отношении. 

    $\textbullet$ Антисимметричность следует из того, что строгие порядки не обладают свойством симметричности (для них одновременно не выполняется то, что $a < b$ и $b < a$).

    $\textbullet$ Для транзитивности рассмотрим три случая. Если $a = b$ и $b = c$, то $a = c$. Если $a = b$ и $b < c$ или $a < b$ и $b = c$, то $a < c$. Иначе, $a < b < c \Rightarrow a < c$. Во всех трех случаях транзитивность выполняется.
}

2. Докажем, что отношение $(a \le b$ и $a \neq b)$ обладает свойствами отношения строгого порядка:

\hspace{0.5cm}\parbox{15cm}{
    $\textbullet$ Антирефлексивность следует из того, что в отношении рассматриваются $a \neq b$. Т.к $a = a$, при любых $a$ этот элемент не будет находиться в отношении с самим собой.

    $\textbullet$ Пусть $a \le b, b \le c$, $a \neq b, b \neq c$, докажем что $a \le c$ и $a \neq c$. Первое следует из транзитивности нестрогого частичного порядка $\le$. Если же $a = c$, то $a \le b$ и $b \le c = a$ $\Rightarrow a \le b$ и $b \le a \Rightarrow a = b$. Противоречие.
}

Порядки $\le_P$ и $\le_Q$ изоморфны, если существует биекция $\phi: P \rightarrow Q$ такая, что $x \le_P y \Rightarrow \phi(x) \le_Q \phi(y)$.

Обозначается как $P \cong Q$

Пример: порядки $((0, 1), <) и ((-\infty, -1), <)$ изоморфны, т.к есть биекция $\phi(x) = -\frac{1}{x}$.

\textbf{Когда порядки могут быть не изоморфны?} Например, когда:

\hspace{0.5cm}\parbox{15cm} {
    1. Один порядок имеет максимальный/минимальный элементы, а второй не имеет. Например, поэтому порядки $[0, 1]$ и $(0, 1)$ не изоморфны: первый порядок имеет очевидные минимальный и максимальный элементы, а второй таковых не имеет.

    2. Порядки в целом имеют разное количество максимальных/минимальных/наименьших/наибольших элементов. Например, покоординатные порядки на $[0, 1]^2$ и $\{(x, y) \in \mathbb{R}^2 | |x| + |y| \le 1\}$ неизоморфны: в порядке $[0, 1]^2$ всего лишь один минимальный элемент (0, 0), а наименьших элементов нет: с другой стороны, в $\{(x, y) \in \mathbb{R}^2 | |x| + |y| \le 1\}$ минимальных элементов нет, а наименьших бесконечно много. Потому данные порядки не изоморфны.

    3. Изоморфизм не порождает изоморфизма отрезков. В частности, отрезок из конечного числа элементов должен соответствовать отрезку из конечного числа элементов. Потому например $\mathbb{Z}$ не изоморфен $\mathbb{Q}$: в первом случае все отрезки конечны, во втором случае все отрезки бесконечны.

    4. Ну и не будем забывать что мы строим биекцию. В частности, потому множество $\mathbb{Q}$ не изоморфно интервалу $(0, 1)$: если бы мы могли её построить, то существовала бы биекция из $\mathbb{N}$ в $(0, 1)$, что значит что интервал был бы счетным множеством. Потому такой биекции нет. Аналогично, потому не изоморфны остальные пары порядков.

 }
