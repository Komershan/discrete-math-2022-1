\subsection{Теорема Рамсея. Верхняя оценка чисел Рамсея.}

Кликой называется множество вершин графа, каждая пара которых соединена ребром. \\

\textbf{Теорема Рамсея.} Для любых $k, n$ найдётся такое число $N_0$, что в любом графе на $N \geq N_0$ вершинах есть или клика размера k, или независимое множество размера n. \\

Ясно, что если утверждение теоремы справедливо для графов на N вершинах, то оно справедливо и для графов с $N' > N$ вершинами. Обозначим через $R(k,n)$ число Рамсея — минимальное количество вершин, для которого справедлива теорема. \\

\noindent \textbf{Доказательство:}

Будем доказывать индукцией по s, что для любой пары чисел k, n такой, что k + n = s справедливо утверждение теоремы.

\textbf{База индукции} $s = 2$ очевидна: $2 = 1 + 1$ — это единственный способ разложить число 2 в сумму целых положительных слагаемых, а одна вершина является одновременно и кликой, и независимым множеством.

\textbf{Шаг индукции.} Предположим, что утверждение выполнено для всех пар $(k, n)$ таких, что $k + n = s$.

Докажем его для пары $(k, n)$ такой, что $k + n = s + 1$. По индуктивному предположению утверждение теоремы выполнено для пар $(k - 1, n)$ и $(k, n - 1)$.

Рассмотрим граф на $N_0 = R(k - 1, n) + R(k, n - 1)$ вершине и возьмём какую-то вершину $v$ этого графа.

Вершин в графе за исключением вершины $v$ ровно $N_0 - 1$ штук. Среди них $x$ соседей и $y$ несоседей.

Докажем, что выполняется хотя бы одно из неравенств

$$x \geq R(k - 1, n)$$
$$y \geq R(k, n - 1)$$

В противном случае выполняются два неравенства

$$x < R(k - 1, n)$$
$$y < R(k, n - 1)$$

из которых следует $x+y \leq R(k - 1, n) - 1+R(k, n - 1) - 1 = R(k - 1, n)+R(k, n - 1) - 2$.

Получаем противоречие

$$N_0 - 1 = x + y \leq R(k - 1, n) - 1 + R(k, n - 1) - 1 = N_0 - 2$$

Поэтому у вершины $v$ есть $R(k - 1, n)$ соседей или есть $R(k, n - 1)$ несоседей. \\

Оба случая рассматриваются аналогично.

\textbf{Первый случай.} В индуцированном соседями вершины $v$ подграфе по предположению индукции найдётся клика размера $k - 1$ или независимое множество размера n. В первом варианте добавление вершины $v$ даёт клику в исходном графе размера $k$, во втором варианте в исходном графе есть независимое множество размера $n$.

\textbf{Второй случай.} В индуцированном несоседями вершины $v$ подграфе по предположению индукции найдётся клика размера k или независимое множество размера $n - 1$. В первом варианте в исходной графе есть клика размера k, а во втором добавление вершины $v$ даёт независимое множество размера n в исходном графе.

Итак, мы доказали утверждение теоремы и для произвольной пары $(k, n)$, для которой $k + n = s + 1$. Индуктивный переход доказан, и теорема следует из принципа математической индукции. \\

\textbf{Оценка сверху.} Докажем по индукции $R(k, n) \leq C^{k - 1}_{n + k - 2} = C^{n - 1}_{n + k - 2}$. Будем опять делать индукцию по $s = n + k$. База очевидна. \\

Переход, имеем: $R(k - 1, n) \leq C^{k - 2}_{n + k - 3},\ R(k, n - 1) \leq C^{k - 1}_{n + k - 3}$, тогда пользуясь $C_{n}^{k} = C_{n - 1}^{k - 1} + C_{n - 1}^{k}$ получаем, что $R(k, n) \leq C^{k - 2}_{n + k - 3} + 
C^{k - 1}_{n + k - 3} = C^{k - 1}_{n + k - 3}$