\subsection{Предполные классы булевых функций. Описание предполных классов булевых функций}

В $P_2$ есть ровно пять предполных классов: $T_0, T_1, L, S, M$

Доказательство:

Сперва докажем, что эти пять классов являются различными. Для этого, например, можно написать таблицу в которой на пересечении строки и столбца будет выписана функция, принадлежащая классу соответствующему строке и не принадлежащая классу соответствующему столбцу.

\begin{tabular}{ |c|c|c|c|c|c| } 
	\hline
	$\#$ & $T_0$ & $T_1$ & $M$ & $S$ & $L$ \\ 
	\hline
	$T_0$ & $\#$ & $0$ & $x \oplus y$ & $0$ & $x \wedge y$\\ 
	\hline
	$T_1$ & $1$ & $\#$ & $x \oplus y \oplus 1$ & $1$ & $x \wedge y$\\ 
	\hline
	$M$ & $1$ & $0$ & $\#$ & $1$ & $x \wedge y$\\ 
	\hline
	$S$ & $\lnot x$ & $\lnot x$ & $\lnot x$ & $\#$ & $MAJ(x, y, z)$\\ 
	\hline
	$L$ & $1$ & $0$ & $x \oplus y$ & $1$ & $\#$\\ 
	\hline
\end{tabular}

Теперь докажем, что эти классы являются предполными: допустим мы взяли клас $T_0$ и добавили в него какую-то функцию не из $T_0$. Из таблицы выписанной выше видно, что в $T_0$ для оставшихся четырех классов найдется функция не входящая в эти классы, а мы добавили функцию не из $T_0$. Получается, по критерию Поста, мы получили полную систему. А значит $T_0$ - предполный класс. Аналогично это можно доказать и для $T_1, M, S, L$.

Теперь докажем, что никакой другой класс не является предполным. Пусть существует еще какой-то предполный класс $F$. $F$ должен полностью содержатся в одном из классов $T_0, T_1, M, S, L$, иначе, по критерию Поста, $F$ - полная система. Пускай $F$ содержится в $T_0$(для других классов все аналогично), но при этом $F \neq T_0$, ведь мы предположили, что это какой-то другой класс. Но тогда найдется функций $g$ такая, что $g \in T_0 \setminus F$. Значит $\{g\} \cup F \subseteq T_0$ но из этого следует, что $[\{g\} \cup F] \neq P_2$. Значит $F$ - не предполный класс. 
