\subsection{Несчетность множества бесконечных последовательностей из 0 и 1. Сравнение мощностей, теорема Кантора о сравнении мощности множества и множества всех его подмножеств.}
\textbf{Несчетность множества бесконечных последовательностей из 0 и 1.}

\noindent \textbf{Доказательство от противного:}\\

Пусть $\mathbb{B}^{\infty}$ - счетно. Тогда можно каждому натуральному поставить во взаимно-однозначное соответствие
бесконечную последовательность из 0 и 1. Тогда выпишем в столбик все натуральные числа, а в строчку к ним припишем соответствующие
им последовательности. Тогда у нас не будет последовательностей, которых нет в этой таблице. Воспользуемся диагональным
методом Кантора.

\begin{equation*}
    \begin{array}{c|cccccc}
        1 & \colorbox{lightgray}{$a_{11}$} & a_{12} & a_{13} & a_{14} & ...&\dots\\
        2 & a_{21} & \colorbox{lightgray}{$a_{22}$} & a_{23} & a_{24} & ...&\dots\\
        3 & a_{31} & a_{32} & \colorbox{lightgray}{$a_{33}$} & a_{34} & ...&\dots\\
        4 & a_{41} & a_{42} & a_{43} & \colorbox{lightgray}{$a_{44}$} & ...&\dots\\
        \vdots & \dots & \dots & \dots & \dots & \ddots&\dots\\
        n & \dots & \dots & \dots & \dots & \dots & \colorbox{lightgray}{$a_{nn}$}\\
        \vdots & \dots & \dots & \dots & \dots & \dots & \dots\\
    \end{array}
\end{equation*}

Теперь возьмем все элементы с диагонали и инвертируем их, то есть возьмем обратные к ним, заметим теперь, что
мы из каждой последовательности взяли по элементу, и изменили его на обратный, то есть полученная последовательность
не равна никакой из таблицы, то есть этой последовательности нет в таблице. Противоречие.\\

\textbf{Сравнение мощностей, теорема Кантора о сравнении мощности множества и множества всех его подмножеств.}\\

Пусть $X$ - множество. Тогда $|X| < |2^X|$.\\

\noindent \textbf{Доказательство:}\\

1) $|X| \leq |2^X|$, так как существует инъекция $f : X \to 2^X$, $f(x) = \{x\} \in 2^X$.\\

2) $X \not \sim 2^X$

Докажем от обратного, пусть существует биекция $f : X \to 2^X$. Пусть $Y = \{x \in X \mid x \notin f(x)\}$. Очевидно, что
$Y \subseteq X \Rightarrow Y \in 2^X$. Значит $\exists x \in X : f(x) = Y$.

\begin{enumerate}
    \item $x \in Y \Rightarrow x \notin f(x) = Y$ - противоречие
    \item $x \notin Y \Rightarrow x \in f(x) = Y$ - противоречие
\end{enumerate}

Во все случаях получили противоречие, значит такой биекции нет. Чтд