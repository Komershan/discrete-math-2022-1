% !TEX options=--shell-escape
\documentclass{article}
\usepackage{tikz,amsthm,amsmath,cancel,pgfplots,animate,multirow,unicode-math,adjustbox,booktabs,array,pst-solides3d,pst-all,pst-3dplot,color,colortbl,sets,mathtools}
\usepackage[margin=96pt]{geometry}

\usetikzlibrary{automata,positioning,calc,decorations.pathmorphing,patterns,external}
\tikzexternalize[prefix=external/]

\newtheorem{problem}{Задача}
\newtheorem{lemma}{Лемма}
\newtheorem{theorem}{Теорема}
\newtheorem{definition}{Определение}
\newtheorem{example}{Пример}
\renewcommand*{\proofname}{Доказательство}
\newcommand{\const}{\mathrm{const}}
\DeclareMathOperator{\ord}{ord}
\DeclareMathOperator{\maj}{MAJ}
\newcommand{\abs}[1]{\left\lvert#1\right\rvert}
\newcommand{\floor}[1]{\left\lfloor#1\right\rfloor}
\newcommand{\ceil}[1]{\left\lceil#1\right\rceil}
\newcommand{\fr}[1]{\left\{#1\right\}}
\newcommand{\N}{\mathbb{N}}
\newcommand{\Z}{\mathbb{Z}}
\newcommand{\Q}{\mathbb{Q}}
\newcommand{\R}{\mathbb{R}}
\newcommand{\C}{\mathbb{C}}
\newcommand{\continuum}{\mathfrak{c}}
\DeclarePairedDelimiterX\set[1]\lbrace\rbrace{\def\given{\;\delimsize\vert\;}#1}

\newcolumntype{o}{>{\columncolor{orange}}c}

\makeatletter
\newenvironment{sqcases}{%
	\matrix@check\sqcases\env@sqcases
}{%
	\endarray\right.%
}
\def\env@sqcases{%
	\let\@ifnextchar\new@ifnextchar
	\left\lbrack
	\def\arraystretch{1.2}%
	\array{@{}l@{\quad}l@{}}%
}
\makeatother

\setlength{\parindent}{0pt}
\setlength{\parskip}{5pt}
\setmainfont{CMU Serif}
\widowpenalties 1 10000
\raggedbottom

\date{12 ноября, 2022}
\title{Дискретная математика \\ \Large Домашнее задание 9}
\author{Иван Мачуговский}

\AtBeginDocument{
	\renewcommand{\setminus}{\mathbin{\backslash}}
}

\begin{document}
	\maketitle

	\begin{section}{Задача 1}
		Если некоторая цепь и антицепь пересекаются по хотя бы двум элементам, то эти два элемента обязаны одновременно быть сравнимыми и несравнимыми, что невозможно. Следовательно, антицепь может включать в себя не более $n - k + 1$ элементов: не более одного из максимальной цепи и не более $n - k$ из оставшихся $n - k$ элементов. Эта граница строго достигается, например, на частично упорядоченном множестве $(\{1, 2, \dots, n\}, \preceq)$, где $a \preceq b$, если $a \le b \le k$.
	\end{section}

	\begin{section}{Задача 2}
		Введем порядок на прямых: пусть $f \preceq g$, если график функции $g$, ограниченный на верхнюю полуплоскость (включая $y = 0$), лежит нестрого справа от графика функции $f$, ограниченной на то же множество. Это эквивалентно тому, что $f$ и $g$ пересекаются в нижней полуплоскости (включая $y = 0$), и угловой коэффициент $g$ меньше, чем у $f$.

		Если существует антицепь мощности $5$, то из этих пяти прямых любые две несравнимы, то есть при $y \ge 0$ ни у одной из них график не лежит нестрого правее другой, а значит, любые две из этих прямых пересекаются в верхней полуплоскости, что и требовалось.

		Если же все антицепи имеют мощности $\le 4$, то по теореме Дилуорса найдется разбиение множества прямых (коих $16$) на не более чем $4$ непересекающихся цепи. Следовательно, в силу принципа Дирихле хотя бы одна из этих цепей будет иметь мощность $\ge 4$, то есть найдутся $4$ прямые, любые две из которых не пересекаются в верхней полуплоскости, а значит, пересекаются в нижней. Что несколько подозрительно, потому что в условии оценка хуже, но как есть.
	\end{section}

	\begin{section}{Задача 3}
		а) Нет: пусть $A$ состоит их двух копий $\N$, где сравнения внутри копий наследуются из $\N$, а между копиями не определены. Формально,

		\begin{equation*}
			A = (\set{n, n' \mid n \in \N}, \preceq); \ n \le m \iff n \preceq m \iff n' \preceq m'; \ \neg (n \preceq m'), \neg (n' \preceq m).
		\end{equation*}

		Легко видеть, что это множество образует частичный порядок. Оно также фундированно, поскольку любое $B \subseteq A, B \ne \emptyset$ пересекается либо с $\N$, либо с $\N'$ (либо с обоими); наименьший элемент в $B \cap \N$ или $B \cap \N'$ соответственно будет минимальным в $B$.

		Тогда определим $f: A \to A$ так, что $f(n) = n', f(n') = n$. Если $x \prec y$, то либо

		\begin{equation*}
			x = n, y = m, n < m \implies f(x) = n' \prec m' = f(y),
		\end{equation*}

		либо

		\begin{equation*}
			x = n', y = m', n < m \implies f(x) = n \prec m = f(y),
		\end{equation*}

		поэтому $f$ возрастает. Но, например, $f(1) = 1' \succeq 1$ не выполняется.

		б) Да, верно. Докажем это от противного.

		То, что элементы $x$ и $f(x)$ сравнимы и при этом $f(x) \not\succeq x$, эквивалентно $f(x) < x$. Тогда пусть

		\begin{equation*}
			B = \set{x \in A \mid f(x) < x}
		\end{equation*}

		и при этом $B \ne \emptyset$. $A$ фундированно, поэтому $B$ содержит минимальный элемент $x'$.

		Тогда $f(x') < x'$. Значит, в силу возрастания $f$ имеем $f(f(x')) < f(x')$, но тогда $f(x') \in B$. С другой стороны, $f(x') < x$, значит, $x$ не могло быть минимальным элементом $B$. Противоречие.
	\end{section}

	\begin{section}{Задача 4}
		Сперва посчитаем количество цепей из $6$ элементов, не включающих $0$. Это будут произвольные шестиэлементные подмножества $\{1, 2, \dots, 9\}$, поскольку подмножество цепи -- цепь. Таких подмножеств $\binom{9}{6}$.

		Сколько существует цепей, включающих $0$? По условию все элементы от $1$ до $9$ сравнимы, поэтому те $4$ пары несравнимых элементов, описанные в условии, обязаны иметь вид $(0, n)$, то есть $0$ несравним с ровно $4$ элементами от $1$ до $9$, а значит, сравним ровно с $5$ элементами от $1$ до $9$. Следовательно, ровно и только с этими пятью элементами он будет образовывать $6$-элементную цепь.

		Итого ответ

		\begin{equation*}
			\binom{9}{6} + 1 = 85.
		\end{equation*}
	\end{section}

	\begin{section}{Задача 5}
		Введем на отрезках частичный порядок: $[a, b] < [c, d]$, если $b < c$. Легко проверить, что это действительно порядок.

		Тогда цепь -- это такое множество отрезков, что любые два из них не пересекаются. Антицепь -- это множество отрезков, из которых любые два пересекаются; то, что это эквивалентно тому, что \textit{все} отрезки пересекаются -- классическая задача, для доказательства которой проверяющему предлагается избавиться от отрезков, целиком содержащих другие отрезки, и внимательно посмотреть на получившуюся картинку.

		Следовательно, в каждой антицепи можно выбрать точку, содердащуюся в каждом из отрезке, и наоборот, множество отрезков, содердащих произвольную фиксированную точку, образует антицепь. Поэтому каждому подмножеству точек на прямой, удовлетворяющему условию, соответствует некоторое разбиение множества данных отрезков на антицепи (возможно, пересекающиеся, но от пересечений легко избавиться, вычитая из каждой антицепи объединение всех предыдущих).

		По теореме, дуальной теореме Дилуорса (судя по всему, теореме Мирского), длина максимальной цепи (то есть $k$) равна минимальному числу антицепей, на которое можно разбить множество (то есть $n$), что и требовалось.
	\end{section}

	\begin{section}{Задача 6}
		Для фиксированной перестановки $\sigma$ введем порядок на ее индексах $(\{1, \dots, n\}, \preceq)$ так, что $i \preceq j$, если $(i, \sigma(i)) \le (j, \sigma(j))$ поэлементно.

		Хорошая перестановка -- это такая перестановка, у которой в соответствующем ей порядке нельзя выбрать антицепь из $11$ элементов. Следовательно, у хорошей перестановки мощность максимальной антицепи $\le 10$, поэтому по теореме Дилуорса ее можно разбить на $\le 10$ непересекающихся цепей.

		Произвольно пронумеруем эти $\le 10$ цепей натуральными числами от $1$ до $10$. Если цепей меньше, чем $10$, то какие-то из чисел останутся неиспользованными.

		Для каждого индекса $i$ выпишем номер цепи, в которой он лежит. Получится однозначное соответствие \textit{индексов} и цепей -- строка из $n$ чисел от $1$ до $10$.

		Аналогично, для каждого элемента $\sigma_i$ выпишем номер цепи, в которой он лежит. Получится однозначное соответствие \textit{значений} и цепей -- строка из $n$ чисел от $1$ до $10$.

		Итого для $\sigma$ выписано $2n$ чисел от $1$ до $10$. По такой строке $\sigma$ восстанавливается однозначно, ведь элементы цепи, если их выписать по возрастанию индексов, сами возрастают: для каждой цепи мы знаем принадлежащие ей индексы и значения; отсортируем и то, и другое, и сопоставим $k$-му индексу $k$-е значение. Естественно, некоторые строки будут невалидны, а именно те, в которых в какой-то цепи число индексов не равно числу значений.

		Таким образом, описанное преобразование перестановки в строку является инъекцией. Строк всего $10^{2n}$, поэтому перестановок не более $10^{2n}$, что и требовалось.
	\end{section}
\end{document}
