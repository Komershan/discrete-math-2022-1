\subsection{Домашнее задание 11}
\begin{center}
\textbf{Задача 1}
\end{center}
Рассмотрим последнюю цифру у чисел последовательности:

$\displaystyle  \begin{array}{{>{\displaystyle}l}}
a_{0}\rightarrow 5\\
a_{1}\rightarrow 8\\
a_{2}\rightarrow 7\\
a_{3}\rightarrow 2\\
a_{4}\rightarrow 7\\
...
\end{array}$

Последняя цифра не зависит от других, поэтому будет цикл. $\displaystyle a_{2015}$ имеет нечётный номер, поэтому последней цифрой будет $\displaystyle 7$.

\begin{center}
\textbf{Задача 2}
\end{center}
Назовём вершины с нечётной степенью $\displaystyle a,b,c,d$. Добавим в граф рёбра $\displaystyle ( a,b)$ и $\displaystyle ( c,d)$. Теперь в графе обязательно есть Эйлеров цикл по всем рёбрам. Выкинем из этого цикла рёбра, которые мы добавили. Получилось 2 непересекающихся пути, проходящие по всем рёбрам начального графа. Также пути начинаются и заканчиваются в вершинах с нечётными степенями, так как удалённые рёбра соединяли их.\qed 

\begin{center}
\textbf{Задача 3}
\end{center}
Докажем по индукции:

База: $\displaystyle n=1$ - всего одна вершина, поэтому Гамильтонов путь есть

Переход $\displaystyle n\rightarrow n+1$: пусть для $\displaystyle n$ есть какой-то путь $\displaystyle [ a,\ b,\ c,\ ...]$. Назовём $\displaystyle x_{0}$ вершину в кубе размерности $\displaystyle n+1$, у которой первые $\displaystyle n$ координат равны массиву $\displaystyle x$, а последняя - ноль. Аналогично, в $\displaystyle x_{1}$ последняя координата - 1.Тогда новый путь можно построить так: $\displaystyle [ a_{0} ,\ a_{1} ,\ b_{1} ,\ b_{0} ,\ c_{0} ,\ c_{1} ,...]$. Так мы посетим все вершины ровно по одному разу. Такой путь существует, так как раз есть путь $\displaystyle [ a,\ b,\ c...]$, то есть пути и $\displaystyle [ a_{0} ,b_{0} ,c_{0} ,...]$ и аналогичный с $\displaystyle [ a_{1} ,...]$.

\begin{center}
\textbf{Задача 4}
\end{center}
Сделаем алгоритм, который ищет вершину, расстояние из которой до всех остальных не больше $\displaystyle 2$. Будем поддерживать множество $\displaystyle A$ - вершин на расстоянии $\displaystyle 1$ от текущего ответа и $\displaystyle B$ - вершины на расстоянии $\displaystyle 2$. 

Изначально ответной вершиной будем считать $\displaystyle 1$. Будем перебирать вершины по очереди. Если вершина на расстоянии $\displaystyle 1$ или $\displaystyle 2$ от текущего ответа, то добавим её в соответствующее множество. Если же это не так, то сделаем её новый ответом. Тогда предыдущий ответ добавится в $\displaystyle A$ (так как в неё было ребро), а остальные элементы $\displaystyle A$ и $\displaystyle B$ останутся прежними, так как у нового ответа обязательно есть рёбра во все вершины $\displaystyle A$, иначе был бы путь длины $\displaystyle 2$.

Так в конце мы найдём искомую вершину.

\begin{center}
\textbf{Задача 5}
\end{center}
Докажем по индукции:

База: для 1 вершины путь есть, он состоит из одной вершины

Переход $\displaystyle n\rightarrow n+1$: пусть быть путь $\displaystyle a$ и в граф добавилась вершина $\displaystyle v$. Тогда если есть ребро $\displaystyle v\rightarrow a_{1}$, то добавляем вершину в начало пути и получаем новый путь. Если есть ребро $\displaystyle a_{n}\rightarrow v$, то мы можем получить новый путь, добавив вершину в конец.

Если мы пока не смогли получить новый путь, то в графе есть рёбра $\displaystyle \left( a_{1}\rightarrow v\right)$ и $\displaystyle \left( v\rightarrow a_{n}\right)$. Также для каждого $\displaystyle i$ есть ребро между $\displaystyle a_{i}$ и $\displaystyle v$, направленное в какую-то сторону. Но между вершинами с ребром в $\displaystyle v$ и ребром из $\displaystyle v$ должен быть переход, так как начало и конец пути имеют разный тип. То есть существует такое $\displaystyle i$, что есть рёбра $\displaystyle \left( a_{i}\rightarrow v\right)$ и $\displaystyle \left( v\rightarrow a_{i+1}\right)$. Тогда просто вставляем вершину $\displaystyle v$ между этими двумя элементами пути и переход доказан.\qed 

\begin{center}
\textbf{Задача 6}
\end{center}
Дан граф, в котором есть простые циклы нечётной длины, но после удаления любого ребра он пропадает. Это значит, что все рёбра графа представляют собой один нечётный цикл, так как если есть 2 несовпадающих цикла, то есть ребро, которое принадлежит одному циклу и не принадлежит другому, а значит условие не выполняется.

Если в графе нет изолированных вершин, и при этом все рёбра графа образуют цикл, то этот цикл будет такой же длины, сколько вершин будет в графе. А если вершин 1000, то цикл будет чётным, а значит граф изначально был 2-раскрашиваемым.

Значит, в графе есть изолированные вершины.\qed 

\begin{center}
\textbf{Задача 7}
\end{center}
Построим граф состояний, где вершина отвечает за количество камней и последнюю коробку, в которую попал камень в предыдущий ход. Ребро из вершины в другую будет обозначать ход в игре. Из каждой вершины выходит по 1 ребру.

Рассмотрим как выглядит ход: сначала мы достаём все камни из одной коробки, а после раскладываем их по часовой стрелке. Тогда если мы знаем, в какую коробку попал последний камень, то мы можем убирать камни по одному, идя против часовой стрелке. Так мы в какой-то момент придём в коробку, где нет камней, и это будет та коробка, из которой достали камни в предыдущий ход, так как если мы забрали меньше камней, чем положили на прошлом ходе, то во всех коробках обязательно должны быть камни, так как мы их туда положили.

Итого, по расположению камней и последней коробке мы можем однозначно определить, из какой позиции мы пришли, а значит граф ходов распадается на циклы, и если мы начнём идти из какой-то вершины, то мы обязательно в неё вернёмся.

Стоит заметить, что мы можем сделать любой первых ход, а значит вариантов для начальной вершины несколько, но мы всё равно вернёмся в начало (может даже раньше попадём в состояние с такой же расстановкой камней, но другой последней коробкой, но на решение это не влияет).

\begin{center}
\textbf{Задача 8}
\end{center}
Пусть мы смогли пройти по полю. Сделаем граф, где вершины клеток поля ($\displaystyle 9\cdotp 9$ штук) будут вершинами. Тогда если кубик перекатился через границу между двумя клетками поля, то проведём ребро по этой границе. Если в графе есть вершина степени 3, то мы получаем противоречие, так как в таком случае кубик попал на 2 соседние клетки одной стороной (кубик входит и выходит из каждой клетки по 1 разу кроме первой и последней, поэтому кубик точно прокатился через эти 3 границы клеток подряд).

Пусть мы смогли расставить рёбра так, что нет вершин степени 2. Всего кубик сделает $\displaystyle 63$ переката, поэтому сумма степеней всех вершин будет $\displaystyle 126$. С другой стороны, вершины на краях доски (по 7 на каждой из 4 сторон) могут иметь степень максимум $\displaystyle 1$, а вершины внутри поля ($\displaystyle 7\cdotp 7$ штук) могут иметь максимум $\displaystyle 2$ ребра. Тогда суммарно концов рёбер будет $\displaystyle 7\cdotp 4+7\cdotp 7\cdotp 2=126$. Итого получаем, что если расстановка без вершин степени 3 есть, то в ней у каждой вершины будет максимальная возможная степень.

Теперь раскрасим узлы в 2 цвета в сине-белую шахматную раскраску.

Любое ребро будет соединять синюю и белую вершину, значит суммарная степень синих вершин должна быть равна $\displaystyle 63$. Но посчитаем её: $\displaystyle 4\cdotp 4+( 3\cdotp 4+4\cdotp 3) \cdotp 2=16+48=64$. Получается, что есть ребро из синей клетки в синюю. Противоречие. Значит в графе точно есть вершина степени 3, а значит мы не можем пройтись кубиком. \qed 
