% !TEX options=--shell-escape
\documentclass{article}
\usepackage{tikz,amsthm,amsmath,cancel,pgfplots,animate,multirow,unicode-math,adjustbox,booktabs,array,pst-solides3d,pst-all,pst-3dplot,color,colortbl}
\usepackage[margin=96pt]{geometry}

\usetikzlibrary{automata,positioning,calc,decorations.pathmorphing,patterns,external}
\tikzexternalize[prefix=external/]

\newtheorem{problem}{Задача}
\newtheorem{lemma}{Лемма}
\newtheorem{theorem}{Теорема}
\newtheorem{definition}{Определение}
\newtheorem{example}{Пример}
\renewcommand*{\proofname}{Доказательство}
\newcommand{\const}{\mathrm{const}}
\DeclareMathOperator{\ord}{ord}
\DeclareMathOperator{\maj}{MAJ}
\newcommand{\abs}[1]{\left\lvert#1\right\rvert}
\newcommand{\floor}[1]{\left\lfloor#1\right\rfloor}
\newcommand{\ceil}[1]{\left\lceil#1\right\rceil}
\newcommand{\fr}[1]{\left\{#1\right\}}
\newcommand{\N}{\mathbb{N}}
\newcommand{\Z}{\mathbb{Z}}
\newcommand{\Q}{\mathbb{Q}}
\newcommand{\R}{\mathbb{R}}
\newcommand{\C}{\mathbb{C}}

\newcolumntype{o}{>{\columncolor{orange}}c}

\makeatletter
\newenvironment{sqcases}{%
	\matrix@check\sqcases\env@sqcases
}{%
	\endarray\right.%
}
\def\env@sqcases{%
	\let\@ifnextchar\new@ifnextchar
	\left\lbrack
	\def\arraystretch{1.2}%
	\array{@{}l@{\quad}l@{}}%
}
\makeatother

\setlength{\parindent}{0pt}
\setlength{\parskip}{5pt}
\setmainfont{CMU Serif}
\widowpenalties 1 10000
\raggedbottom

\date{9 октября, 2022}
\title{Дискретная математика \\ \Large Домашнее задание 5}
\author{Иван Мачуговский}

\AtBeginDocument{
	\renewcommand{\setminus}{\mathbin{\backslash}}
}

\begin{document}
	\maketitle

	\begin{section}{Задача 1}
		а)

		\begin{equation*}
			\abs{S \cap (T_1 \setminus L)} = \abs{S \cap T_1} - \abs{S \cap T_1 \cap L}.
		\end{equation*}

		Первая половина таблицы истинности любой самодвойственной функции ($x_1 = 0$) однозначно восстанавливается по второй половине ($x_1 = 1$). $T_1$ добавляет ограничение на значение последней строки таблицы истинности. Итого независимых строк в таблице истинности $2^{n-1}-1$, поэтому

		\begin{equation*}
			\abs{S \cap T_1} = 2^{2^{n-1}-1}.
		\end{equation*}

		Линейные функции представляются в виде

		\begin{equation*}
			f(x_1, \dots, x_n) = a_0 \oplus a_1 x_1 \oplus \dots \oplus a_n x_n,
		\end{equation*}

		где $a_i$ -- константы. Условие $f \in S$ добавляет требование

		\begin{multline*}
			\neg f(\neg x_1, \dots, \neg x_n) = (a_0 \oplus 1) \oplus a_1 (x_1 \oplus 1) \oplus \dots \oplus a_n (x_n \oplus 1) = \\
			= (a_0 \oplus a_1 x_1 \dots \oplus a_n x_n) \oplus (1 \oplus a_1 \oplus \dots \oplus a_n) = \\
			= f(x_1, \dots, x_n) \oplus (1 \oplus a_1 \oplus \dots \oplus a_n) = f(x_1, \dots, x_n),
		\end{multline*}

		то есть

		\begin{equation*}
			a_1 \oplus \dots \oplus a_n = 1,
		\end{equation*}

		а условие $f \in T_1$ представляется как

		\begin{equation*}
			f(1, \dots, 1) = a_0 \oplus a_1 \oplus \dots \oplus a_n = 1.
		\end{equation*}

		Соответственно, $a_0 = 0$, а $a_1$ единственным образом определяется по $a_2, \dots, a_n$, поэтому свободных коэффициентов $n - 1$, откуда

		\begin{equation*}
			\abs{S \cap T_1 \cap L} = 2^{n-1},
		\end{equation*}

		итого

		\begin{equation*}
			\abs{S \cap (T_1 \setminus L)} = 2^{2^{n-1}-1} - 2^{n-1}.
		\end{equation*}

		б) $F$ содержит функцию $\neg x$, поскольку она линейна и самодвойственна, а также $x \land y$, поскольку она монотонна и не самодвойственна. Эти две функции образуют полную систему, поэтому $F$ полна.
	\end{section}

	\begin{section}{Задача 2}
		\begin{equation*}
			x_1 \oplus x_2 \oplus x_3 = ((x_1 \lor x_2 \lor x_3) \land \neg (x_1 x_2 \lor x_1 x_3 \lor x_2 x_3)) \lor x_1 x_2 x_3,
		\end{equation*}

		поскольку

		\begin{gather*}
			a = (x_1 \lor x_2 \lor x_3) = [\text{среди } x_1, x_2, x_3 \text{ 1, 2 или 3 единицы}], \\
			b = (x_1 x_2 \lor x_1 x_3 \lor x_2 x_3) = [\text{среди } x_1, x_2, x_3 \text{ 2 или 3 единицы}], \\
			c = (x_1 x_2 x_3) = [\text{среди } x_1, x_2, x_3 \text{ 3 единицы}],
		\end{gather*}

		откуда

		\begin{equation*}
			(a \land \neg b) \lor c = [\text{среди } x_1, x_2, x_3 \text{ 1 или 3 единицы}].
		\end{equation*}
	\end{section}

	\begin{section}{Задача 3}
		Да, это так.

		Обозначим $n = 2022$.

		Рассмотрим отображение $\Psi: M \to (P_2 \setminus M)$, построенное следующим образом. Если $f(x_1, \dots, x_n) \in M$ -- не тождественный нуль или единица, то

		\begin{equation*}
			\Psi(f)(x_1, \dots, x_n) = \neg f(x_1, \dots, x_n).
		\end{equation*}

		Легко видеть, что эта функция не монотонна, потому что если $f$ -- не константа, то $f(0, \dots, 0) = 0$ и $f(1, \dots, 1) = 1$, откуда $\Psi(f)(0, \dots, 0) = 1$, $\Psi(f)(1, \dots, 1) = 0$, откуда следует немонотонность.

		Тогда $\Psi$ инъективна. В самом деле, если $f \ne g$, то $\Psi(f) = \neg f \ne \neg g = \Psi(g)$.

		Для констант же определим

		\begin{equation*}
			\Psi(0)(x_1, \dots, x_n) = x_1 \oplus x_2, \ \Psi(1)(x_1, \dots, x_n) = x_2 \oplus x_3.
		\end{equation*}

		Легко видеть, что эти функции также немонотонны.

		С этим доопределением $\Psi$ остается инъективной, потому что если $f \ne 0$ и

		\begin{equation*}
			\Psi(0) = \Psi(f),
		\end{equation*}

		то либо $f = 1$, что неправда, поскольку $x_1 \oplus x_2 \ne x_2 \oplus x_3$, либо $f$ монотонна и не является константой, но тогда

		\begin{equation*}
			f = \neg \Psi(f) = \neg \Psi(0) = x_1 \oplus x_2 \oplus 1,
		\end{equation*}

		что не является монотонной функцией. Аналогично проверяется, что $\Psi(1) \ne \Psi(f)$.

		Раз $\Psi$ инъективна, то

		\begin{equation*}
			\abs{M} \le \abs{P_2 \setminus M} \implies \abs{M} \le \frac{\abs{P_2}}{2}.
		\end{equation*}

		Методами, подобными используемыми в решении задачи 1.а), можно показать, что

		\begin{gather*}
			\abs{T_0} = \abs{T_1} = 2^{2^n-1} = \frac{\abs{P_2}}{2}, \\
			\abs{L} = 2^{n+1} < 2^{2^n-1} = \frac{\abs{P_2}}{2}, \\
			\abs{S} = 2^{2^{n-1}} < 2^{2^n-1} = \frac{\abs{P_2}}{2}.
		\end{gather*}

		Итак, мощность каждого из классов $T_0, T_1, L, S, M$ не превосходит половины мощности $P_2$, поэтому если выбрать более половины функций из $P_2$, полученная система $F$ не будет целиком включаться ни в один из этих классов, откуда по критерию Поста следует, что $F$ полна.
	\end{section}

	\begin{section}{Задача 4}
		Проанализируем по отдельности принадлежность каждой из данных функций стандартным классам:

		\begin{gather*}
			f_1 = x \oplus y \in T_0, \not\in T_1, \not\in S, \not\in M, \in L, \\
			f_2 = x \oplus y \oplus z \oplus 1 \not\in T_0, \not\in T_1, \in S, \not\in M, \in L, \\
			f_3 = (x \land y) \oplus z \in T_0, \not\in T_1, \not\in S, \not\in M, \not\in L, \\
			f_4 = \maj(x, y, z) \in T_0, \in T_1, \in S, \in M, \not\in L.
		\end{gather*}

		Базис обязательно должен содержать функцию не из $T_0$, поэтому $f_2$ обязательно лежит в базисе.

		Сама по себе $f_2$ одноэлементный базис не образует, так как $f_2 \in L$.

		Переберем двухэлементные системы. Система $\{f_2, f_3\}$ не вложена ни в один из классов и потому образует базис. Следовательно, другие базисы $f_3$ не содержат. Системы $\{f_1, f_2\} \subseteq L$ и $\{f_2, f_4\} \subseteq S$ базис не образуют.

		Помимо рассмотренных систем, потенциальный базис, не содержащий $f_3$, единственный -- $\{f_1, f_2, f_4\}$. Эта система не вложена ни в один из классов и потому образует базис.

		Итак, базисы:

		\begin{equation*}
			\{f_2, f_3\}, \{f_1, f_2, f_4\}.
		\end{equation*}
	\end{section}

	\begin{section}{Задача 5}
		Формально, мы хотим показать, что

		\begin{equation*}
			((0 \not\in [F]) \land (1 \not\in [F])) \iff ((F \subseteq T_0 \cap T_1) \lor F \subseteq S).
		\end{equation*}

		Если правая сторона выполняется, то есть $F \subseteq T_0 \cap T_1$ или $F \subseteq S$, то $[F] \subseteq T_0 \cap T_q$ или $[F] \subseteq S$. Константы $0$ и $1$ не попадают ни в $S$, ни в $T_0 \cap T_1$, поскольку $0 \not\in T_1, 1 \not\in T_0$, поэтому они не попадают и в их подмножество $[F]$ в каждом из случаев, что и требовалось.

		С другой стороны, если правая сторона не выполняется, то есть $F \not\subseteq S$ и $F \not\subseteq T_0 \cap T_1$, то существует не самодвойственная функция $f_S \in F$ и хотя бы одна из функций $f_0, f_1 \in F$, где $f_0$ не сохраняет нуль, а $f_1$ не сохраняет единицу. Требуется показать, что и левая сторона не выполняется, то есть что $0 \in [F]$ или $1 \in [F]$.

		Если существует $f_0$, причем $f_0(1, \dots, 1) = 1$, то в силу того, что $f_0 \not\in T_0$, имеем $f_0(0, \dots, 0) = 1$, а следовательно, $f_0(x, \dots, x)$ представляет собой константу $1$, что и требовалось, и на этом доказательство завершается. Остался случай, когда $f_0(1, \dots, 1) = 0$, откуда следует $f_0(x, \dots, x) = \neg x$.

		Аналогично, если существует $f_1$, причем $f_1(0, \dots, 0) = 0$, то в силу того, что $f_1 \not\in T_1$, имеем $f_1(1, \dots, 1) = 0$, а следовательно, $f_1(x, \dots, x)$ представляет собой константу $0$, что, опять же, и требовалось. Остался случай, когда $f_1(0, \dots, 0) = 1$, откуда следует $f_1(x, \dots, x) = \neg x$.

		Если доказательство к текущему моменту еще не завершено, в обоих вариантах мы уже построили $\neg x$. Вспомним о существовании $f_S$. $f_S \not\in S$ означает, что для некоторого набора значений $y_1, \dots, y_n$ имеет место равенство

		\begin{equation*}
			f_S(y_1, \dots, y_n) = f_S(\neg y_1, \dots, \neg y_n).
		\end{equation*}

		Это эквивалентно

		\begin{equation*}
			f_S(0 \oplus y_1, \dots, 0 \oplus y_n) = f_S(1 \oplus y_1, \dots, 1 \oplus y_n),
		\end{equation*}

		а значит, следующая функция от переменной $x$ является константой:

		\begin{equation*}
			f_S(x \oplus y_1, \dots, x \oplus y_n).
		\end{equation*}

		$y_1, \dots, y_n$ -- фиксированные значения, поэтому выражения $x \oplus y_i$ представляются в виде $x$ или $\neg x$ в зависимости от конкретных значений. $\neg x$ представимо в $[F]$, поэтому и эта функция представима в $[F]$, а следовательно, $[F]$ содержит константу, что и требовалось.
	\end{section}

	\begin{section}{Задача 6}
		Базис мощности более $5$ невозможен. В самом деле, если $F$ -- некоторый базис $P_2$, то он содержит функции не из $T_0$, не из $T_1$, не из $L$, не из $M$ и не из $S$. Но подмножество из этих пяти функций (или менее, если среди них есть повторы) само по себе образует полную систему, поэтому мощность базиса не может превосходить $5$.

		Более того, мощность базиса не может быть равна $5$: это бы требовало наличия пяти функций, каждая из которых не лежит ровно в одном из классов. Тогда найдется функция $f_1 \not\in T_1$, лежащая в $T_0, L, M, S$. Но если

		\begin{gather*}
			f_1 \in T_0 \implies f_1(0, 0, \dots, 0) = 0, \\
			f_1 \in S \implies f_1(1, 1, \dots, 1) = \neg f_1(0, 0, \dots, 0),
		\end{gather*}

		то получаем $f_1(1, 1, \dots, 1) = 1$, откуда $f_1 \in T_1$ -- противоречие.

		Возможен базис мощности $1$: например, штрих Шеффера (NAND) образует полную систему.

		Возможен базис мощности $2$: например, знакомая всем система $\{\neg x, x \land y\}$ является полной, а ее одноэлементные подмножества $\{\neg x\} \subseteq S$ и $\{x \land y\} \subseteq M$ полную систему не образуют.

		Возможен базис мощности $3$: например, система $\{1, x \oplus y, x \land y\}$ образуют полную систему, поскольку

		\begin{gather*}
			1 \not\in T_0, \\
			x \oplus y \not\in T_1, S, M, \\
			x \land y \not\in L.
		\end{gather*}

		Если убрать первую операцию, то оставшиеся две обе сохраняют нуль, если убрать вторую -- оставшиеся две обе сохраняют сохраняют единицу, если убрать третью -- оставшиеся две обе линейны, поэтому это базис.

		Возможен базис мощности $4$: например, система $\{0, 1, x \land y, x \oplus y \oplus z\}$ образует полную систему, поскольку

		\begin{gather*}
			0 \not\in T_1, S \\
			1 \not\in T_0, S \\
			x \land y \not\in L, \\
			x \oplus y \oplus z \not\in M.
		\end{gather*}

		Если убрать первую операцию, то оставшиеся три сохраняют единицу, если убрать вторую -- оставшиеся три сохраняют сохраняют нуль, если убрать третью -- оставшиеся три линейны, если убрать четвертую -- оставшиеся три монотонны, поэтому это базис.
	\end{section}

	\begin{section}{Задача 7}
        Функция голосования от трех аргументов - самодвойственная, а функции голосований от четного количество аргументов нет. Получается, что если мы имеем функции голосования от трех переменных мы никак из нее не соберем функцию которая имеет четное количество аргументов.

Теперь, давайте докажем, что имея функцию от трех переменных мы сможем собрать любую функцию которая принимает не четное количество переменных. Докажем это по индукции. Базой у нас будет функция голосования от трех переменных.

Переход. Допустим, мы научились получать функцию голосования от $n$ переменных, покажем, как получать функцию голосования от $n+2$ переменных. Я утверждаю, что функция голосования от $n+2$ может выглядеть так: $MAJ_{n+2} = MAJ_3(MAJ_n(x_1, x_2, x_3, ..., x_n),\\ MAJ_n(MAJ_n(x_2, ... x_{n+1}), MAJ_n(x_1, x_3, ... x_{n+1}), ..., MAJ_n(x_1, ..., x_{n-1}, x_{n+1})),\\
MAJ_n(MAJ_n(x_2, ... x_{n+2}), MAJ_n(x_1, x_3, ... x_{n+2}), ..., MAJ_n(x_1, ..., x_{n-1}, x_{n+2})))$. Достаточно просто увидеть, что если разница между количество ноликов и единичек в переменных: $x_1, x_2, x_3, ..., x_n$ по модулю больше одного, т
то эта функция всегда выведет правильный ответ. Остальные варианты можно просто перебрать, но я не вижу смысла это записывать.
	\end{section}
\end{document}
