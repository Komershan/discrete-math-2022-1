% !TEX options=--shell-escape
\documentclass{article}
\usepackage{tikz,amsthm,amsmath,cancel,pgfplots,animate,multirow,unicode-math,adjustbox,booktabs,array,pst-solides3d,pst-all,pst-3dplot}
\usepackage[margin=96pt]{geometry}

\usetikzlibrary{automata,positioning,calc,decorations.pathmorphing,patterns,external}
\tikzexternalize[prefix=external/]

\newtheorem{problem}{Задача}
\newtheorem{lemma}{Лемма}
\newtheorem{theorem}{Теорема}
\newtheorem{definition}{Определение}
\newtheorem{example}{Пример}
\renewcommand*{\proofname}{Доказательство}
\newcommand{\const}{\mathrm{const}}
\newcommand{\abs}[1]{\lvert#1\rvert}
\newcommand{\floor}[1]{\lfloor#1\rfloor}
\newcommand{\ceil}[1]{\lceil#1\rceil}
\newcommand{\N}{\mathbb{N}}
\newcommand{\Z}{\mathbb{Z}}
\newcommand{\Q}{\mathbb{Q}}
\newcommand{\R}{\mathbb{R}}
\newcommand{\C}{\mathbb{C}}

\makeatletter
\newenvironment{sqcases}{%
	\matrix@check\sqcases\env@sqcases
}{%
	\endarray\right.%
}
\def\env@sqcases{%
	\let\@ifnextchar\new@ifnextchar
	\left\lbrack
	\def\arraystretch{1.2}%
	\array{@{}l@{\quad}l@{}}%
}
\makeatother

\setlength{\parindent}{0pt}
\setlength{\parskip}{5pt}
\setmainfont{CMU Serif}
\widowpenalties 1 10000
\raggedbottom

\date{15 сентября, 2022}
\title{Дискретная математика \\ \Large Домашнее задание 2}
\author{Иван Мачуговский}

\AtBeginDocument{
	\renewcommand{\setminus}{\mathbin{\backslash}}
}

\begin{document}
	\maketitle

	\begin{section}{Задача 1}
		\begin{multline*}
			(x, y) \in (A_1 \setminus A_2) \times (B_1 \setminus B_2)
			\iff
			\begin{cases}
				x \in A_1 \setminus A_2 \\
				y \in B_1 \setminus B_2
			\end{cases}
			\iff
			\begin{cases}
				x \in A_1, x \not\in A_2 \\
				y \in B_1, y \not\in B_2
			\end{cases}
			\implies \\ \implies
			\begin{cases}
				x \in A_1, y \in B_1, \\
				x \not\in A_2 \lor y \not\in B_2
			\end{cases}
			\iff
			\begin{cases}
				(x, y) \in A_1 \times B_1, \\
				(x, y) \not\in A_2 \times B_2 \\
			\end{cases}
			\iff
			(x, y) \in (A_1 \times B_1) \setminus (A_2 \times B_2).
		\end{multline*}

		Единственный не эквивалентный переход отмечен $\implies$. Для равенства множеств требуется, чтобы

		\begin{equation*}
			\begin{cases}
				x \in A_1, x \not\in A_2 \\
				y \in B_1, y \not\in B_2
			\end{cases}
			\iff
			\begin{cases}
				x \in A_1, y \in B_1, \\
				x \not\in A_2 \lor y \not\in B_2
			\end{cases}.
		\end{equation*}

		Слева направо импликация верна всегда, справа налево она верна при условии

		\begin{gather*}
			\forall x \in (A_1 \setminus A_2), y \in B_1: y \not\in B_2; \\
			\forall x \in A_1, y \in (B_1 \setminus B_2): x \not\in A_2.
		\end{gather*}

		Первое выражение истинно либо если $A_1 \setminus A_2$ пусто (тривиальный случай), либо если $\forall y \in B_1: y \not\in B_2$, то есть $B_1 \cap B_2$ пусто. Аналогично со вторым выражением. Итого

		\begin{equation*}
			(A_1 \subseteq A_2 \lor B_1 \cap B_2 = \emptyset) \land (A_1 \cap A_2 = \emptyset \lor B_1 \subseteq B_2)
		\end{equation*}
	\end{section}

	\begin{section}{Задача 2}
		% Если $(x, y) \in R$, то $\frac{x}{y} > 0$, то есть $x$ и $y$ оба ненулевые и одинаковых знаков.

		Отношение $R$ рефлексивно при $x \ne 0$, поэтому:

		\begin{equation*}
			(x, y) \in R
			\implies
			\begin{cases}
				(x, y) \in R \\
				x \ne 0
			\end{cases}
			\implies
			\begin{cases}
				(x, x) \in R \\
				(x, y) \in R
			\end{cases}
			\implies
			(x, y) \in R \circ R.
		\end{equation*}

		И обратно:

		\begin{equation*}
			(x, y) \in R \circ R
			\implies
			\exists z \in \R: \frac{x}{z} > 0, \frac{z}{y} > 0
			\implies
			\exists z \in \R: \frac{x}{\cancel{z}} \cdot \frac{\cancel{z}}{y} > 0
			\implies
			\frac{x}{y} > 0
			\implies
			(x, y) \in R.
		\end{equation*}

		Поэтому $R \circ R = R$.
	\end{section}

	\begin{section}{Задача 3}
		Пусть для некоторой последовательности (индексируемой произвольными объектами) множеств $A_{\lambda \in \Lambda}$ и функций $F$, $G$ из последовательности множеств в множество верно

		\begin{equation*}
			F(A) \ne G(A).
		\end{equation*}

		Тогда из определения равенства множеств следует, что

		\begin{equation*}
			\exists x: (x \in F(A)) \ne (x \in G(A)).
		\end{equation*}

		Обозначим $B_\lambda = A_\lambda \cap \{x\}$, тогда утверждается, что

		\begin{equation*}
			(x \in F(B)) \ne (x \in G(B)) \implies F(B) \ne G(B).
		\end{equation*}

		В самом деле, для этого достаточно показать, что

		\begin{equation*}
			x \in F(B) \iff x \in F(A).
		\end{equation*}

		Это можно сделать по индукции по числу операций в $F$. База -- ноль операций, тогда

		\begin{equation*}
			x \in B_\lambda
			\iff
			x \in A_\lambda \cap \{x\}
			\iff
			\begin{cases}
				x \in A_\lambda \\
				x \in \{x\}
			\end{cases}
			\iff
			x \in A_\lambda.
		\end{equation*}

		Шаг индукции:

		\begin{enumerate}
			\item Если $F(X) = F_a(X) \cap F_b(X)$, имеем

			\begin{equation*}
				x \in (F_a(A) \cap F_b(A))
				\iff
				\begin{cases}
					x \in F_a(A) \\
					x \in F_b(A)
				\end{cases}
				\iff
				\begin{cases}
					x \in F_a(B) \\
					x \in F_b(B)
				\end{cases}
				\iff
				x \in (F_a(B) \cap F_b(B)).
			\end{equation*}

			\item Если $F(X) = F_a(X) \cup F_b(X)$, имеем

			\begin{equation*}
				x \in (F_a(A) \cup F_b(A))
				\iff
				\begin{sqcases}
					x \in F_a(A) \\
					x \in F_b(A)
				\end{sqcases}
				\iff
				\begin{sqcases}
					x \in F_a(B) \\
					x \in F_b(B)
				\end{sqcases}
				\iff
				x \in (F_a(B) \cup F_b(B)).
			\end{equation*}

			\item Если $F(X) = F_a(X) \setminus F_b(X)$, имеем

			\begin{equation*}
				x \in (F_a(A) \setminus F_b(A))
				\iff
				\begin{cases}
					x \in F_a(A) \\
					x \not\in F_b(A)
				\end{cases}
				\iff
				\begin{cases}
					x \in F_a(B) \\
					x \not\in F_b(B)
				\end{cases}
				\iff
				x \in (F_a(B) \setminus F_b(B)).
			\end{equation*}
		\end{enumerate}
	\end{section}

	\begin{section}{Задача 4}
		Т.к. $f(x) \in A \iff x \in f^{-1}(A)$ по определению прообраза множества, имеем

		\begin{multline*}
			x \in f^{-1}(A \triangle B)
			\iff
			f(x) \in A \triangle B
			\iff
			\begin{sqcases}
				\begin{cases}
					f(x) \in A \\
					f(x) \not\in B
				\end{cases} \\
				\begin{cases}
					f(x) \not\in A \\
					f(x) \in B
				\end{cases}
			\end{sqcases}
			\iff \\ \iff
			\begin{sqcases}
				\begin{cases}
					x \in f^{-1}(A) \\
					x \not\in f^{-1}(B)
				\end{cases} \\
				\begin{cases}
					x \not\in f^{-1}(A) \\
					x \in f^{-1}(B)
				\end{cases}
			\end{sqcases}
			\iff
			x \in f^{-1}(A) \triangle f^{-1}(B),
		\end{multline*}

		поэтому оба знака корректны.
	\end{section}

	\begin{section}{Задача 5}
		Покажем, что $f^n$ (под возведением функции в степень здесь понимается композиция ее с самой собой $n$ раз) на отрезке $[0; 1]$ представляет из себя последовательность из $2^n$ отрезков монотонности, где на концах каждого промежутка одно из значений равно $0$, а другое -- $1$.

		Доказательство проведем по индукции. База: $n = 1$, тогда $f^1 = f$ действительно удовлетворяет условию, поскольку $f$ -- парабола с ветвями вниз и вершиной $(1/2, 1)$, а также $f(0) = f(1) = 0$.

		Для перехода от $n$ к $n + 1$ нам понадобится очевидная

		\begin{lemma}
			Если некоторая функция $g: [0; 1] \to \mathbb{R}$ имеет промежуток монотонности $[a, b]$, а функция $h: [l, r] \to [0; 1]$ сюръективна и монотонно возрастает, то $g \circ h$ имеет промежуток монотонности $[h^{-1}(a), h^{-1}(b)]$ с теми же значениями в крайних точках.
		\end{lemma}

		Следовательно, если функция $f^n$ имела промежуток монотонности $[a, b]$, то, подставляя в лемму $g = f^n$, $h: [0; 1/2] \to [0; 1] = 4x(1-x)$, получаем, что $f^{n+1}$ имеет промежуток монотонности $[h^{-1}(a), h^{-1}(b)]$, причем один из концов сохраняется единичным, другой -- нулевым. Применяя эту лемму ко всем $2^n$ промежуткам монотонности $f^n$, получаем, что $f^{n+1}$ имеет $2^n$ промежутков монотонности на $[0; 1/2]$.

		Аналогично можно показать, что на отрезке $[1/2; 1]$ функция $f^{n+1}$ также имеет $2^n$ промежутков монотонности. Итого промежутков монотонности $2^{n+1}$. \qed

		Итак, рассмотрим произвольный промежуток монотонности $[a; b]$ функции $g = f^{2022}$. Как было показано, $g(a) = 0, g(b) = 1$ либо $g(a) = 1, g(b) = 0$. Рассмотрим первый случай, второй рассматривается аналогично. Если $g(a) = 0$, то $a - g(a) \ge 0$; если $g(b) = 1$, то $b - g(b) \le 0$. $x - g(x)$ -- непрерывная функция, поэтому найдется $x \in [a; b], g(x) - x = 0$, следовательно, $x$ -- одно из искомых решений. Легко видеть, что никаким двум промежуткам не может соответствовать один и тот же $x$, потому что если это так, то $x = a$ либо $x = b$, откуда $f(x) \in \{0, 1\}$, откуда $x \in \{0, 1\}$, но промежуток с концом $0$ единственен, как и промежуток с концом $1$. Итак, каждому промежутку монотонности соответствует свое решение (не обязательно единственное), а значит, решений как минимум $2^{2022}$.

		С другой стороны, $f^n(x)$ -- многочлен степени $2^n$, поскольку $\deg f^1(x) = 2$ и

		\begin{equation*}
			\deg f^{n+1}(x) = \deg 4 f^n(x) (1 - f^n(x)) = \deg (f^n(x))^2 = 2 \deg f^n(x).
		\end{equation*}

		Следовательно, $f^{2022}(x) - x$ может иметь не более $2^{2022}$ корней.

		Итак, решений не менее $2^{2022}$ и не более $2^{2022}$, значит, их ровно столько.
	\end{section}

	\begin{section}{Задача 6}
		Нет, докажем это от противного. Предположим, существует функция $f: \N \to \N$, удовлетворяющая условию $f^2(n) = n + 1$ (здесь под $f^k$ понимается композиция функции с самой собой $k$ раз).

		Отсюда следует, что $f^{2k}(n) = n + k$.

		Обозначим $a = f(1)$. Тогда

		\begin{align*}
			f^{2a-1}(1) &= f^{2(a-1)}(a) = a + (a - 1) = 2a - 1; \\
			f^{2a-1}(1) &= f(f^{2(a-1)}(1)) = f(1 + (a - 1)) = f(a) = f^2(1) = 2.
		\end{align*}

		Но $2a - 1$ и $2$ имеют разную четность и потому не могут быть равны. \qed
	\end{section}

	\begin{section}{Задача 7}
		Обозначим за $\{a_n\}$ последовательность всех чисел $a \in \N$, удовлетворяющих $\sqrt{a} \not\in \N$, упорядоченную по возрастанию.

		Тогда каждое натуральное число $n$, кроме единицы, представимо в виде $a_i^{2^k}$ для некоторых $i, k$. В самом деле, если $n$ -- минимальное непредставимое число, то возможны два случая. Если $\sqrt{n} \not\in \N$, то $n$ есть в последовательности $a$, и подставив $k = 0$, мы получаем требуемое. Если же $\sqrt{n} \in \N$, то в силу минимальности $n$ число $\sqrt{n}$ представимо как $a_i^{2^k}$, тогда $n = a_i^{2^{k+1}}$.

		Легко видеть и обратное: каждое число, кроме единицы, представимо в таком виде единственным образом, который можно выразить конструктивно, итеративно беря квадратный корень от $n$, пока он представим в натуральных числах. Единица же не представима никак.

		Тогда определим $f$ следующим образом:

		\begin{align*}
			f(1) &= 1, \\
			f \left( a_{2i-1}^{2^k} \right) &= a_{2i}^{2^k}, \\
			f \left( a_{2i}^{2^k} \right) &= a_{2i-1}^{2^{k+1}}.
		\end{align*}

		Тогда

		\begin{equation*}
			(f \circ f)(x) = \begin{cases}
				1, & \text{если } x = 1 \\
				a_{2i-1}^{2^{k+1}}, & \text{если } x = a_{2i-1}^{2^k} \\
				a_{2i}^{2^{k+1}}, & \text{если } x = a_{2i}^{2^k}
			\end{cases},
		\end{equation*}

		то есть $(f \circ f)(x) = x^2$. \qed
	\end{section}
\end{document}
