% !TEX options=--shell-escape
\documentclass{article}
\usepackage{tikz,amsthm,amsmath,cancel,pgfplots,animate,multirow,unicode-math,adjustbox,booktabs,array,pst-solides3d,pst-all,pst-3dplot,color,colortbl,braket,mathtools}
\usepackage[margin=96pt]{geometry}

\usetikzlibrary{automata,positioning,calc,decorations.pathmorphing,patterns,external}
\tikzexternalize[prefix=external/]

\newtheorem{problem}{Задача}
\newtheorem{lemma}{Лемма}
\newtheorem{theorem}{Теорема}
\newtheorem{definition}{Определение}
\newtheorem{example}{Пример}
\renewcommand*{\proofname}{Доказательство}
\newcommand{\const}{\mathrm{const}}
\DeclareMathOperator{\ord}{ord}
\DeclareMathOperator{\maj}{MAJ}
\newcommand{\abs}[1]{\left\lvert#1\right\rvert}
\newcommand{\floor}[1]{\left\lfloor#1\right\rfloor}
\newcommand{\ceil}[1]{\left\lceil#1\right\rceil}
\newcommand{\fr}[1]{\left\{#1\right\}}
\newcommand{\N}{\mathbb{N}}
\newcommand{\Z}{\mathbb{Z}}
\newcommand{\Q}{\mathbb{Q}}
\newcommand{\R}{\mathbb{R}}
\newcommand{\C}{\mathbb{C}}
\newcommand{\continuum}{\mathfrak{c}}

\newcolumntype{o}{>{\columncolor{orange}}c}

\makeatletter
\newenvironment{sqcases}{%
	\matrix@check\sqcases\env@sqcases
}{%
	\endarray\right.%
}
\def\env@sqcases{%
	\let\@ifnextchar\new@ifnextchar
	\left\lbrack
	\def\arraystretch{1.2}%
	\array{@{}l@{\quad}l@{}}%
}
\makeatother

\setlength{\parindent}{0pt}
\setlength{\parskip}{5pt}
\setmainfont{CMU Serif}
\widowpenalties 1 10000
\raggedbottom

\date{21 ноября, 2022}
\title{Дискретная математика \\ \Large Домашнее задание 10}
\author{Иван Мачуговский}

\AtBeginDocument{
	\renewcommand{\setminus}{\mathbin{\backslash}}
}

\begin{document}
	\maketitle

	\section*{Задача 1}

	а) Нет. В самом деле, если удалить из графа эту вершину степени $1$ и исходящее из нее ребро, останется граф из $7$ вершин и $22$ ребер. Но в графе из $7$ вершин может быть не более $\binom{7}{2} = 21$ ребер, противоречие.

	б) Нет. В самом деле, если у трех вершин степень $4$, а у остальных семи $\ge 1$ (поскольку граф связен), то

	\begin{equation*}
		\sum_i \deg v_i \ge 3 \cdot 4 + 7 \cdot 1 = 19.
	\end{equation*}

	Но

	\begin{equation*}
		\sum_i \deg v_i = 2 \abs{E} = 2 (\abs{V} - 1) = 18,
	\end{equation*}

	противоречие.


	\section*{Задача 2}

	Обозначим высоту вершины в дереве, то есть ее расстояние от корня, за $h(v)$. Тогда в силу уникальности простого пути между любыми двумя вершинами в дереве

	\begin{equation*}
		d(u, v) = (h(u) - h(l)) + (h(v) - h(l)) = h(u) + h(v) - 2 h(l),
	\end{equation*}

	где $l$ -- самый нижний общий предок вершин $u$ и $v$. Понятно, что в бинарном дереве глубины $n$ всегда выполняется

	\begin{equation*}
		d(u, v) \le 2n.
	\end{equation*}

	Легко видеть, что для полного дерева $d(u, v) = 2n$, когда $u$ и $v$ -- листья, простой путь между которыми проходит через корень. В бинарном дереве такие листья, конечно, есть, поэтому диаметр равен $2n$.

	Если обозначить корень за $0$, а его детей за $1$ и $2$, то путь между $u$ и $v$ является диаметром, если $u$ лежит в поддереве $1$, а $v$ -- в поддереве $2$, или наоборот. Поскольку всего в дереве листьев $2^n$, способов выбрать диаметр, то есть неупорядоченную пару листьев, не лежащих одновременно в одном и том же поддереве $1$ или $2$, будет

	\begin{equation*}
		2^{n-1} \cdot 2^{n-1} = 2^{2n-2}.
	\end{equation*}


	\section*{Задача 3}

	Оценка: Обозначим ответ на эту задачу за $f(n, k)$, при этом не требуется $k \le \frac{n}{2}$. Тогда

	\begin{equation*}
		f(n, 0) = f(0, k) = 1,
	\end{equation*}

	поскольку можно взять семейство из одного пустого множества, а больше одного элемента взять нельзя.

	Если же $n, k > 0$, то оптимальное семейство $\mathcal{P}$ можно разделить на два семейства $\mathcal{P}_1, \mathcal{P}_2$, первое из которых состоит из множеств, включающих $n$, а второе -- не включающих. Если в $\mathcal{P}$ никакие два подмножества не вложены, то то же свойство сохранится для $\mathcal{P}_1$ и $\mathcal{P}_2$. Если удалить из каждого множества из $\mathcal{P}_2$ элемент $n$, свойство о невложенности также сохранится. Следовательно, $\mathcal{P}_1$ теперь удовлетворяет условию для $n-1, k$, а $\mathcal{P}_2$ -- для $n-1, k-1$, поэтому

	\begin{equation*}
		f(n, k) = \abs{\mathcal{P}} = \abs{\mathcal{P}_1} + \abs{\mathcal{P}_2} \le f(n-1, k) + f(n-1, k-1).
	\end{equation*}

	Тут уже проглядывает что-то знакомое. Но просто $\binom{n}{k}$ в качестве оценки сверху подставлять нельзя, например, потому, что не выполняется база при $n = 0, k > 0$.

	Поэтому будем по индукции по $n$ показывать, что

	\begin{equation*}
		f(n, k) \le \binom{n}{\min(k, \floor{\frac{n}{2}})}.
	\end{equation*}

	По теореме Шпернера максимальное число невложенных подмножеств (без ограничения на размер) $\binom{n}{\floor{\frac{n}{2}}}$, поэтому

	\begin{equation*}
		f(n, n) = \binom{n}{\floor{\frac{n}{2}}}.
	\end{equation*}

	$f(n, k)$, очевидно, монотонна по $k$, поэтому при всех $k$ имеем

	\begin{equation*}
		f(n, k) \le \binom{n}{\floor{\frac{n}{2}}}.
	\end{equation*}

	При $k \ge \frac{n}{2}$ это совпадает с доказываемой нами оценкой. Если же $k < \frac{n}{2}$, то $k \le \frac{n-1}{2}$ и потому

	\begin{equation*}
		f(n, k) \le f(n-1, k) + f(n-1, k-1) \le \binom{n-1}{k} + \binom{n-1}{k-1} = \binom{n}{k}.
	\end{equation*}

	Итак, оценка доказана. В нашем случае $k \le \frac{n}{2}$, поэтому верна оценка $\binom{n}{k}$.

	Пример: Пусть $\mathcal{P}$ -- семейство всех $k$-элементных подмножеств $A$. Тогда оба условия, очевидно, выполняются, и

	\begin{equation*}
		\abs{\mathcal{P}} = \binom{n}{k}.
	\end{equation*}


	\section*{Задача 4}

	Если $k$ четно, пусть

	\begin{gather*}
		V = \set{0, 1, \dots, n-1}, \\
		E = \Set{(i, (i+j) \bmod n) | 0 \le i < n, 1 \le j \le \frac{k}{2}}.
	\end{gather*}

	Это определение корректно, потому что каждое ребро здесь указано только один раз. В самом деле, две пары

	\begin{equation*}
		(i_1, (i_1+j_1) \bmod n) \text{ и } (i_2, (i_2+j_2) \bmod n)
	\end{equation*}

	могут совпадать, если они равны поэлементно:

	\begin{equation*}
		\begin{cases}
			i_1 = i_2 \\
			(i_1+j_1) \bmod n = (i_2+j_2) \bmod n
		\end{cases}
	\end{equation*}

	что, очевидно, невозможно, если только не $i_1 = i_2, j_1 = j_2$, либо если элементы совпадают крест-накрест:

	\begin{equation*}
		\begin{cases}
			i_1 = (i_2+j_2) \bmod n \\
			(i_1+j_1) \bmod n = i_2
		\end{cases}
	\end{equation*}

	но это бы означало, что

	\begin{equation*}
		j_1 + j_2 \equiv 0 \pmod n,
	\end{equation*}

	но

	\begin{equation*}
		j_1 + j_2 \le 2 \cdot \frac{k}{2} = k < n.
	\end{equation*}

	Петель в этом графе нет, т.к. $j \not\equiv 0 \pmod n$.

	Вершине $u$ инцидентны ребра вида 

	\begin{gather*}
		(u, u+j) \text{ или } (u-j, u), \ 1 \le j \le \frac{k}{2},
	\end{gather*}

	коих ровно $k$.

	Если же $k$ нечетно, но при этом $n$ четно, решим задачу для $(n, n-1-k)$ -- ее мы решать умеем, т.к. $n-1-k$ тогда будет четным. Дополнение построенного графа будет обладать требуемым свойством.

	% Если же $n$ четно и $k \le \frac{n}{2}$, пусть

	% \begin{gather*}
	% 	V = \set{0, 1, \dots, n-1}, \\
	% 	E = \Set{\left( i, \frac{n}{2} + \left( i+j \right) \bmod \frac{n}{2} \right) | 0 \le i < \frac{n}{2}, 0 \le j < k}.
	% \end{gather*}

	% Это определение корректно, потому что каждое ребро здесь указано только один раз. В самом деле, две пары

	% \begin{equation*}
	% 	\left( i_1, \frac{n}{2} + \left( i_1+j_1 \right) \bmod \frac{n}{2} \right) \text{ и } \left( i_2, \frac{n}{2} + \left( i_2+j_2 \right) \bmod \frac{n}{2} \right) \right)
	% \end{equation*}

	% могут совпадать, если они равны поэлементно:

	% \begin{equation*}
	% 	\begin{cases}
	% 		i_1 = i_2 \\
	% 		\frac{n}{2} + \left( i_1+j_1 \right) \bmod \frac{n}{2} \right) = \frac{n}{2} + \left( i_2+j_2 \right) \bmod \frac{n}{2} \right)
	% 	\end{cases}
	% \end{equation*}

	% что, очевидно, невозможно, если только не $i_1 = i_2, j_1 = j_2$, либо если элементы совпадают крест-накрест:

	% \begin{equation*}
	% 	\begin{cases}
	% 		i_1 = \frac{n}{2} + \left( i_2+j_2 \right) \bmod \frac{n}{2} \right) \\
	% 		\frac{n}{2} + \left( i_1+j_1 \right) \bmod \frac{n}{2} \right) = i_2
	% 	\end{cases}
	% \end{equation*}

	% но в первом равенстве $i_1 < \frac{n}{2}$ сравнивается с числом, не меньшим $\frac{n}{2}$. Петель в этом графе нет по той же причине.

	% Если $u < \frac{n}{2}$, вершине $u$ инцидентны ребра вида

	% \begin{equation*}
	% 	\left( u, \frac{n}{2} + \left( u+j \right) \bmod \frac{n}{2} \right),
	% \end{equation*}

	% коих $k$. Если же $u \ge \frac{n}{2}$, то вершине $u$ иницидентны ребра вида

	% \begin{gather*}
	% 	\left( \left( u - \frac{n}{2} - j \right) \bmod \frac{n}{2}, u \right),
	% \end{gather*}

	% коих также $k$.

	% Наконец, если $n$ четно и $k > \frac{n}{2}$, решим задачу для $(n, n-1-k)$, тогда дополнение графа будет обладать требуемым свойством.


	\section*{Задача 5}

	Предположим, после удаления ребер граф несвязен. Значит, его вершины можно разделить на две непустых группы, между которыми нет ребер. Поскольку всего вершин $2n + 1$, одна из этих компонент имеет размер $1 \le k \le n$, вторая -- $2n + 1 - k$. 

	Каждая из вершин первой компоненты сейчас имеет степень не более $k - 1$, а раньше имела $n$, поэтому было удалено как минимум $k (n - (k - 1))$ ребер. Но

	\begin{equation*}
		k (n - (k - 1)) \ge n
	\end{equation*}

	независимо от значения $k$, поскольку

	\begin{equation*}
		n - k (n - (k - 1)) = k^2 - (n+1)k + n = (k - n)(k - 1) \le 0 \text{ при } 1 \le k \le n,
	\end{equation*}

	то есть ребер удалили больше, чем было разрешено. Противоречие.


	\section*{Задача 6}

	Из условия следует, что между любыми двумя вершинами существует путь длины $2$, поэтому граф связен. Следовательно, достаточно показать, что степени совпадают у любых двух соседних вершин.

	Пусть $u, v$ -- вершины, между которыми проведено ребро. Пусть $x$ -- произвольный сосед $u$, не совпадающий с $v$. Тогда по условию для вершин $x$ и $v$ найдется ровно две вершины, связанные с ними обоими. Одну такую вершину мы уже нашли -- это $u$. Вторую вершину назовем $y$. Итак, по $x$ мы однозначно построили $y$. Можно сказать, что мы построили функцию из множества соседей $u$ в множество соседей $v$.

	Заметим, что это отображение инъективно. В самом деле, если для двум различным аргументам $x_1 \ne x_2$ соответствует один $y$, то с вершинами $y, u$ одновременно связаны вершины $v, x_1, x_2$, а по условию таких вершин только две.

	Так как отображение инъективно, степень $u$ не превышает степень $v$. Этот аргумент можно повторить, поменяв в рассуждениях $u$ и $v$ местами. Следовательно, степени $u$ и $v$ совпадают.


	\section*{Задача 7}

	Докажем от противного. Предположим, в графе нет ни одной вершины, соединенной со всеми остальными. Следовательно, в дополнении графа каждая вершина имеет степень хотя бы $1$.

	Рассмотрим произвольную компоненту связаности дополнения. Выберем произвольно какую-нибудь вершину $v$ из нее и обозначим за $U_d$ множество вершин этой компоненты, минимальное расстояние от которых до $v$ равно $d$.

	Обозначим $A = U_0 \cup U_2 \cup U_4 \cup \dots, B = U_1 \cup U_3 \cup U_5 \cup \dots$. Заметим, что для любой вершины $a \in A$ в графе-дополнении найдется ребро из $a$ в $B$, и наоборот, для любой вершины $b \in B$ в графе-дополнении найдется ребро из $b$ в $A$.

	В самом деле, если $b \in U_{2k+1}$, то путь длины $2k+1$ из $v$ в $b$ проходит через некоторую вершину $a \in U_{2k}$, и будет существовать ребро из $a$ в $b$.

	Аналогично, если $a \in U_{2k}$, то если $k > 0$, то путь длины $2k$ из $v$ в $a$ проходит через некоторую вершину $b \in U_{2k-1}$. Если же $k = 0$, то тогда $a = v$, а у $v$ обязательно есть инцидентное ребро (по предположению), и конец этого ребра будет лежать в $U_1$.

	Обозначим теперь за $E_i$ меньшее из множеств $A$ и $B$ в $i$-й компоненте. Аналогично, обозначим за $F_i$ большее из множеств $A$ и $B$ в $i$-й компоненте. Если $\abs{A} = \abs{B}$, пусть $E_i = A, F_i = B$.

	Обозначим $E = \cup E_i, F = \cup F_i$. Тогда $E \sqcup F = V, \abs{E} \le \abs{F}$, и, следовательно, $\abs{E} \le n$.

	Предположим для начала, что $\abs{E} = n$. Тогда к множеству $E$ применимо свойство из условия задачи, и найдется вершина $f \not\in E$, связанная со всеми вершинами из $E$. $f$, очевидно, лежит в некотором $F_i$. Но мы знаем, что в графе-дополнении $f$ связана с какой-то вершиной $e \in E_i$, то есть $f$ не может быть связана со всеми вершинами из $E$. Противоречие.

	Если $\abs{E} < n$, то перекинем $n - \abs{E}$ вершин из $F$ в $E$. Заметим, что конструкция из предыдущего абзаца не ломается. В самом деле, поскольку из $F$ мы вершины только убирали, $f$ все еще лежит в некотором $F_i$. Поскольку в $E$ вершины только добавлялись, то $e$ все еще будет лежать в соответствующем $E_i$. 
\end{document}
