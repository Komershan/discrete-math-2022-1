\subsection{Домашнее задание 6}


	\begin{center}
    \textbf{Задача 1}
\end{center}
		Воспользуемся формулой включений-исключений. Обозначим за $A_i$ множество строк, содержащих подстроку $1110$, начиная с $i$-й позиции ($0 \le i \le 12$).

		Строк, содержащих подстроку $1110$ на данной $i$-й позиции -- $2^{12}$, поэтому суммарная мощность множеств по всем $i$ -- $2^{12} \cdot 13$.

		Строк, содержащих подстроки $1110$ на $i$-й и $j$-й позициях ($i < j$) -- $2^8$, если $i + 4 \le j$ и $0$ иначе, итого суммарная мощность пересечений всех пар множеств:

		\begin{equation*}
			\sum_{i=0}^8 2^8 \cdot (9-i) = 2^8 \cdot 45.
		\end{equation*}

		Строк, содержащих подстроки $1110$ на $i$, $j$ и $k$-х позициях ($i < j < k$) -- $2^4$, если $i + 4 \le j, j + 4 \le k$, и $0$ иначе, итого суммарная мощность пересечений всех троек множеств:

		\begin{equation*}
			\sum_{i=0}^4 \sum_{j=i+4}^8 2^4 \cdot (9-j) = 2^4 \cdot \sum_{i=0}^4 \sum_{j=1}^{5-i} j = 2^4 \cdot \sum_{i=1}^5 \sum_{j=1}^i j = 2^4 \cdot \sum_{i=1}^5 \frac{i(i+1)}{2} = 2^4 \cdot (1 + 3 + 6 + 10 + 15) = 2^4 \cdot 35.
		\end{equation*}

		Строк, содержащих подстроки $1110$ на четырех позициях -- ровно одна: $1110111011101110$.

		Тогда по формуле включений-исключений ответ:

		\begin{equation*}
			2^{12} \cdot 13 - 2^8 \cdot 45 + 2^4 \cdot 35 - 1 = 42287.
		\end{equation*}


	\begin{center}
    \textbf{Задача 2}
\end{center}
		а) Построим биекцию между $A \setminus B$ и $A = (A \setminus B) \sqcup B$.

		$A \setminus B$ бесконечно, поэтому в нем можно выбрать некоторую счетную последовательность $a_1, a_2, \dots, a_n, \dots$. Обозначим множество ее значений $K$. $B$ само также счетно, поэтому занумеруем его элементы $b_1, b_2, \dots, b_n, \dots$.

		Пусть $f: (A \setminus B) \to A$ такова, что:

		\begin{align*}
			& f(a_{2i}) = a_i, \\
			& f(a_{2i-1}) = b_i, \\
			& f(x) = x \ (x \not\in K).
		\end{align*}

		Легко видеть, что это биекция: функция, суженная на первую строку этого определения, инъективна и имеет образ $K$, на вторую -- инъективна и имеет образ $B$, на третью -- инъективна и имеет образ $A \setminus B \setminus K$, а эти множества не пересекаются и в объединении дают все $A$.

		б) Нет, неверно: если $A = B = \N$, то $A \triangle B = \emptyset$, что неравномощно $\N$.


	\begin{center}
    \textbf{Задача 3}
\end{center}
		Многочленов с целыми коэффициентами степени $n$ -- $\mathbb{Z}^{n+1} \sim \mathbb{N}^{n+1} \sim \mathbb{N}$. У каждого из этих многочленов не более $n$ корней, поэтому всего корней таких многочленов не более $n \times \mathbb{N} \sim \mathbb{N}$.

		Соответственно, корней суммарно по всем степеням многочленов -- счетное объединение не более чем счетных множеств, то есть не более $\mathbb{N} \times \mathbb{N} \sim \mathbb{N}$.

		С другой стороны, алгебраических чисел как минимум счетно, потому что каждое целое число является алгебраическим.

		Следовательно, алгебраических чисел счетно.


	\begin{center}
    \textbf{Задача 4}
\end{center}
		Из бесконечного множества $A$ выпишем счетную последовательность $a_1, a_2, \dots, a_n, \dots$.

		Разделим все элементы этой последовательность на группы согласно степени вхождения простого $2$ в их номер. Тогда в $k$-ю группу ($k \ge 0, k \in \Z$) попадают элементы

		\begin{equation*}
			a_{2^k}, a_{3 \cdot 2^k}, a_{5 \cdot 2^k}, a_{7 \cdot 2^k}, \dots,
		\end{equation*}

		коих счетно, и всего групп счетно, что и требовалось.


	\begin{center}
    \textbf{Задача 5}
\end{center}
		В каждом отрезке можно выбрать рациональную точку. Отрезки не пересекаются, поэтому все выбранные рациональные точки будут различны. Всего рациональных чисел счетно, поэтому выбранных точек, а следовательно, и отрезков, не более, чем счетно.


	\begin{center}
    \textbf{Задача 6}
\end{center}
		Будем строить биекции итеративно: сначала установим им значения в точке $0$, затем $1, -1, 2, -2, 3, -3$, и так далее.

		Пусть к текущему моменту уже построены некоторые функции $g_1, g_2, g_3$, определенные на конечном числе точек, и мы хотим присвоить им некоторое значение в очередной точке $x$.

		Найдем ближайшее к нулю значение, отсутствующее в образе хотя бы одной из этих трех функций (образы будем обозначать как $g_i(\Z)$) на текущий момент. Ничью разбиваем, предпочитая положительные числа. Пусть это значение $y$, и без ограничения общности $y \not\in g_1(\Z)$. Тогда присвоим $g_1(x) = y$. Образы $g_2$ и $g_3$ конечны, поэтому найдется такое $n$, что $n \not\in g_2(\Z)$ и $f(x) - y - n \not\in g_3(\Z)$. Присвоим $g_2(x) = n, g_3(x) = f(x) - y - n$. Тогда $g_1(x) + g_2(x) + g_3(x) = f(x)$, что и требовалось.

		Строя $g_1, g_2, g_3$ таким образом итеративно, мы в итоге получим некоторые отображения, определенные на всем множестве целых чисел.

		Во-первых, эти отображения будут инъекциями, поскольку каждый раз мы присваиваем $g_1, g_2, g_3$ значения, отсутствовавшие до этого в их образах.

		Во-вторых, они будут сюръективны, поскольку мы на каждом шаге выбираем ближайшее к нулю не использованное число. Более формально, $g_1, g_2, g_3$ гарантированно принимают значение $0$ за первые три итерации, $1$ -- за следующие три итерации, $-1$ -- за следующие три и так далее; в общем случае каждая из трех функций гарантированное принимает $k$-е значение из последовательности $0, 1, -1, 2, -2, \dots$ после $3k$ итераций.


	\begin{center}
    \textbf{Задача 7}
\end{center}
		Нет, не обязательно. Приведем контрпример.

		Будем строить возрастающую последовательность $a_1, a_2, \dots, a_n, \dots$, множество значений которой будет $A$.

		Заметим, что если последовательность $\{a_n\}$ возрастает достаточно быстро, то она не будет содержать трехчленных арифметических прогрессий:

		\begin{equation*}
			\forall n \in N: \frac{a_{n+1}}{a_n} \ge 2 \implies \{a_n\} \text{ не содержит трехчленных арифметических прогрессий}.
		\end{equation*}

		В самом деле, если для некоторых $i < j < k$ мы имеем

		\begin{equation*}
			a_k - a_j = a_j - a_i,
		\end{equation*}

		то

		\begin{equation*}
			a_j - a_i < a_j \implies a_k - a_j < a_j \implies a_k < 2 a_j \implies \frac{a_k}{a_j} < 2,
		\end{equation*}

		но

		\begin{equation*}
			\frac{a_k}{a_j} = \frac{a_{j+1}}{a_j} \cdot \frac{a_{j+2}}{a_{j+1}} \cdot \dots \cdot \frac{a_k}{a_{k-1}} \ge 2^{k-j},
		\end{equation*}

		противоречие.

		Таким образом, достаточно построить достаточно быстро возврастающую последовательность $\{a_n\}$, такую, что дополнение ее образа не содержит бесконечной арифметической прогрессии, или, что то же самое, множество ее значений пересекается с каждой из бесконечных арифметических прогрессий. Для этого воспользуемся методом, похожим на диагональный аргумент Кантора.

		Каждая бесконечная арифметическая прогрессия задается двумя натуральными числами -- первым элементом и шагом, поэтому всего таких прогрессий $\N^2 \sim \N$. Занумеруем их $I_1, I_2, \dots, I_n, \dots$. Последовательность $\{a_n\}$ строим итеративно: первым элементом выбираем что угодно из $I_1$, а далее на $n$-м шаге в качестве $a_n$ выбирается произвольный элемент $I_n$, не меньший $2 a_{n-1}$ (такой найдется, так как $I_n$ неограничена).

		Тогда по построению $\frac{a_{n+1}}{a_n} \ge 2$ и $A$ пересекается с каждой прогрессией, чего мы и добивались.


