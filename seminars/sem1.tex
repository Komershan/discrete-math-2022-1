\documentclass{article}
\usepackage[utf8]{inputenc}
\usepackage[margin=1cm]{geometry}
\usepackage{fullpage}
\usepackage{enumitem}
\usepackage{amssymb}
\usepackage{amsmath}
\usepackage{cancel}
\usepackage{gensymb}
\usepackage{graphicx}
\usepackage{indentfirst}
\usepackage{listings}
\usepackage{algpseudocode}
\usepackage{amsmath}
\usepackage{float}
\usepackage{url}
\usepackage{tikz}
\usepackage{hyperref}
\usepackage{color}
\usepackage[russian]{babel}
\pagestyle{empty}

\newcommand{\shrug}[1][]{%
\begin{tikzpicture}[baseline,x=0.8\ht\strutbox,y=0.8\ht\strutbox,line width=0.125ex,#1]
\def\arm{(-2.5,0.95) to (-2,0.95) (-1.9,1) to (-1.5,0) (-1.35,0) to (-0.8,0)};
\draw \arm;
\draw[xscale=-1] \arm;
\def\headpart{(0.6,0) arc[start angle=-40, end angle=40,x radius=0.6,y radius=0.8]};
\draw \headpart;
\draw[xscale=-1] \headpart;
\def\eye{(-0.075,0.15) .. controls (0.02,0) .. (0.075,-0.15)};
\draw[shift={(-0.3,0.8)}] \eye;
\draw[shift={(0,0.85)}] \eye;
% draw mouth
\draw (-0.1,0.2) to [out=15,in=-100] (0.4,0.95); 
\end{tikzpicture}}
\providecommand{\openbox}{\leavevmode
  \hbox to.77778em{%
  \hfil\vrule
  \vbox to.675em{\hrule width.6em\vfil\hrule}%
  \vrule\hfil}}
\makeatletter
\DeclareRobustCommand{\qed}{%
  \ifmmode
    \eqno \def\@badmath{$$}%$$
    \let\eqno\relax \let\leqno\relax \let\veqno\relax
    \hbox{\openbox}%
  \else
    \leavevmode\unskip\penalty9999 \hbox{}\nobreak\hfill
    \quad\hbox{\openbox}%
  \fi
}

\begin{document}
Семинар 1
\begin{center}
\textbf{Задача 1}
\end{center}
а) 

База индукции: $\displaystyle n=1$: $\displaystyle 1+\frac{1}{2} \geqslant \frac{1}{2} +1$\qed 

Переход: $\displaystyle n\rightarrow n+1$: $\displaystyle 1+\frac{1}{2} +\frac{1}{3} +...+\frac{1}{2^{n}} \geqslant \frac{n}{2} +1$

$\displaystyle  \begin{array}{{>{\displaystyle}l}}
1+\frac{1}{2} +\frac{1}{3} +...+\frac{1}{2^{n}} +\frac{1}{2^{n} +1} +...+\frac{1}{2^{n+1}} \geqslant \frac{n}{2} +1+\left(\frac{1}{2^{n} +1} +...+\frac{1}{2^{n+1}}\right) \geqslant \\
\geqslant \frac{n}{2} +1+\left(\frac{1}{2^{n+1}} \cdotp 2^{n}\right) =\frac{n+1}{2} +1
\end{array}$\qed 

б) Докажем, что $\displaystyle \frac{1}{n+1} +\frac{1}{n+2} +...+\frac{1}{2n} \leqslant \frac{3}{4} -\frac{1}{2n}$

База: $\displaystyle n=1$: $\displaystyle \frac{1}{2} \leqslant \frac{3}{4} -\frac{1}{4} =\frac{1}{2}$\qed 

Переход $\displaystyle n\rightarrow n+1$: $\displaystyle  \begin{array}{{>{\displaystyle}l}}
\frac{1}{n+2} +...+\frac{1}{2n+2} =\frac{1}{n+1} +\frac{1}{n+2} +...+\frac{1}{2n} -\frac{1}{n+1} +\frac{1}{2n+1} +\frac{1}{2n+2} \leqslant \\
\leqslant \frac{3}{4} -\frac{1}{2n} -\frac{1}{n+1} +\frac{1}{2n+1} +\frac{1}{2n+2} =\frac{3}{4} -\frac{1}{2n} +\frac{1}{2n+1} -\frac{1}{2n+2} \leqslant \frac{3}{4} -\frac{1}{2n+2}
\end{array}$\qed 

\begin{center}
\textbf{Задача 2}
\end{center}
Докажем, что 1 можно представить в виде суммы любого числа различных обыкновенных дробей:

База: $\displaystyle n=1$: $\displaystyle 1=\frac{1}{1}$\qed 

Переход: $\displaystyle n\rightarrow n+1$: Пусть мы представили $\displaystyle 1$ в виде суммы $\displaystyle n$ дробей и дробь с минимальным значением равна $\displaystyle \frac{1}{n}$. Тогда возьмём от этой суммы все дроби кроме наименьшей, а её заменим на $\displaystyle \left(\frac{1}{n+1} +\frac{1}{n( n+1)}\right)$. Проверка: $\displaystyle \frac{1}{n+1} +\frac{1}{n( n+1)} =\frac{( n+1)^{2}}{n( n+1)^{2}} =\frac{1}{n}$\qed 

\begin{center}
\textbf{Задача 3}
\end{center}
Докажем что мы можем получить любую комбинацию из любой по индукции, не применяя операцию 2 типа если последовательность равна $\displaystyle 00...01$:

База: $\displaystyle n=1$. Тогда мы первой операцией можем изменить значение.\qed 

Переход: $\displaystyle n\rightarrow n+1$. Если последний символ начальной последовательности равен тому, каким он должен быть в итоговой комбинации, то просто сделаем префикс длины $\displaystyle n$ такой, каким он нам нужен (мы можем это сделать, так как по индукции мы умеем решать задачу для $\displaystyle n$ и мы запретили операцию, которая изменила бы последний символ).

Если же последний символ нужно изменить, то сделаем префикс длины $\displaystyle n$ равным $\displaystyle 0...001$, после применим 2 операцию, а после сделаем префикс длины $\displaystyle n$ таким, как надо.

Мы получили требуемую комбинацию и не сделали запрещённую операцию, поэтому переход доказан.\qed 

\begin{center}
\textbf{Задача 4}
\end{center}
Будем доказывать, что мы не только можем проехать $\displaystyle ��$ километров, но и можем сделать сколь угодно большой запас бензина в точке на расстоянии $\displaystyle ��$ километров от края пустыни, оказавшись в этой точке после окончания перевозок. 

База: $\displaystyle ��=1$: рейс на расстояние 1 и обратно требует 2 единиц бензина (будем называть единицей количество бензина на километр пути), поэтому мы можем оставить 1 единицу бензина в хранилище. За несколько рейсов в хранилище можно сделать запас произвольного размера, какого нам потребуется.

Переход: $\displaystyle n\rightarrow n+1$: Пусть мы хотим запасти в следующей точке $\displaystyle b$ единиц бензина. Тогда запасём в предыдущей точке $\displaystyle 3x$ единиц бензина. Тогда после этого сделаем $\displaystyle x$ рейсов: залить в бак 3 единицы, за 1 проехать в $\displaystyle n+1$, вылить там $\displaystyle 1$ и за $\displaystyle 1$ вернуться назад (в последнем рейсе это не надо делать). Мы доказали переход.\qed 

\begin{center}
\textbf{Задача 5}
\end{center}
Представим число $\displaystyle x=2^{s} \cdotp t$, где $\displaystyle t$ - нечётное. Тогда если представить так все числа от $\displaystyle 1$ до $\displaystyle 2n$, то значения $\displaystyle t$ будут равны множеству $\displaystyle \{1,\ 3,\ ...,\ 2n-1\}$ - $\displaystyle n$ элементов. Тогда если выбрать $\displaystyle n+1$ число, то у хотя бы двух будут равны $\displaystyle t$, значит они отличаются в $\displaystyle 2^{s_{2} -s_{1}}$ раз, значит одно делится на другое.\qed 

\begin{center}
\textbf{Задача 6}
\end{center}
Введём систему координат. Заметим, что пусть сейчас максимальная сумма координат клетки среди всех клеток равна $\displaystyle n$, тогда на следующем шагу она не может быть больше чем $\displaystyle n-1$ (все с суммой $\displaystyle n$ пропадут, так как для существования им нужны были клетки с суммой $\displaystyle n+1$, а клетки с суммой больше $\displaystyle n$ также появиться не могут).

А так как не может появиться клетки ниже чем самая низкая клетка изначально (так как ей для появления нужна либо она сама, либо клетка справа, но таких нет). Также не могло появиться клетки левее самой левой. Значит сумма координат ограничена снизу, значит все клетки пропадут за конечное число шагов.

\begin{center}
\textbf{Задача 7}
\end{center}

а) Частный случай пункта в)

б) Пусть мы всегда применяли операцию к самой левой паре $\displaystyle 01$. Мы получили какое-то число операций. Теперь докажем, что любая последовательность операций равносильна этой.

Пусть мы в какой-то момент применили операцию не к самой левой паре $\displaystyle 01$. Тогда пока мы не применим операцию к ней значения на этих позициях не изменятся, как как эти символы не входят ни в одну другую подстроку $\displaystyle 01$. Значит, в какой-то момент мы применим операцию к этой самой левой паре. Теперь если бы мы сделали операцию как только эта пара стала бы самой левой, то все будущие операции не изменились (только сдвинулись налево на 1), поэтому количество операций в любой последовательности будет равно последовательности из начала.

в) Заметим, что данная операция лексикографически увеличивает слово.

Нужно доказать, что слово нельзя увеличивать бесконечно. Отбросим префикс, который мы ни разу не поменяли (он не влияет на операции), количество единиц в строке могло уменьшиться. Рассмотрим операцию, которая изменила первый элемент строки: $\displaystyle 0$ заменился на $\displaystyle 1$, после этого эта $\displaystyle 1$ уже не может участвовать ни в одной операции. Значит после повторного отбрасывания префикса, который не будет изменяться, единиц гарантированно станет меньше.

Мы не можем увеличить количество единиц и гарантированно его уменьшаем, а значит в какой-то момент мы придём к строке из нулей, к которой нельзя применить операцию. Значит количество операций обязательно конечно.\qed 

\end{enumerate}
\end{document}


