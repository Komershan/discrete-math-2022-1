\subsection{Семинар 2}
\begin{center}
\textbf{Задача 1}
\end{center}
а) $\displaystyle ( A\cup B) \backslash ( A\cap B)$ - элементы, которые либо только в $\displaystyle A$, либо только в $\displaystyle B$.

$\displaystyle ( A\backslash B) \subseteq (( A\cup B) \backslash ( A\cap B)) \Longrightarrow ( A\backslash B) \cap (( A\cup B) \backslash ( A\cap B)) =( A\backslash B)$.

б) Если элемент в $\displaystyle A\backslash C$ или в $\displaystyle B\backslash С$, то он есть в $\displaystyle ( A\cap B) \backslash C$.

Если элемент в $\displaystyle ( A\cap B) \backslash C$, то он либо в $\displaystyle A$, либо в $\displaystyle B$, а так как его нет в $\displaystyle C$, то он либо в $\displaystyle A\backslash C$, либо в $\displaystyle B\backslash C$.

в) $\displaystyle ( A_{1} \times B_{1}) \cap ( A_{2} \times B_{2})$ $\displaystyle \Leftrightarrow $ только те пары, в которых первый элемент в $\displaystyle A_{1} \cap A_{2}$, а второй в $\displaystyle ( B_{1} \cap B_{2})$$\displaystyle \Leftrightarrow $$\displaystyle ( A_{1} \cap A_{2}) \times ( B_{1} \cap B_{2})$.

\begin{center}
\textbf{Задача 2}
\end{center}
Нужно доказать, что любой элемент из левого множества принадлежит правому.

Если $\displaystyle x\in ( A_{1} \cap A_{2} \cap ...\cap A_{n})$ и $\displaystyle x\not{\in }( B_{1} \cap B_{2} \cap ...\cap B_{n})$, то есть $\displaystyle B_{i}$, которому он не принадлежит, тогда $\displaystyle x\in A_{i} \vartriangle B_{i}$, значит элемент входит в объединение справа.

Аналогично если выполняется зеркальное условие $\displaystyle \vartriangle $.

\begin{center}
\textbf{Задача 3}
\end{center}
а) $\displaystyle y=x+2$

б) $\displaystyle y=( 1-( 1-x)) =x$

\textit{(нужна проверка)}

\begin{center}
\textbf{Задача 4}
\end{center}
Не верно: пусть множество элементов равно $\displaystyle \{1,2,3\}$, $\displaystyle A=\{( 1,1) ,\ ( 2,2) ,\ ( 3,3) ,\ ( 1,2) ,\ ( 2,1)\} ,$ $\displaystyle B=\{( 1,1) ,( 2,2) ,( 3,3) ,( 2,3) ,\ ( 3,2)\}$. Тогда композиция $\displaystyle A$ и $\displaystyle B$ не включает в себя $\displaystyle ( 3,1)$, но включает $\displaystyle ( 1,3)$, поэтому не выполняется симметричность.

\begin{center}
\textbf{Задача 5}
\end{center}
$\displaystyle f\left( f^{-1}( B)\right) \subseteq B$: пусть $\displaystyle x\in f\left( f^{-1}( B)\right) \Longrightarrow \exists y:\ f( y) =x\land y\in f^{-1}( B) \Longrightarrow $ $\displaystyle \exists x_{1} :f^{-1}( x_{1}) =y_{1} ,x_{1} \in B\Longrightarrow $$\displaystyle x=x_{1} \Longrightarrow x\in B$.

Нельзя поставить знак $\displaystyle =$, так как пусть $\displaystyle X=\{1\} ,\ Y=\{2,3\} ,\ f( 1) =2$. Тогда $\displaystyle f\left( f^{-1}(\{2,\ 3\})\right) =\{2\}$.

\begin{center}
\textbf{Задача 6}
\end{center}
а, б) $\displaystyle A=\{0\} ,\ B=\{1,\ 2\} ,\ f( 0) =1,\ g( 1) =g( 2) =0$. Тогда $\displaystyle ( g\circ f)( x) =x$, но $\displaystyle ( f\circ g)( 2) =1$, значит $\displaystyle g$ является левой обратной к $\displaystyle f$, но не является правой.

в) Пусть $\displaystyle l\circ f=id,\ f\circ r=id\Longrightarrow r=id\circ r=( l\circ f) \circ r=l\circ ( f\circ r) =l$ по теореме по об ассоциативности композиции.

г) \ todo

д) \ todo

\begin{center}
\textbf{Задача 7}
\end{center}
todo
\begin{center}
\textbf{Задача 9}
\end{center}
Нет на колке