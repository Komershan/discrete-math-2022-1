\subsection{Семинар 8}
\begin{center}
\textbf{Задача 1}
\end{center}
Нет, так как в первом у $\displaystyle 1$ есть предыдущий элемент, а во втором его нет.

\begin{center}
\textbf{Задача 2}
\end{center}
Перенесём оба отрезка в $\displaystyle [ 0,1]$ (параллельный перенос и домножение на константу), получим биекцию.

\begin{center}
\textbf{Задача 3}
\end{center}
$\displaystyle ( 0,1) \sim \mathbb{R} ,\ \mathbb{Q} \sim \mathbb{N} \Longrightarrow $ не изоморфны.

\begin{center}
\textbf{Задача 4}
\end{center}
Разобьём $\displaystyle ( 0,1)$ на $\displaystyle \left( 0,\frac{1}{2}\right] ,\left(\frac{1}{2} ,\frac{1}{4}\right] ,...$, а $\displaystyle \left( 0,\sqrt{2}\right)$ на $\displaystyle ( 0,1] \cup ( 1,1.4] \cup ( 1.4,1.141] \cup ...$ - десятичные приближения. Между полуинтервалами есть изоморфность, поэтому изоморфны и объединения.

\begin{center}
\textbf{Задача 5}
\end{center}
Сравниваем пары лексикографически. Пусть $\displaystyle ( 0,1)\rightarrow ( a,b)$. Тогда есть какое-то $\displaystyle ( c,d)\rightarrow ( a,b-1)$. Но тогда $\displaystyle ( c,d+1)$ должно переходить в пару между $\displaystyle ( a,b-1)$ и $\displaystyle ( a,b)$, что невозможно.

\begin{center}
\textbf{Задача 6}
\end{center}
 $\displaystyle 1 >01 >001 >0001 >...$ - не фундировано.

\begin{center}
\textbf{Задача 7}
\end{center}
Пусть смотрящие налево - 1, смотрящие направо - 0. Тогда пара $\displaystyle 10$ изменится на пару $\displaystyle 01$. То есть бинарная строка лексикографически уменьшится. А так как она конечна, то множество фундировано, то есть в какой-то момент изменения прекратятся.

\begin{center}
\textbf{Задача 8}
\end{center}
(Копия 1.7) Заметим, что данная операция лексикографически увеличивает слово.

Нужно доказать, что слово нельзя увеличивать бесконечно. Отбросим префикс, который мы ни разу не поменяли (он не влияет на операции), количество единиц в строке могло уменьшиться. Рассмотрим операцию, которая изменила первый элемент строки: $\displaystyle 0$ заменился на $\displaystyle 1$, после этого эта $\displaystyle 1$ уже не может участвовать ни в одной операции. Значит после повторного отбрасывания префикса, который не будет изменяться, единиц гарантированно станет меньше.

Мы не можем увеличить количество единиц и гарантированно его уменьшаем, а значит в какой-то момент мы придём к строке из нулей, к которой нельзя применить операцию. Значит количество операций обязательно конечно.

\begin{center}
\textbf{Задача 9}
\end{center}
todo

\begin{center}
\textbf{Задача 10}
\end{center}
todo

\begin{center}
\textbf{Задача 11}
\end{center}
Построим конструктивно биекцию. Пусть мы уже сделали отношения между $\displaystyle a\subseteq A,\ b\subseteq B$. Тогда выберем любое $\displaystyle x\in A\backslash a$. Оно может быть либо больше всех элементов $\displaystyle a$, либо меньше всех, либо между какими-то $\displaystyle a_{i}$ и $\displaystyle a_{i+1}$. Тогда найдём элемент в $\displaystyle B$, который в таком же отношении с элементами $\displaystyle b$ (такой элемент обязательно есть, так как нет граничных элементов и множество плотно). Сделаем между ними ребро биекции. Мы сделали переход $\displaystyle n\rightarrow n+1$.

Также чтобы каждый элемент гарантированно сопоставить пронумеруем $\displaystyle A,B$ и на чётные шаги будем сопоставлять минимальный неиспользованный элемент $\displaystyle A$, а на чётных - из $\displaystyle B$.