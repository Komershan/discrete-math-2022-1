\subsection{Семинар 10}
\begin{center}
\textbf{Задача 2}
\end{center}
Выберем 4 вершины степени $\displaystyle 4$. Из этого подграфа выходит минимум $\displaystyle 4$ ребра. Но осталась только вершина размера $\displaystyle 2$ - противоречие.

\begin{center}
\textbf{Задача 3}
\end{center}
а) Пусть все вершины степени $\displaystyle \leqslant 15$, тогда суммарно в графе не более $\displaystyle 15\cdotp 100=1500$ концов рёбер. Но в нём $\displaystyle 800$ рёбер, а значит $\displaystyle 1600$ концов. Противоречие.

б) Может, расставим вершины по кругу и соединим каждую со следующими $\displaystyle 8$ и предыдущими $\displaystyle 8$ по кругу.

\begin{center}
\textbf{Задача 4}
\end{center}
Пусть в графе $\displaystyle n$ вершин. Тогда в нём есть вершина размера $\displaystyle 0$ и размера $\displaystyle n-1$. Первая не соединена ни с какой вершиной, а вторая соединена со всеми. Противоречие.

\begin{center}
\textbf{Задача 5}
\end{center}
Пусть все вершины степени $\displaystyle \geqslant 2$, тогда концов рёбер $\displaystyle \geqslant 2n$, но в дереве их всего $\displaystyle 2n-2$ - противоречие.
\begin{center}
\textbf{Задача 6}
\end{center}
а) В дереве $\displaystyle 8$ рёбер $\displaystyle \Longrightarrow 16$ концов рёбер. Если есть $\displaystyle 2$ вершины степени 5, то пусть остальные вершины степени $\displaystyle 1$, тогда концов рёбер $\displaystyle 2\cdotp 5+1\cdotp 7=17 >16$. Противоречие.

б) В дереве $\displaystyle 2n-2$ конца. Пусть $\displaystyle x$ листьев, тогда концов рёбер не меньше $\displaystyle x+3\cdotp ( n-x) =3n-2x\leqslant 2n-2\Longrightarrow n+2\leqslant 2x\Longrightarrow x\geqslant \frac{n}{2} +1$.

\begin{center}
\textbf{Задача 7}
\end{center}
Пусть в графе не связаны вершины $\displaystyle A$ и $\displaystyle B$, а в дополнении - вершины $\displaystyle C$ и $\displaystyle D$. Тогда рёбра $\displaystyle AC$ и $\displaystyle BD$ могут либо оба существовать в графе, либо оба не существовать, пусть $\displaystyle AC$ существует. Тогда если $\displaystyle BC$ существует, то $\displaystyle A$ и $\displaystyle B$ соединены в графе, а если не существует, то $\displaystyle C$ и $\displaystyle D$ соединены в дополнении. Противоречие.

\begin{center}
\textbf{Задача 8*}
\end{center}
Можно разрезать на $\displaystyle 27$ кусов (разрезать на $\displaystyle 15$, поверх разрезать на $\displaystyle 13$, $\displaystyle gcd=1$, поэтому совпадение будет только одно).

Пусть можно разрезать на меньшее число кусков. Построим двудольный граф, в первой доле будет 13 людей, во второй будет 15 людей. Соединим ребром людей, которым достался один кусок и на ребре напишем, какую долю пирога оно представляет, если общий объём пирога равен $\displaystyle 15\cdotp 13$).

Рассмотрим какую-то компоненту связности этого графа. В нём сумма чисел на рёбрах должно делиться и на $\displaystyle 15$, и на $\displaystyle 13$ (так как каждому досталось только, сколько людей в другой группе), поэтому он равен $\displaystyle 15\cdotp 13$ и в графе одна компонента. А так как граф на $\displaystyle 15+13$ вершинах связен, то в нём не меньше $\displaystyle 15+13-1=27$ вершин.