\documentclass[a4paper]{article}
\usepackage[utf8]{inputenc}
\usepackage[russian]{babel}
\usepackage{amsfonts}
\usepackage{amssymb}
\usepackage{csquotes}
\usepackage{amsmath}
\usepackage{longtable}
\usepackage[unicode=true, colorlinks=true, linkcolor=blue, urlcolor=blue]{hyperref}
\usepackage[table,xcdraw]{xcolor}
\usepackage{graphicx}%Вставка картинок правильная
\usepackage{float}%Плавающие картинки
\usepackage{wrapfig}%Обтекание фигур (таблиц, картинок и прочего)

\newcommand{\F}{\mathbb{F}}
\renewcommand{\C}{\mathbb{C}}
\newcommand{\N} {\mathbb{N}}
\newcommand{\Z} {\mathbb{Z}}
\newcommand{\R} {\mathbb{R}}
\newcommand{\Q}{\mathbb{Q}}
\newcommand{\I}{\mathbb{I}}
\newcommand{\linf}[2]{\lim\limits_{#1 \to #2}}
\newcommand{\LLim}[2]{\mathop{\underline{\lim}}\limits_{#1 \rightarrow #2}}
\newcommand{\HLim}[2]{\mathop{\overline{\lim}}\limits_{#1 \rightarrow #2}}
\newcommand{\supremum}[2]{\sup\limits_{#1 \in #2}}
\newcommand{\infinum}[2]{\inf\limits_{#1 \in #2}}
\newcommand{\Sum}[2]{\sum\limits_{#1}^{#2}}
\newcommand{\task}[1]{\section*{Задание #1}}
\newcommand{\subtask}[1]{\subsection*{Пункт #1}}
\renewcommand{\epsilon}{\varepsilon}

\setlength{\parindent}{0pt}
\setlength{\parskip}{5pt}
\widowpenalties 1 10000
\raggedbottom

\begin{document}

\task{1}

$\chi_{C_n}(x)$ - количество способов правильно раскрасить $C_n$ в $x$ цветов.

Если в $G$ удалить ребро, то $\chi_{G - uv} (k)= \chi_{G}(k) + \chi_{G \cdot uv}(k)$. Когда мы сжимаем ребро, получаем цикл на один меньше, когда
убираем ребро, получаем дерево, тогда для дерева хроматический многочлен равен $\chi_T(x) = x(x - 1)^{n - 1}$, тогда получим рекуренту.

$$\chi_{C_n} (x) = x(x - 1)^{n - 1} - \chi_{C_{n - 1}} = x(x - 1)^{n - 1} - x(x - 1)^{n - 2} + ... + (-1)^n \cdot \chi_{P_2}(x)$$

$$\chi_{C_n} (x) = x \cdot \Sum{k = 1}{n - 1} (-1)^{n - k - 1}(x - 1)^k = (-1)^{n}x(x - 1) \cdot \Sum{k = 0}{n - 2} (x - 1)^k (-1)^k$$

$$\chi_{C_n} (x) = \frac{(x - 1)^{n - 1} - 1}{1 - x - 1} \cdot x (x - 1) (-1)^{n} = (1 - (1 - x)^{n - 1})(x - 1)(- 1)^{n}$$

$$\chi_{C_n} (x) = (x - 1)^n + (-1)^n(x - 1)$$

\task{2}

$$x(x - 1)^{n - 1} = x^n - (x - 1)\cdot x^{n - 1} + ... + (-1)^{n - 1}x$$

Значит у нас $n - 1$ ребро и 1 компонента связности, тогда этот гра дерево, доказано.

\task{3}

$f$ - число элементов в $S_p$. Пусть $g_i = \begin{cases}
    1, i \in S_p\\
    0, else
\end{cases}$

$$f = \Sum{i = 1}{n} = g_i$$

$$Ef = \Sum{i = 1}{n} = g_i = n \cdot p$$

\task{4}

$a$ - строка, $a \in B^n = \{0, 1\}^n$, C - наше подмножество

каждая строка с вероятностью $\frac{1}{2}$ входит в случайное подмножество.

Суммарное число единиц во всех строках равно $\frac{2^n}{2} n$, так как каждую строку можно сопоставить парой с обратной ей, у них в сумме $n$ единиц,
таких пар всего $\frac{2^n}{2}$. Пусть $\#a$ - количество единиц

$$Ef = \Sum{a \in B}{} P(a \in С) \cdot \#a = \Sum{a \in B}{} \frac{1}{2} \#a = \frac{1}{2} \cdot \frac{2^n}{2} n = 2^{n - 2} \cdot n$$

Пункт б

Выбрать упорядоченно различных $k$ строк

$$f = \Sum{i = 1...k, j = 1...n}{} g_{ij}, Eg_{ij} = P(\text{в ячейке (i, j) стоит 1}) = \frac{1}{2}$$

$$Ef = \frac{kn}{2}$$

\task{5}

\task{6}

\task{7}

Воспользуемся неравенством Чебышёва.

Пусть $f$ - количество единиц в строке, $Ef = \frac{n}{2}$

$$P(|f - \frac{n}{2} \geq \sqrt{n}|) \leq \frac{Df}{n}$$

$$Df = E(f^2) - (Ef)^2 = E(f^2) - \frac{n^2}{4}$$

$$E(f^2) = (g_1 + ... + g_n)^2 = \Sum{i = 1}{n} g_i^2 + 2 \Sum{i < j}{} g_i \cdot g_j = f + 2 \Sum{i < j}{} g_i g_j =$$

$$Ef + 2 \Sum{i < j}{} E g_i g_j = \frac{n}{2} + 2 \cdot \frac{n(n - 1)}{2} \cdot \frac{1}{4} = \frac{n}{2} + \frac{n(n - 1)}{4}$$

$$Df = \frac{n}{2} + \frac{n(n - 1)}{4} - \frac{n^2}{4} = \frac{n}{4}$$

$$P(|f - \frac{n}{2} \geq \sqrt{n}|) \leq \frac{\frac{n}{4}}{n} = \frac{1}{4}$$

\end{document}