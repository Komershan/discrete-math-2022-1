\subsection{Семинар 6}
\begin{center}
\textbf{Задача 1}
\end{center}
$\displaystyle 15\cdotp C_{4}^{1} -6\cdotp C_{4}^{2} +2\cdotp C_{4}^{3} -1=31$
\begin{center}


\textbf{Задача 2}
\end{center}
Их не более чем счётно, так как они - подмножество натуральных. Далее пусть их конечно. Тогда перемножим их все и прибавим 1. Новое число не будет делиться ни на одно известное простое, а значит это были не все простые числа. Противоречие. Значит их ровно счётно.

\begin{center}
\textbf{Задача 3}
\end{center}
Докажем, что количество подмножеств фиксированного размера счётно. А так как таких множеств счётно, то получится что всего различных множеств счётно.

Пусть мы зафиксировали количество элементов $\displaystyle c$ и пронумеровали $\displaystyle \mathbb{Q}$. Тогда выберем индекс максимального элемента, который будет в множестве, пусть это $\displaystyle d$. Тогда различных множеств $\displaystyle 2^{d}$ - конечно. А так как вариантов $\displaystyle d$ счётно, то всего количество множеств будет счётным. Также оно не может быть конечным, так как количество множеств из одного элемента счётно.

\begin{center}
\textbf{Задача 4}
\end{center}
а) Представим каждую цифру как двоичный код из 2 символов.

б) $\displaystyle 0\leftrightarrow 0,1\leftrightarrow 10,2\leftrightarrow 11$. Из бинарной построить двоичную - очевидно, любая двоичная однозначно построит бинарную так как никакая не является префиксом другой.

\begin{center}
\textbf{Задача 5}
\end{center}
Количество подмножеств $\displaystyle B$ размера $\displaystyle |A|$ счётно (зафиксируем максимальный индекс, для него множеств конечно, индексов счётно). А функция - это перестановка длины $\displaystyle |A|$. А так как перестановок каждого множества конечно, то всего получается счётное число функций.

\begin{center}
\textbf{Задача 6}
\end{center}
Зафиксируем $\displaystyle T$ (их счётно), теперь нам нужно выбрать $\displaystyle T$ значений для одного периода функции. Их конечно, так как выбрать подмножеств фиксированного размера у счётного множества счётно (зафиксируем максимальный и минимальный элемент, таких множеств конечно).

\begin{center}
\textbf{Задача 7}
\end{center}
Выберем подпоследовательность $\displaystyle \left[\frac{1}{2} ,\frac{1}{4} ,\frac{1}{8} ,...\right]$. Тогда $\displaystyle 0\rightarrow \frac{1}{2} ,1\rightarrow \frac{1}{4} ,$если число входит в последовательность, то берём элемент на 2 индекса больше, остальные переходят в себя. Это биекция, так как из каждой и в каждую входят рёбра.

\begin{center}
\textbf{Задача 8}
\end{center}
Верно. Выберем в $\displaystyle A$ счётное множество. После этого сделаем последовательность $\displaystyle C=[ A_{1} ,B_{1} ,A_{2} ,B_{2} ,...]$. Теперь в биекции $\displaystyle A_{i} \leftrightarrow C_{i}$, остальные элементы переходят сами в себя. Значит $\displaystyle |A\cup B|=|A|$, так как есть биекция.

\begin{center}
\textbf{Задача 9}
\end{center}
Введём полярные координаты с центром в центре окружности, тогда вектора домножаются на $\displaystyle R$ (при переходе обратно делятся на $\displaystyle R$).

\begin{center}
\textbf{Задача 10}
\end{center}
Проведём диаметр $\displaystyle AB$ и касательную $\displaystyle l$ через точку $\displaystyle B$, перпендикулярную $\displaystyle AB$. Тогда для любой точки окружности $\displaystyle X$ кроме $\displaystyle A$ точкой на прямой будет $\displaystyle AX\cap l$. Осталась одна точка $\displaystyle A$, чтобы её добавить выберем последовательность и сделаем её первой (то есть биекция отображает в следующий элемент последовательности, см предыдущие задачи).

\begin{center}
\textbf{Задача 11} 
\end{center}
Сначала сделаем очевидные отображения $\displaystyle ( 0,1)\rightarrow ( 0,2)\rightarrow ( -1,1)$. Теперь для $\displaystyle ( -1,0)$ сделаем $\displaystyle f( x) =\frac{x}{1+x}$ (перейдёт в $\displaystyle ( -\infty ,0)$), для $\displaystyle ( 0,1)$ сделаем $\displaystyle f( x) =\frac{x}{1-x}$ (перейдёт в $\displaystyle ( 0,+\infty )$), и $\displaystyle 0\rightarrow 0$. Получилась биекция.

\begin{center}
\textbf{Задача 12}
\end{center}
В прошлом номере мы сделали биекцию между $\displaystyle ( 0,+\infty )$ и $\displaystyle ( 0,1)$. Применим её и задача свёдется просто к прошлому номеру.

\begin{center}
\textbf{Задача 14}
\end{center}
Выберем из интервала последовательность, прочередуем её с $\displaystyle \mathbb{Z}$, сделаем биекцию.