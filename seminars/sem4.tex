\subsection{Семинар 4}
\begin{center}
\textbf{Задача 1}
\end{center}
Идея: выбрать позиции для О, потом расставить остальные буквы (некоторые другие буквы тоже повторяются, нужно разделить на факториал количества их вхождений)

\begin{center}
\textbf{Задача 2}
\end{center}
Выражение равносильно $\displaystyle a\lor b\lor c$ (выводится перебором количества единичных переменных).

\begin{center}
\textbf{Задача 3}
\end{center}
а) $\displaystyle ( x_{1} \land x_{2}) \oplus ( x_{1} \land x_{3}) \oplus ( x_{2} \land x_{3})$ (ясно что нужны такие переменные, после проверка показывает что такое выражение подходит)

б) $\displaystyle ( x_{1} \land x_{2}) \oplus ( x_{1} \land x_{3}) \oplus ( x_{1} \land x_{4}) \oplus ( x_{2} \land x_{3}) \oplus ( x_{2} \land x_{4}) \oplus ( x_{3} \land x_{4}) \oplus ( x_{1} \land x_{2} \land x_{3} \land x_{4})$ - если две единицы, то работает, если 3 то 3 выражения 1, значит работает. Если все 4, то получается что нужно было добавить И всех аргументов.

\begin{center}
\textbf{Задача 4}
\end{center}
а) Верно когда нет выражений $\displaystyle 1\rightarrow 0$, значит либо все $\displaystyle x=0$, тогда в $\displaystyle y$ что угодно, либо в $\displaystyle x$ есть 1 и тогда все игреки $\displaystyle 1$. Получается количество способов $\displaystyle 2^{5} +2^{5} -1=63$.

б) todo

в) Если все переменные 1, то в многочлене Жегалкина все мономы равны $\displaystyle 1$, поэтому чтобы выражение было равно 1 нужно чтобы мономов было нечётно.

\begin{center}
\textbf{Задача 5}
\end{center}
Если выражение - тождественная единица, то $\displaystyle a\lor \neg a$ (нужно узнать можно ли так). Иначе построим СДНФ для выражения $\displaystyle \neg f$, а после по закону Де-Мограна мы получим КНФ для $\displaystyle f$.

\begin{center}
\textbf{Задача 6}
\end{center}
Мы знаем, что связка $\displaystyle \{\neg ,\land \}$ полная. Отрицание - $\displaystyle X\ |\ X$, И - $\displaystyle ( X\ |\ Y) \ |\ ( X\ |\ Y)$.

\begin{center}
\textbf{Задача 7}
\end{center}
а) Нет, так как на наборе из всех 0 нельзя получить 1

б) Нет, так как на наборе из 0 и на наборе из 1 обязательно будут одинаковые результаты.

\begin{center}
\textbf{Задача 8}
\end{center}
Нет, так как обе функции в ней самодвойственны, то есть на наборе из всех 0 и на наборе из всех 1 они не могут дать одинаковые значения. Доказательство по рекурсии, в выражении всё до текущего момента изменилось, значит изменится и результат текущей операции. (критерий Поста явно не используется)

