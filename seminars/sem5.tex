\subsection{Семинар 5}
\begin{center}
\textbf{Задача 1}
\end{center}
а) Верно, так как если $\displaystyle f\in P\cap Q\Longrightarrow f\in P\land f\in Q\Longrightarrow [ f] \in P,\ [ f] \in Q\Longrightarrow [ f] \in P\cap Q$ (я очень в этом не уверен)

б) Не верно. Пример: $\displaystyle S\cup M$. В нём содержатся $\displaystyle \land $и $\displaystyle \neg $, значит замыкание равно множеству всех функций, что не равно $\displaystyle S\cup M$.

\begin{center}
\textbf{Задача 2}
\end{center}
а) $\displaystyle 2^{2^{n} -1}$ (для всех масок кроме нулевой любые значения)

б) Так как линейная функция представляется в виде полиномо Жегалкина где есть только свободный моном и мономы с одной переменной, то есть всего $\displaystyle 2^{n-1}$ вариант коэффициентов, а значит есть ровно столько линейных функций.

в) $\displaystyle 2^{2^{n} /2} =2^{2^{n-1}}$, так как каждое значение задаёт значение и для обратной маски

\begin{center}
\textbf{Задача 3}
\end{center}
Нет, так как $\displaystyle \land \in M,\lor \in M,\ XOR_{3}\not{\in } M$.

\begin{center}
\textbf{Задача 4}
\end{center}
Выберем все маски, в которых $\displaystyle \lfloor \frac{n}{2} \rfloor $ переменных. Сделаем значения во всех масках с меньшим количеством нулей 0, а с большим - 1. Тогда мы можем сделать любые значения в масках с ровно таким количеством нулей и функция будет монотонной. Значит, таких функций хотя бы $\displaystyle 2^{\lfloor \frac{n}{2} \rfloor }$.

\begin{center}
\textbf{Задача 5}
\end{center}
а) Является, так как $\displaystyle \rightarrow $ принадлежит только $\displaystyle T_{1}$, а $\displaystyle diff$ этому классу не принадлежит

б) Является, так как \ $\displaystyle 0\not{\in } T_{1} ,\ 1\not{\in } T_{0} ,\ \land \not{\in } S,L,\ evn\ \not{\in } M$

в) Является, так как для каждого класса есть функция, которая ему не принадлежит (с 1 взгляда это так, но легче это проверить на месте)

\begin{center}
\textbf{Задача 6}
\end{center}
todo
\begin{center}
\textbf{Задача 7}
\end{center}
todo
\begin{center}
\textbf{Задача 8}
\end{center}
Я не понимаю пункт В, так что скорее всего моя логика неправильна и описаться на неё нельзя. Но вот что пока написано:

а) Обе функции сохряняют 0, кроме этого $\displaystyle \oplus $только линейна, а $\displaystyle \land $ не линейна, поэтому их замыкание равно $\displaystyle T_{0}$.

б) $\displaystyle \equiv $сохраняет $\displaystyle 1$ и линейна, $\displaystyle \land $ сохраняет $\displaystyle 1$ и не линейна, значит замыкание - $\displaystyle T_{1}$.

в) $\displaystyle \{0,1,\land ,\lor \}$. Все монотонны, $\displaystyle \land $



