\documentclass[a4paper]{article}
\usepackage[utf8]{inputenc}
\usepackage[russian]{babel}
\usepackage{amsfonts}
\usepackage{amssymb}
\usepackage{csquotes}
\usepackage{amsmath}
\usepackage{longtable}
\usepackage[unicode=true, colorlinks=true, linkcolor=blue, urlcolor=blue]{hyperref}
\usepackage[table,xcdraw]{xcolor}
\usepackage{graphicx}%Вставка картинок правильная
\usepackage{float}%Плавающие картинки
\usepackage{wrapfig}%Обтекание фигур (таблиц, картинок и прочего)

\newcommand{\F}{\mathbb{F}}
\renewcommand{\C}{\mathbb{C}}
\newcommand{\N} {\mathbb{N}}
\newcommand{\Z} {\mathbb{Z}}
\newcommand{\R} {\mathbb{R}}
\newcommand{\Q}{\mathbb{Q}}
\newcommand{\I}{\mathbb{I}}
\newcommand{\linf}[2]{\lim\limits_{#1 \to #2}}
\newcommand{\LLim}[2]{\mathop{\underline{\lim}}\limits_{#1 \rightarrow #2}}
\newcommand{\HLim}[2]{\mathop{\overline{\lim}}\limits_{#1 \rightarrow #2}}
\newcommand{\supremum}[2]{\sup\limits_{#1 \in #2}}
\newcommand{\infinum}[2]{\inf\limits_{#1 \in #2}}
\newcommand{\Sum}[2]{\sum\limits_{#1}^{#2}}
\newcommand{\task}[1]{\section*{Задание #1}}
\newcommand{\subtask}[1]{\subsection*{Пункт #1}}
\renewcommand{\epsilon}{\varepsilon}

\setlength{\parindent}{0pt}
\setlength{\parskip}{5pt}
\widowpenalties 1 10000
\raggedbottom

\title{discrete-math-2022-1}
\author{Цыганов Аскар БПМИ221}
\date{13 Декабрь, 2022}

\begin{document}
\maketitle

%% тут текст

\task{1}

\begin{enumerate}
    \item $P(a \mid B) > P(A), A = B$
    \item $P(a \mid B) < P(A), P(A \cap B) = 0$
    \item $P(a \mid B) = P(A), A $ и $ B$ - независимые
\end{enumerate}

\task{2}

Построим дерево решений.

$P(\text{2-я монета золотая} \mid \text{1 - золотая}) = P(\text{Обе золотые}) / P(\text{взятая - золотая}) = \frac{1/3}{3/6} = \frac{2}{3}$

\task{3}

Тут меня вызвали к доске % TODO

\task{4}

Выгодно менять выбор, можно просто в тупую посчитать вероятность.

\task{5}

A - картное 2, B - кратное 3

\begin{equation*}
    P(A) = \frac{1}{2}, P(B) = \frac{1}{3}, P(A \cap B) = \frac{1}{6} = P(A) \cdot P(B)
\end{equation*}

\task{6}

1) В силу симметри, можно построить биекцию просто поменять местами два элемента
$$P(x_{24} > x_{25}) = \frac{1}{2}, P(x_{25} > x_{26}) = \frac{1}{2}$$

Зафиксируем все, кроме 24, 25 и 26 элементов. Нам будет подходить только 1 случай из 6.

$$P(x_{24} > x_{25} > x_{26}) = \frac{1}{6} \neq P(x_{24} > x_{25}) \cdot P(x_{25} > x_{26})$$

2) Пусть A - $x_{24}$ больше всех последующих, B - $x_{25}$ больше всех последующих.

Посчитаем $P(A)$ забьем на первые 23 элемента. Всего расставить в тех 23 расставить числа
можно $\binom{49}{23} \cdot 23! = \frac{49!}{26!}$ способами. Выберем максимальный и расставим остальные 25! способами.

Тогда $P(A) = \frac{1}{26}$, аналогично $P(B) = \frac{1}{25}$.

Но $P(A \cap B)$, аналогичными рассуждениями забьем на первые 23 их расставить можно
$\frac{49!}{26!}$ способами, выбираем два наибольших и расставляем остальные $24!$ способами.

Значит $P(A \cap B) = \frac{49! \cdot 24!}{26! \cdot 26!} = \frac{1}{25 \cdot 26} = P(A) \cdot P(B)$,
тогда $A, B$ - независимые.

\task{7}

Пусть A - первое событие, а B - второе.

$$P(A) = \frac{\binom{10}{5} \cdot 5!}{10^5} = \frac{10 \cdot 9 \cdot 8 \cdot 7 \cdot 6}{10^5} > \frac{1}{2^5}$$

Из-за независимости:

$$P(B) = \frac{1}{2^5}$$

Значит первое вероятнее

\task{8}

\begin{equation*}
    P(A \mid B) = 1 - P(\hat{A} \mid B)
\end{equation*}

\begin{equation*}
    P(A \mid \hat{B}) = \frac{P(\hat{B} \mid A) \cdot P(A)}{P(\hat{B})} = \frac{(1 - P(B \mid A)) \cdot P(A)}{1 - P(B)}
\end{equation*}

Мы выразили $P(A \mid \hat{B})$, остальные аналогично выражаем

\task{9}

Если $A_1, A_2, ..., A_n$ - независимы в совокупности, то $\hat{A_1}, A_2, ..., A_n$ - тоже
независимы в совокупности. Возьмем $P(\hat{A_1} \cap A_{i_1} \cap ... \cap A_{i_k})$.

Пусть $B = A_{i_1} \cap ... \cap A_{i_k}$.

Докажем, что $P(\hat{A_1} \cap B) = P(\hat{A_1}) \cdot P(B) = (1 - P(A_1)) \cdot P(B)$

$$P(\hat{A_1} \cap B) = P(B) - P(A_1 \cap B) = P(B) - P(A_1) \cdot P(B) = (1 - P(A_1)) \cdot P(B)$$

Чтд


%%%
\end{document}