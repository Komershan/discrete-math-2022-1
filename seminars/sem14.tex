\subsection{Семинар 13}
\textbf{Задача 1}\\

$\Omega = \{1, 2,  3, 4, 5, 6\}$\\

$\frac{1}{36} = P(x), \forall x \in \Omega$\\

\begin{equation*}
\begin{array}{c}
1 + 4\\
2+ 3\\
3 + 2\\
4 + 1\\    
\end{array}
\end{equation*}

$P(A) = \frac{1}{9}$\\

\textbf{Задача 2}\\

Монетка выпадает решкой с вероятностью $p$. Пусть $\Omega = \{0, 1\}, P(1) = p, P(0) = 1 -p$ и $\omega = \Omega^6$.
Тогда если $n$ - кол-во решек, то $P(x_1x_2 ... x_6) = p^n(1 - p)^{6 - n}$. Проверим, что в такой модели сумма вероятностей равна 1.

\begin{equation*}
    \sum\limits_{n = 0}^{6} \binom{6}{n} p^n(1 - p)^{6 - n} = (p + (1 - p))^6 = 1^6 = 1
\end{equation*}

Теперь посчитаем вероятность, что выпадет ровно 3 решки. Всего таких исходов $\binom{6}{3}$ и все исходы равновероятны. Тогда
ответом будет $\binom{6}{3} p^3(1 - p)^{3}$. Подставим $p$ и получим $\frac{5}{16}$.\\

\textbf{Задача 3}\\

$\Omega = \{1, ..., 6\}^3, \forall x \in \Omega \ \ P(x) = \frac{1}{6}$. Введем события $A = \{(x, y, z) \mid (x + y + z) \vdots 2 \}$,
$B = \{(x, y, z) \mid (x + y + z) \vdots 3 \}$. По формуле включений-исключений:

\begin{equation*}
    P(A \cup B) = P(A) + P(B) - P(A \cap B) = \frac{1}{2} + \frac{1}{3} - \frac{1}{6} = \frac{2}{3}
\end{equation*}

Докажем, что $P(A) = \frac{1}{2}$. Количество исходов $(x, y, z)$ с четной и нечетной суммой одинаково. Зафиксируем два броска, равновероятно их сумма будет четна или нечетна. Тогда мы равновероятно добавим к ним четное
или нечетное. Тогда действительно четность равновероятна. Аналогично можно доказать, что $P(B) = \frac{1}{3}$.

И последнее найдем $P(A \cap B)$. Снова аналогичными утверждениями(зафиксировав два числа нам подойдет только одно третье число), значит $P(A \cap B) = \frac{1}{6}$.\\

\textbf{Задача 4}\\

\textbf{Пункт а}\\

Колода из 36 карт - перестановка длины 36. Пусть 1-9 - первая масть, ..., 28 - 36 - четвертая масть. Данное пространство равновероятно, поэтому
$P(g) = \frac{1}{36!}$. Будем считать, что первому достались числа на позициях 1-9 и аналогично для остальных. Можно распределить масти по людям
$4!$ способами, внутри масти их можно перемещать $9!$ способами и так в каждом. Значит $P(A) = \frac{4! \cdot (9!)^4}{36!}$.\\

\textbf{Пункт б}\\

Возьмем неупорядоченное вероятностное пространство. $P(g) = \frac{(9!)^4}{36!}$. Всего есть $4!$ способов поставить людям королей. Мы уже расположили королей
и осталось раздать по 8 карт, всего способов так сделать $\frac{32!}{(8!)^4}$. В итоге получится

\begin{equation*}
    \frac{4! \frac{32!}{(8!)^4}}{ \frac{(9!)^4}{36!}} = \frac{4! (9!)^4 32!}{36!(8!)^4} = \frac{4! 9^4}{36 \cdot 35 \cdot 34 \cdot 33}
\end{equation*}

\textbf{Задача 5}\\

\textbf{Пункт а}\\

$\Omega = \{1, 2, ..., 10\}, P(\omega) = \frac{1}{10}$. Очевидно, тогда ответ $\frac{1}{10}$.\\

\textbf{Пункт б}\\

$\Omega = S_{10}$. Построим дерево, на первом слое будет вероятность $\frac{1}{10}$, на втором слое будет $\frac{1}{9}$ и тд. Заметим, что на каждом слое
сумма будет 1. В листах будет задана какая-то перестановка билетов. Листов будет в точности $10!$, тогда $P(\omega) = \frac{1}{10!}$. Всего благоприятных исходов
с выученным билетом в конце очевидно будет $9!$. Тогда $P(A) = \frac{9!}{10!} = \frac{1}{10}$.\\

\textbf{Задача 6}\\

$\Omega = \{(1, i_1, ..., i_k) \mid 1 < i_1 < ... < i_k, k \geq 0\}$

\begin{equation*}
    \begin{array}{ccccccc}
        1 & \overbrace{2 ...}^{id} & i_1 &\overbrace{...}^{id} & i_2 &\overbrace{...}^{id}&i_k\\
        \downarrow  & \downarrow &\downarrow & \downarrow & \downarrow& \downarrow & \downarrow\\
        i_1 & 2   ... & i_2 & ... & i_3 & ... & 1\\
    \end{array}
\end{equation*}

Получили цикл $(1, i_1, i_2, ..., i_k)$. $P((1, i_1, i_2, ..., i_k)) = \frac{1}{100} \cdot \frac{1}{101 - i_1} \cdot \frac{1}{101 - i_2} \cdot \dots \cdot \frac{1}{101 - i_k}$.

Теперь докажем, что сумма вероятностей равна 1.

\begin{equation*}
    \frac{s}{100} = \sum\limits_{k = 0}^{99} \sum\limits_{1 < i_1 <...<i_k} \frac{1}{100} \cdot \frac{1}{101 - i_1} \cdot \frac{1}{101 - i_2} \cdot \dots \cdot \frac{1}{101 - i_k} = 1
\end{equation*}

\begin{equation*}
    s = \sum\limits_{a_1 < ... < a_k < 100} \frac{1}{a_1} \cdot ... \cdot \frac{1}{a_k}
\end{equation*}

\begin{equation*}
    s = (1 + \frac{1}{1}) (1 + \frac{1}{2})...(1 + \frac{1}{99}) = 2 \cdot \frac{3}{2} \cdot \frac{4}{3} \cdot ... \cdot \frac{100}{99} = 100
\end{equation*}

$A = \{(1, i_1, i_2, ..., i_k) \mid i_k \neq 100, k \geq 100\} \subseteq \Omega$. Построим биекцию из плохих исходов в хорошие. Уберем последний элемент 100 и теперь
$i_k = 100$, там можно поставить либо 0, либо 1, значит плохие исходы и хорошие равновероятны. Значит $P(A) = 1 - P(A) \Rightarrow P(A) = \frac{1}{2}$.\\

\textbf{Задача 8}\\

Пусть $a_n$ - за $n$ ходов, не запачкавшись вернулись в $A$. Аналогично введем $b_n, c_n, f_n = b_n, e_n = c_n$.
$b_n = \frac{a_{n -1} + c_{n - 1}}{2}, c_n = \frac{b_{n - 1}}{2}, a_n = b_n - 1 \Rightarrow$\\

$b_n = \frac{3b_{n - 2}}{4}$, но $b_1 = \frac{1}{2}, b_2 = 0$, тогда.

\begin{equation*}
    b_n =
    \begin{cases}
        \left(\frac{3}{4}\right)^{\frac{n - 1}{2}} \cdot \frac{1}{2}, b - \text{нечетно}\\
        0, \text{иначе}
    \end{cases}
\end{equation*}

Далее легко посчитать