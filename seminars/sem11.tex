\subsection{Семинар 11}
\begin{center}
\textbf{Задача 1}
\end{center}
Построим граф цифр, которые могут следовать друг за другом

$\displaystyle 1\rightarrow 4,\ 4\rightarrow 2,\ 4\rightarrow 9,\ 9\rightarrow 2,\ 9\rightarrow 8,\ 8\rightarrow 4$

$\displaystyle 3\rightarrow 5,5\rightarrow 6,6\rightarrow 3$

$\displaystyle 7\rightarrow 0,\ 7\rightarrow 7$

Жадно набираем ответ, получаем $\displaystyle 98431$.

\begin{center}
\textbf{Задача 2}
\end{center}
Нет, пример: $\displaystyle 1\rightarrow 2,\ 2\rightarrow 1,\ 1\rightarrow 3,\ 3\rightarrow 1$. Везде только $\displaystyle 1$ простой путь, а исходящая степень вершины $\displaystyle 1$ равна $\displaystyle 2$.

\begin{center}
\textbf{Задача 3}
\end{center}
$\displaystyle k\cdotp ( k-1)^{n-1}$, раскрашиваем одну вершину, потом у каждой есть 1 заблокированный вариант.

\begin{center}
\textbf{Задача 4}
\end{center}
а) Если добавленные рёбра соединяют вершины одной чётности то раскраски нет, значит она есть только при $\displaystyle n\underset{mod\ 2}{=} 1$.

б) При нечётных $\displaystyle n$ граф $\displaystyle 2-$раскрашиваем. Рассмотрим чётные. При $\displaystyle n=2$ получается полный граф на $\displaystyle 4$ вершинах, в нём нет $\displaystyle 3-$ раскраски. А для $\displaystyle n >2$ сделаем так: сначала раскрасим $\displaystyle n$ вершин цикла в $\displaystyle 1212...$, потом $\displaystyle 3$, потом $\displaystyle 1212...$ и в конце ещё одна тройка. Добавленные рёбра будут соединять разные цвета, так как мы сбили цикл.

\begin{center}
\textbf{Задача 5}
\end{center}
Докажем по полной индукции. База $\displaystyle n=2$ - очевидно.

Переход: при $\displaystyle n\geqslant 3$ в графе есть вершина степени хотя бы $\displaystyle 2$. Если после её удаления граф не распадается, то доказано. Иначе выберем одну из компонент связности после удаления этой вершины. В этой компоненте по индукции можно найти требуемую вершину. Доказано

\begin{center}
\textbf{Задача 6}
\end{center}
Если нет пути, то можно разбить: поместим все достижимые из $\displaystyle s$ в $\displaystyle S$, достижимые из $\displaystyle t$ в $\displaystyle T$. Остальные тоже добавим в $\displaystyle S$. Пересечений нет, так как иначе был бы путь.

Если можно разбить, то нет пути: пусть путь есть, тогда в нём есть 2 соседние вершины из разных множеств - противоречие.

\begin{center}
\textbf{Задача 7}
\end{center}
Сделаем топологическую сортировку. Теперь сделаем чтобы из всех меньших вершин в большие вело ребро. Все старые пути остались и могли появиться новые, то есть количество путей в таком графе не меньше, чем в любом другом. Максимальное количество путей между двумя вершинами этого графа будет между первой и последней вершиной и количество путей будет равно $\displaystyle 2^{n-2}$ (так как из любой мы можем перейти в любую, то есть мы можем как угодно выбрать набор посещённых вершин).

\begin{center}
\textbf{Задача 8}
\end{center}
Индукцией по количеству рёбер докажем, что есть Эйлеров цикл во всем рёбрам (что гарантирует сильную связность).

Выберем начальную вершину и будем жадно идти по выходящим рёбрам, пока не зациклимся. После этого удалим цикл из графа. По индукции у компонент связности есть Эйлеровы циклы и каждая компонента соединена с удалённым циклом, поэтому мы можем построить новый цикл по всему начальному графу. 